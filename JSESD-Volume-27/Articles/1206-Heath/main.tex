\documentclass{sig-alternate} % sets document style to sig-alternate
% packages
% typesetting
% \usepackage{dirtytalk} % can be used to typset quotes easier, automatically sets correct quotation marks with \say{content}
% \usepackage{hanging} % hanging paragraphs with \hanging, like in references. doesn't translate to HTML
% \usepackage[defaultlines=3,all]{nowidow} % avoid widows
\usepackage[pdfpagelabels=false]{hyperref} % produce hypertext links, includes backref and nameref
\usepackage{xurl} % defines url linebreaks, loads url package
\usepackage{microtype} % better typography
% \usepackage{textcomp} % for better tildes
% \newcommand{\texttildemid}{\raisebox{0.4ex}{\texttildelow}}
% layout
\usepackage{calc} % so we can do inline math within \setlength
% \usepackage{enumitem} % control layout of itemize, enumerate, description
\usepackage{fancyhdr} % control page headers and footers
% \usepackage{float} % improved interface for floating objects, adds H float
% \usepackage{multicol} % intermix single and multiple column pages
% \textgreek % typeset greek letters in text mode
% language
\usepackage[utf8]{inputenc} % utf8 encoding, wider character set
\usepackage[english]{babel} % multilanguage support
% misc
\usepackage{graphicx} % builds upon graphics package, \includegraphics
%\usepackage{lastpage} % reference number of pages
\usepackage{xcolor} % color extensions
\usepackage[backend=biber, style=apa]{biblatex} % sophisticated bibliographies % necessary for HTML to display author info and date on abstract page
\usepackage{csquotes} % advanced quotations, makes biblatex happy
\usepackage{authblk} % support for footnote style author/affiliation
% tables and figures
%\usepackage{array} % extend array and tabular environments
\usepackage{caption} % customize captions in figures and tables (rotating captions, sideways captions, etc)
%\usepackage{cuted} % allow mixing of \onecolumn and \twocolumn on same page
\usepackage{multirow} % create tabular cells spanning multiple rows
%\usepackage{subfigure} % deprecated, support for manipulation of small figures
%\usepackage{tabularray} % better table construction, does not translate to HTML
%\usepackage{wrapfig} % allows figures or tables to have text wrapped around them
% dummy text
%\usepackage{blindtext} % blind text dummy text
%\usepackage{kantlipsum} % Kant style dummy text
\usepackage{lipsum} % lorem ipsum dummy text

\setlength{\paperheight}{11in}

\pagestyle{fancy} % sets pagestyle to fancy for fancy headers and footers
% allows the header to take the full width of the page https://www.reddit.com/r/LaTeX/comments/awtrb2/how_to_you_make_the_headerfooter_extend_the/
\newlength{\oddmarginwidth}
\setlength{\oddmarginwidth}{1in+\hoffset+\oddsidemargin}
\newlength{\evenmarginwidth}
\setlength{\evenmarginwidth}{\evensidemargin+1in}
\fancyhfoffset[LO,RE]{\oddmarginwidth}
\fancyhfoffset[LE,RO]{\evenmarginwidth}

% header and footer
% modern way to set header image
\renewcommand{\headrulewidth}{0pt} % defines thickness of line under header
\renewcommand{\footrulewidth}{0pt} % defines thickness of line above header
\setlength\headheight{80.0pt} % sets height between top margin and header image, effectively moves page contents down
\addtolength{\textheight}{-80.0pt} % seems to affect the lower height. maybe only works properly if footer numbers enabled?
\fancyhf{}
\fancyhead[CE, CO]{\includegraphics[width=\pdfpagewidth]{headerImage.png}}

\hypersetup{colorlinks=true,urlcolor=blue} % sets link color to blue
\urlstyle{same} % sets url typeface to same as rest of text

% set caption and figure to italics, label bold, left align captions, does not transfer to HTML
\captionsetup{labelfont=bf, font={large, it}, justification=raggedright, singlelinecheck=false}
\renewcommand\theContinuedFloat{\alph{ContinuedFloat}} % has something to do with subfigures... don't remember why i used it

%this next bit is confusing, but essentially changes the width of the abstract. Seems to have been copied from this https://tex.stackexchange.com/questions/151583/how-to-adjust-the-width-of-abstract
\let\oldabstract\abstract
\let\oldendabstract\endabstract
\makeatletter %changes @ catcode to enable modification (in parsep)
\renewenvironment{abstract} %alters the abstract environment
{\renewenvironment{quotation}%
        {\list{}{\addtolength{\leftmargin}{1em} % change this value to add or remove length to the the default ?
            \listparindent 1.5em%
            \itemindent  \listparindent%
            \rightmargin  \leftmargin%
            \parsep    \z@ \@plus\p@}%
        \item\relax}%
        {\endlist}%
\oldabstract}
{\oldendabstract}
\makeatother %changes @ catcode to disable modification

% checks
% italics 
% links
% dashes
% tildes
\begin{document}
\title{Intersectional Experiences of STEM Students with Disabilities}

\author[1]{\large \color{blue} Sandy Heath} % make sure there are no spaces after the author's name
\author[2]{\large \color{blue} Jade Metzger}
\author[3]{\large \color{blue} Tarah Loy-Ashe}

\affil[1]{Northern Arizona University}
\affil[2]{Institute for Human Development at Northern Arizona University}
\affil[3]{Southern Connecticut State University}

\maketitle % prints article title
\begin{@twocolumnfalse} 
\begin{abstract}
\item %the abstract is a quotation and a list, so this must be an item
\begin{large}
\textit{Students with disabilities experience barriers to accessing Science, Technology, Engineering, and Mathematics (STEM) careers during academic preparation for these careers. This pilot project explored the experiences of postsecondary STEM students who have a disability. The six students interviewed gave their insights into the need to further diversify who has access to STEM careers by describing that cultural and structural barriers posed more of a barrier than the course content during their STEM education. Academic preparation often starts in early childhood education with exposure to STEM topics and continues through postsecondary education. The implications impact both the individual student and society. STEM careers often pay at a higher rate and may provide financial security for the individual, but also STEM professionals imagine, design, and build the infrastructure of the world we live in. This was a photovoice project that included data in the form of both digital photographs and narrative interviews from these six postsecondary students enrolled in a STEM major. Their stories illuminate barriers, stigma, and bias they face to attain a STEM career and the facilitators that have encouraged their pursuit.}
\item Keywords: Science, students with disabilities, intersectionality, photovoice

\end{large}   
\end{abstract}
\end{@twocolumnfalse}

%% AUTHOR INFORMATION
\textbf{*Corresponding Author, Sandy Heath}\\ % corresponding author
\href{mailto:s.sandy.heath@nau.edu}{(sandy.heath@nau.edu)} \\ % author email
\textit{Submitted Dec 5 2023} \\ % submitted date
\textit{Accepted Sep 16 2024} \\ % accepted date
\textit{Published Online Apr 10 2025} \\ % published online date, updated after author approval
\textit{DOI: 10.14448/jsesd.16.0007} \\ % doi, updated after author approval, in spreadsheet on server
\pagebreak 
\clearpage %both needed to go to next page
\begin{large}
\section*{INTRODUCTION}
Science, Technology, Engineering, and Mathematics (STEM) departments at universities and colleges are experiencing ongoing challenges with keeping and graduating students from their programs. According to a report from the National Center for Education Statistics STEM fields like biology, chemistry, physics, and math have difficulty retaining students in their programs and universities with 28\% of students switching majors to a non-STEM field and 20\% of students leaving college altogether (Chen \& Soldner, 2013). While the success (or lack thereof) of minoritized and marginalized groups in STEM environments has received increased scrutiny (see the National Science Foundation’s Broadening Participation in STEM program for more), students with disabilities majoring in STEM are experiencing unique challenges towardsSTEM identity development, degree obtainment, and ultimately employment opportunities(Friedensen et al., 2021). The goal of this pilot study was to utilize photo-elicitation interviewing/photovoice methods to empower students in TAPDINTO-STEM to share the formative experiences which informed their choice to major in STEM in college and their most recent experiences with being a STEM major. Through their experiences, we seek to highlight the challenges students with disabilities face and thus generate awareness that can lead to changes that improve the chances for marginalized students to complete STEM degrees and access to STEM professions (Downey, 2020; Lee, 2022).

Despite trends showing increased matriculation in post-secondary education and increased declaration of STEM majors, people with disabilities are not graduating at the same rates, nor with similar STEM degrees as their nondisabled peers (Friedensen et al., 2021). Scholars have suggested a variety of possible reasons for why people with disabilities are “leaking out” of post-secondary education generally, and STEM specifically. According to Thurston et al., (2017) students with disabilities lack the necessary accommodations and/or adaptive aids they would require to be successful in coursework, field world, and laboratory environments. They also argue that faculty and staff lacked the necessary knowledge and training to identify and support students with disabilities in science environments (Thurston et al., 2017). Friedensen et al (2017) argued that academic siloing results in colleges/university STEM administration and faculty placing undue emphasis on disability resources offices as the sole entity responsible for ensuring students with disabilities have full accessibility to campus, campus events, and curriculum. To further complicate circumstances, students with disabilities may also face additional barriers to access due to how their disability experience is colored and shaped by their race, gender, sexual orientation, or immigration status. Disability is only one facet of a person’s identity. Scholarship has shown that people with multiple intersecting identity markers have unique and magnified experience of marginalization (Crenshaw, 1991; Annamma, Connor, \& Ferri 2013; Renken et al., 2020; Ellis-Robinson, 2021). The Alliance for Students with Disabilities for Inclusion, Networking, and Transition Opportunities in STEM (TAPDINTO-STEM) is a National Science Foundation (NSF) Eddie Bernice Johnson INCLUDES Initiative with goals of increasing the number of students completing STEM programs and transitioning into STEM careers. As of September 2023, there are roughly 226 students across 6 regional Hubs in the U.S.A. Student data from the TAPDINTO-STEM Alliance during the 2022-2023 academic school year showed that roughly 40\% of all students reported having mental health disability, 25\% Autism, and 20\% Attention Deficit Hyperactivity (ADHD). Students also reported physical and mobility challenges (9\%), chronic pain conditions (7\%), systemic health conditions (17\%), deafness (5\%), blindness (6\%), and traumatic brain injury (5\%). Most students in the Alliance report majoring in Engineering (26\%) and Biological Sciences (20\%). While TAPDINTO-STEM students are disabled, their experience with disability also intersects with other types of marginalization. A demographic survey of all Alliance participants pulled on August 2024 shows that approximately 31\% participants are male, 55\% are white, and 49\% are explicitly heterosexual. The organizational emphasis on disability and the Alliance’s emphasis on recruiting students from intersectional background, aligned with the goals of our pilot study.

Our team interviewed 6 students from 4 large state universities (University of Missouri – Kansas City, Northern Arizona University, University of Nevada – Reno, and University of Wisconsin – Milwaukee) majoring in 4 different STEM fields (Psychological Sciences, Biology, Chemistry, and Engineering) to learn about their early experiences with STEM and their experiences in post-secondary STEM environments. 

To learn about student intersecting marginalization in STEM, this project used a critical approach to Feminist Disability Theory (FDT), while using Intersectionality as a praxis for research design that included reflexive practice of the research team (Collins et al., 2021; Crenshaw, 1991; Schalk, 2022). There is a growing body of knowledge that works to understand the barriers for people with marginalized identities in postsecondary STEM education, however, there is limited research that looks at the intersectional barriers for students with a disability (Garland-Thomson, 2005; Oliver, 2013; Ray, 2017; Scotch, 2016). One overarching question guided this study, “What are the intersectional experiences of STEM students with disabilities?”

\vspace{1em}

\section*{LITERATURE REVIEW}

The social world is home to diverse experiences, which has led to the development of various theories that interrogate societal norms. Feminist Disability Theory (FDT) is one such approach that exists in Critical Disability Studies, critiquing society on the social origins and implications of devaluing individuals with disabilities. It examines how gendered expectations interact with perceptions of disability to shape individual and collective experiences. Through this lens, we gain insight into the ways in which society's structures marginalize and oppress individuals based on their gender and abilities.

\subsection*{A Critical Framework}

Historically, both women and individuals with a disability have been perceived as other\footnote{ Other or othering is a term used to distinguish in-groups from out-groups related to historical marginalization (Brons, 2015).}, marginalized based on societal definitions of what constitutes normalcy and idealized human functioning (Garland-Thomson, 2005). Feminist theorists have long highlighted how patriarchal systems uphold ideals of male superiority, resulting in gendered oppression. Similarly, disability theorists have pointed out societal biases that favor the able-bodied, leading to systemic discrimination against individuals with a disability. FDT merges these two discourses, revealing the ways in which gender and disability intersect to produce unique forms of marginalization. Feminist disability theorists argue that both disability and femininity have been constructed as deviant and inferior in comparison to able-bodied masculinity (Wendell, 1996). The female body has historically been viewed as weaker and more susceptible to ailments, often leading to protective or restrictive policies that limit women's autonomy. In a similar vein, disabled bodies have been perceived as defective or lacking. FDT posits that these constructions are not neutral; they reflect power dynamics that perpetuate inequalities.

\subsection*{Intersecting Oppressions}

Central to feminist disability theory is the assertion that the lived experiences of women with disabilities cannot be understood by examining their gender and disability in isolation (Wendell, 1996). The oppression faced by women with disabilities is unique and is a result of intersecting societal prejudices against both disability and femininity. Wendell (1989) argues that while all women are oppressed by patriarchal standards of physical ability, disabled women encounter a dual prejudice. They are not only marginalized for not fitting the societal mold of femininity but also for their non-conformance to able-bodied norms.

\subsection*{Body as a Battleground}
The body, especially the female body, has historically been a pivotal concern for feminist theory. For feminist disability theorists, the body becomes even more central. Garland-Thomson (2005) presents the body as a site of social discourse, where meanings are constructed, contested, and reconstructed. In her seminal work, she underscores how disabled bodies challenge prevailing norms and, in doing so, they disrupt established hierarchies of power, providing a powerful critique of patriarchal standards.

Understanding the experiences of women with disabilities through the lens of feminist disability theory holds profound implications for social justice and equality. As Erevelles and Minear (2010) suggest, by acknowledging intersecting oppressions, we can pave the way for more inclusive policies, ensuring that the rights of all women, irrespective of their physical abilities, are protected. By challenging established norms of femininity and ability, feminist disability theory propels society towards a more holistic understanding of identity, moving beyond binaries. Moreover, FDT emphasizes the importance of considering intersectionality in all forms of activism and scholarship. Just as gender and disability intersect to create unique experiences of oppression, so do other identities like race, sexuality, class, and more (Crenshaw, 1991). By embracing a holistic view of identity, activists and scholars can develop more nuanced and effective strategies for promoting social justice. Examining STEM education from an intersectional perspective illuminates culturally relevant constraints which could be ignored when participants identities are flattened to one dimension. Renken et al. (2021)’s exploration of STEM identity development in Black, Deaf/Hard of Hearing, adolescents demonstrated that they disconnected from STEM via formal classroom learning because of the narrowness of the instruction and how disconnected curriculum was from their broader lived experiences. Ellis-Robinson’s (2021) meta-analysis of Renken et al.’s (2021) revealed missed opportunities to criticize how STEM environments perpetuate systemic equities for marginalized students. 

\subsection*{STEM Education}
Students with disabilities are often pushed out of STEM educational environment due to systemic barriers of STEM departments and post-secondary institutions; attitudinal bias of STEM faculty; and the constraints on faculty professional development (Bettencourt et al, 2018). One potential area for improvement would be to place higher emphasis on teaching when hiring for tenure positions at universities and colleges (Wu et al., 2023). The University of California system created teaching focused faculty positions for STEM programs which help to provide consistency and fidelity to curriculum, reduce teaching load across faculty, and fill pedagogy knowledge gaps (Harlow et al, 2022). However, such teaching focused positions are often viewed negatively by faculty who do not want to be perceived as working for a teaching institution and fail to fully integrate teaching focused professors into their organizational cultures (Harlow et al., 2022).

The photovoice method evokes rich narratives, is grounded in critical theories, and provides a vital lens through which we can understand the complexities of identity, power, and oppression. By interrogating the ways in which identity impacts an individual’s personal experiences while academically preparing for a STEM profession, valuable insights are gained into the nuances of structural and societal barriers and the possibilities for activism and change. As society continues to grapple with issues of diversity and inclusion, listening to the perspectives of those who directly experience the ramifications of dominance in society will remain crucial in guiding discussions and actions toward a more just world.

\section*{METHODOLOGY}
Photovoice and photo elicitation interviewing (PEI) are similar strategies for qualitative inquiry, which employ visual artifacts i.e. photos to evoke richer, deeper conversation with participants and both can empower participants within research to reveal meaningful or significant social realities otherwise ignored or overlooked in conventional research approaches (Wang \& Burris, 1997; Clark-Ibáñez, 2004). With PEI, photos used to draw out participant narratives can be taken or selected by either the participant or the researcher and used to prompt a participant’s memory, enrich the overall interview, or aid audience understand of outcomes and results (Bates, McCann, Kaye, \& Taylor, 2017). Photovoice, however, relies exclusively on photos taken by the participants themselves, thereby “enabl[ing] people to record and reflect their community’s strengths and concerns” (Wang \& Burris, 1997, p. 370). The participatory action foundation of photovoice research opens opportunities for transformative and emancipatory results (Evans-Agnew, Rosemberg, \& Boutain, 2022). 

\begin{table*}[ht]
\caption{Self-Identifying Demographics\protect\footnotemark}
\begin{tabular}{|l|l|l|}
\hline
\textbf{Demographic Variable} & \textbf{Category} & \textbf{Frequency (n)} \\ \hline
Ability Identity & ADD/ADHD & 4 \\
 & Autism ASD & 4 \\
 & Deaf/HOH & 1 \\
 & Physical/Orthopedic/Mobility & 2 \\
 & Psychological/Psychiatric & 2 \\
 & Learning Disability & 1 \\
 & TBI/frontal cortex injury & 1 \\ \hline
Academic Major & Biochemistry & 1 \\
 & Civil Engineering & 1 \\
 & Microbiology/immunology & 1 \\
 & Chemistry & 1 \\
 & Psychology/sociology & 2 \\ \hline
Gender & Cisgender Male & 1 \\
 & Cisgender Female & 2 \\
 & Non-binary/third gender & 3 \\
 & Transgender & 1 \\
 & Genderqueer & 1 \\ \hline
Sexual Identity & Bisexual & 3 \\
 & Queer & 4 \\
 & Asexual & 1 \\
 & Pansexual & 1 \\ \hline
Ethnicity & White & 5 \\
 & Black/African American & 1\\ \hline
\end{tabular}
\end{table*}

The use of photographs in scholarship on STEM identity is relatively unique, despite firm establishment of PEI and photovoice methods in humanities and social sciences (Wang \& Burris, 1997; Clark-Ibáñez, 2004; Bates, McCann, Kaye, \& Taylor, 2017). Renken et al. (2021) used photos in their intersectional research on STEM identity to capture evidence of observations. However, they did not employ PEI or photovoice methods as photographs of Black, Deaf or Hard of Hearing adolescents at a STEM summer camp were taken by unidentified volunteers and were not used as part of the researchers’ interviews with campers. The aim of our pilot project was to empower students with disabilities to discuss and document their journey to a STEM major in post-secondary environments and to help others better understand the unique challenges they face. These interviews sought to highlight students’ intersectional realities to better ascertain what challenges they are facing to institutionalize systems of oppression, rather than flattening their experiences to demographic categories. Given the tacit power dynamics between the participants (students) and the researchers (professors/researchers) and the need for increased focus on accessibility in methods/methodologies concerning disability (Hickman \& Serlin, 2019) photovoice was an ideal approach for this pilot project on the experiences of STEM students with disabilities. A link to a recruitment questionnaire was sent to TAPDINTO STEM students. Students who responded to the questionnaire reviewed and signed informed consent and were contacted to schedule an orientation meeting. At the orientation meeting, students learned about photovoice and the time commitment for the project (Simmonds et al., 2015; Wang \& Burris, 1997). Students were invited to complete a demographic questionnaire at the end of the orientation, where they also scheduled a check-in and final interview. Six students participated in the project resulting in approximately ten hours of interview data. Demographic data was collected anonymously through our second survey and available in the Self-Identifying Demographic Table.

\footnotetext{Six students were interviewed, and their answers were included in this table. Students were encouraged to pick more than one answer if it best represented their identity.}

Each student drove the final interview to promote an emancipatory process. Each student was asked a grand tour question to set the intention for the space shared by student and researcher, then students shared and discussed their photos. Each student decided when the interview was complete and confirmed the pseudo\-nym, they preferred to be used in any written dissemination activities. The interviews took place in a private virtual room. Each interview was recorded and saved on a private-locked device and also uploaded on a transcription site. The transcripts were reviewed using a biographical analysis approach. The first round was done as a general scan of each transcript. The second round was a note taking review of the audio. Additional rounds were conducted reviewing transcripts and audio to condense themes that arose in earlier rounds of analysis. As a form of member checking, students were asked to review individual direct quotes included in any publications that use this data.

\section*{FINDINGS}

Findings were consistent with systematic power imbalances that occur in broader society (Collins et al., 2021; Crenshaw, 1991) and were contextualized to STEM programs within postsecondary environments. The details of power imbalance, and how imbalance impedes access to education, varied across student experiences due to intersectional identity. Multiple students noted that it was tough to pinpoint if the stigma, bias, and barriers they encountered regularly were due to their disability identity or another intersecting identity. One student found the entrance to their school was a physical barrier that denied them access to education due to a building that does not meet Americans with Disability Act (ADA) Title II standards; reminiscent to segregation-era barriers that impacted other members of their family. Other students described social barriers of the classroom structure that denied them access to course content. The students discussed their experiences while showing pictures they took throughout the school year. Mushroomcap\footnote{Students were asked to choose a pseudonym to protect their anonymity and were encouraged to not censor their choice (Evans-Agnew, Rosemberg, 2022).} describes a photo she took of two of her wheelchairs.

\begin{quote}
One symbolizes more freedom. It’s hard to get around. The medical school is difficult to get into the building. Inaccessibility adds two hours to my day just to get to class. The next picture is a road with no sidewalk (to access the class building) … it’s dangerous and it interferes with study time and adds to my fatigue. Over the course of the semester, this adds to burn out, burnt out on trying to get to class.
\end{quote}

Mushroomcap gives another example of physical barriers out of compliance with 504 of the Rehabilitation Act and Title II of the Americans with Disabilities Act (ADA) when she shows a picture of a soap dispenser in a school restroom,

\begin{quote}
During the Covid-19 pandemic it felt dangerous for me to go to school, to go to the bathroom. I can’t wash my hands because I can’t reach the soap dispenser.
\end{quote}

While Mushroomcap described visible barriers to accessing education that were sometimes STEM specific and sometimes more broadly applied to the university; Jill spoke about how certain lectures, assignments, and content delivery techniques were inaccessible. Jill discussed their experience as a non-traditional student with a disability that is not visible to STEM instructors.

\begin{quote}
I am asked (as the learner) to adapt to each professor’s teaching style while trying to understand new content. The teacher can more easily adapt their teaching because they are experts on the topics they teach. One professor sent me a message saying, “You won’t adapt to my teaching style, so you are a problem as a student.” It’s a constant battle trying to prove that I am a worthwhile student to invest in.
\end{quote}

Marigold discusses an email exchange with an instructor that offers perspective as to how barriers over the course of a semester may impact student success\footnote{This is an academic definition of success to mean the measured outcome of a letter grade to stay consistent with STEM measurements.}. Marigold states feeling confident with the content covered in the class, however, became fatigued by the end of the semester and lost points to their attendance grade. The instructor validates this in an email.

\begin{quote}
The email from the teacher said, “I know what your grade is, but I don’t think your grade reflects your understanding of the course. I think you understand more than your grade reflects.” This was affirming because maybe people see my abilities, and this gave me confidence. 
\end{quote}

Marigold keys into an important distinguishment in post-secondary education; logistics versus rigor. Why did Marigold’s grade not reflect their grasp of the content? Conversely, are students without disabilities receiving higher grades with a lower understanding of the content because they are positioned better to meet the logistical needs of the classroom? This raised more questions than answers which will require future research.

All students spoke of a power dynamic with instructors in addition to outdated social constructs intertwined with STEM content. Some STEM instructors seemed to believe their science was pure and devoid of human bias\footnote{Purity arguments are often rooted in eugenics and belief in a great race. Butterfly made this insidious connection in their interview (Leidig, 2023; Perliger et al., 2023).}. These students were particularly keen to understand that no science is devoid of human interference since it is the scientist who makes study design choices and interprets what they have found (Stage \& Wells, 2014). Butterfly shared how their experience with this negatively impacted their health.

\begin{quote}
I have emailed professors saying “I am experiencing debilitating symptoms” to try to advocate for myself. I feel like I am shamed for stimming and I have to mask in public and in class. I think perfection is a tool of white supremacy.
\end{quote}

Marigold engages in an email exchange with their instructor after a class that discussed autism spectrum disorder (ASD) and autism treatments. Marigold feels compelled to write the instructor because the topic was presented from a perspective that no neurodiverse people were present in the room. Additionally, the therapeutic intervention discussed for ASD seemed like a type of conversion therapy rather than a celebration of diversity. From an intersectional standpoint, Marigold identifying as a queer person with an ASD diagnosis, was offended by the content and how it was presented in class.

\begin{quote}
If I remember rightly, one of the dudes who came up with it ... There's some overlap between applied behavioral analysis and conversion therapy based on the people who are involved in both. There's a quote from the guy who came up with it, I think, and the quote goes something along the lines of, “You have to pretend that you're working with a bunch of cells. You're trying to mold a person as if they weren't already a person.” It's so weird. I was just sitting there like, whoa. He even showed clips of it. In this lecture, there were clips of how applied behavioral analysis is going with actual kids. And this woman's playing with this little boy and she's making him laugh and stuff. He's having a lot of fun, so he starts flapping his hands because he's having fun, he's expressing joy, and as soon as he flaps his hand, she shuts it down, takes his hands and puts them in his lap. He was stimming. It’s like if you authentically express joy in a way that comes naturally to you that doesn't harm other people, no more play time for you. Hands in your lap. You're not allowed to express joy again unless you do it the way we want you to.
\end{quote}

These six students gifted their most personal stories to this project with the intent to help the next generation of STEM students have an easier road. All students spoke of the very real need, that they personally experienced, to diversify who has access to STEM education. No student in this study described STEM content as a barrier. In fact, all students unanimously described STEM content from a standpoint of curiosity and intrigue. They all seemed to enjoy the rigor and work. It was the classroom logistics, social expectations, and systematic power structures that presented barriers that were (in some cases) detrimental to the students’ health. In a broader social context, there are signs of a homogenous cycle that reproduces itself and repels or even rejects diversity. The academy prepares students for a career and career/practice is researched and understood by the academy. To borrow systems theory from the field of engineering to illustrate the findings of this project, this is a closed system that lacks diversity, and without diversity, it rejects the natural integration of diversity (Adams et al., 2014). These dominant discriminatory forces continue to cycle until diversity is forced into the system. These courageous and steadfast students are the impetus for change in postsecondary STEM education and STEM careers.

\section*{LIMITATIONS AND RECOMMENDATIONS}

This pilot project interviewed a small sample of students from the TAPDINTO-STEM program. The diversity of the sample proved to be a limitation of this pilot study. Student sexual identity and academic majors where adequately diverse, while race and ethnicity lacked diverse representation. This project would benefit from a larger sample size with greater representation to match the TAPDINTO-STEM program demographics.

A lack of diversity impacts how social infrastructure is imagined, designed, and implemented. Students in this study described a negative feedback loop when describing the logistics of accessing STEM education. One notable example of this negative feedback loop was wheelchair access to the science building. There are low reports of architects, engineers, or other design professionals who are wheelchair users, which may lead to a lack of innovation inaccessible design. This example is not exclusive to accessible building and planning design but was reference by all students interviewed in this study across degree majors. Bio-medical student, Mushroomcap, referenced the benefits of her unique life experience as a person with a disability to identify and understand variance in the diseases she studied in her internship. She notes that her disability has given her a perspective as a scientific investigator studying infectious disease that could lead to innovation in this sector of healthcare and health sciences. Further research is needed to identify ways to support a positive feedback loop while reducing a negative feedback loop. We recommend using the methods from this study to complete a large-scale collection of narratives. 

\section*{CONCLUSION}

Consistent with critical theories, we believe these systems that work against postsecondary STEM students with a disability, are part of a vast social matrix of domination and oppression that work to inflect determinant factors upon student success (Collins et al., 2021; Downey, 2020; Lee, 2022; Scotch, 2016). Students may or may not be directly forced out of STEM classes due to their disability, however, all students in this study faced oppressive factors that their counterparts presumably did not (as evident from by curb ramps and soap dispensers that did not meet ADA standards). These factors were said to compound over time resulting in a desire to self-select out of a STEM career track or delay academic progress. Students also reported facilitating experiencing a pivotal relationship with a mentor that guided their interests toward STEM. This relationship was often in the form of a teacher or school program during K-12. Students described their reasons for staying with STEM during the difficult times was an understanding that completing a STEM degree and working as a professional in their field would open doors for the next generation.

\section*{AUTHOR NOTE}
This material is based upon work supported by a sub-award granted by Auburn University through their leadership of the National Science Foundation (NSF) Eddie Bernice Johnson INCLUDES Initiative: The Alliance for Students with Disabilities for Inclusion, Networking, and Transition Opportunities in STEM (TAPDINTO-STEM) program under Grant No 2119902. Any opinions, findings, conclusions, or recommendations expressed in this material are those of the author(s) and do not necessarily reflect the views of the NSF. The authors would like to acknowledge the PIs of this project and the leadership of the TAPDINTO-STEM Mountain Hub for their support.

\end{large}
\clearpage
\section*{REFERENCES}\par 

\leftskip 0.25in
\parindent -0.25in % create hanging indents for references. if article has content after references, set leftskip and parindent to 0

Adams, K. M., Hester, P. T., Bradley, J. M., Meyers, T. J., \& Keating, C. B. (2014). Systems theory as the foundation for understanding systems. \textit{Systems Engineering, 17}(1), 112–123. \url{https://doi.org/10.1002/sys.21255}

Annamma, S. A., Connor, D., \& Ferri, B. (2013). Dis/ability critical race studies (DisCrit): Theorizing at the intersections of race and dis/ability. \textit{Race Ethnicity and Education, 16}(1), 1–31. \url{https://doi. org/10.1080/13613324.2012.730511}

Bates, E. A., McCann, J. J., Kaye, L. K., \& Taylor, J. C. (2017). “Beyond words”: A researcher’s guide to using photo elicitation in psychology. \textit{Qualitative Research in Psychology, 14}(4), 459–481. \url{https://doi.org/10.1080/14780887.2017.1359352}

Bettencourt, G., Kimball, E., \& Wells, R. S. (2018). Disability in postsecondary STEM learning environments: What faculty focus groups reveal about definitions and obstacles to effective support. \textit{Journal of Postsecondary Education and Disability, 31}(4), 383-396. \url{https://files.eric.ed.gov/fulltext/EJ1214251.pdf}

Brons, L. L. (2015). Othering, an analysis. \textit{Transcience, a Journal of Global Studies, 6}(1), 69–90. \url{https://philpapers.org/archive/BROOAA-4.pdf}

Chen, X., \& Soldner, M. (2013). STEM Attrition: College Students' Paths Into and Out of STEM Fields. \textit{National Center for Education Statistics}. U.S. Department of Education. \url{https://nces.ed.gov/pubs2014/2014001rev.pdf}

Clark-Ibáñez, M. (2004). Framing the Social World With Photo-Elicitation Interviews. \textit{American Behavioral Scientist, 47}(12), 1507-1527. \url{https://doi.org/10.1177/0002764204266236}

Collins, P. H., da Silva, E. C. G., Ergun, E., Furseth, I., Bond, K. D., \& Martínez-Palacios, J. (2021). Intersectionality as Critical Social Theory: Intersectionality as Critical Social Theory, Patricia Hill Collins, Duke University Press, 2019. \textit{Contemporary Political Theory, 20}(3), 690–725. \url{https://doi.org/10.1057/s41296-021-00490-0}

Crenshaw, K. (1991). Mapping the Margins: Intersectionality, Identity Politics, and Violence against Women of Color. \textit{Stanford Law Review}. \url{https://doi.org/10.2307/1229039}

Downey, M. \&. (2020). Examining the STEM Climate for Queer Students with Disabilities. \textit{Journal of Postsecondary Education and Disability, 33}(2), 169–181. \url{https://files.eric.ed.gov/fulltext/EJ1273676.pdf}

Ellis-Robinson, T. (2021). Identity development and intersections of disability, race, and STEM: Illuminating perspectives on equity. \textit{Cultural Studies of Science Education, 16}(4), 1149–1162. \url{https://doi.org/10.1007/s11422-020-10011-x}

Erevelles, N., \& Minear, A. (2010). Unspeakable offenses: Untangling race and disability in discourses of intersectionality. \textit{Journal of Literary \& Cultural Disability Studies, 4}(2), 127-145. \url{https://consultspringboard.com/wpcontent/uploads/2020/10/Erevelles_2010_UnspeakableOffenses.pdf}

Evans-Agnew R. A., Rosemberg, M. S., \& Boutain, D. M. (2022). Emancipatory photovoice research: A primer. \textit{Health Promotion Practice, 23}(2), 211–220. \url{https://doi.org/10.1177/15248399211062906}

Friedensen, R., Lauterbach, A., Kimball, E., \& Mwangi, C. G. (2021). Students with High-Incidence Disabilities in STEM: Barriers Encountered in Postsecondary Learning Environments. \textit{Journal of Postsecondary Education and Disability (Print), 34}(1), 77-. \url{https://eric.ed.gov/?id=EJ1308649}

Garland-Thomson, R. (2005). Feminist Disability Studies. \textit{Signs: Journal of Women in Culture and Society, 30}(2), 1557-1587. \url{https://www.jstor.org/stable/10.1086/423352}

Harlow, A. N., Buswell, N. T., Lo, S. M., Sato., B. K., (2022). Stakeholder perspectives on hiring teaching-focused faculty at research-intensive universities. \textit{International Journal of STEM Education. 9}(54), 1-14. \url{https://doi.org/10.1186/s40594-022-00370-y}

Hickman, L. \& Serlin, D. (2019). \textit{Towards a crip methodology. In Interdisciplinary Approaches to Disability: Looking towards the future} (Ellis, K., Garland-Thomson, R., Kent, M., \& Robertson, R.) (pp. 131-141). Routledge.

Lee, A. (2022). A Forgotten Underrepresented Group: Students with Disabilities’ Entrance into STEM Fields. \textit{International Journal of Disability, Development and Education, 69}(4), 1295–1312. \url{https://doi.org/10.1080/1034912X.2020.1767762}

Leidig, E. (2023). \textit{The Women of the Far Right: Social Media Influencers and Online Radicalization}. Columbia University.

Oliver, M. (2013). \textit{The social model of disability: Thirty years on. Disability and Society, 28}(7), 1024–1026. \url{https://doi.org/10.1080/09687599.2013.818773}

Perliger, A., Stevens, C., \& Leidig, E. (2023). Mapping the Ideological Landscape of Extreme Misogyny. \textit{International Centre for Counter-Terrorism}. \url{https://www.icct.nl/sites/default/files/2023-06/Mapping%20the%20Ideological%20Landscape%20of%20Misogyny%20v2.pdf}

Ray, S. J. (2017). \textit{Disability Studies and the Environmental Humanities}. (pp. 29–72). Nebraska Press. \url{https://doi.org/10.2307/j.ctt1p6jht5.5}

Renken, M., Scott, J., Enderle, P., \& Cohen, S. (2021). “It’s not a deaf thing, it’s not a black thing; it’s a deaf black thing”: a study of the intersection of adolescents’ deaf, race, and STEM identities. \textit{Cultural Studies of Science Education, 16}(4), 1105–1136. \url{https://doi.org/10.1007/s11422-021-10023-1}

Schalk, S. (2022). \textit{Black disability politics}. Duke University Press.

Scotch, R. K. (2016). Politics and Policy in the History of the Disability Rights Movement Author ( s ): Richard K . Scotch Source : The Milbank Quarterly , Vol . 67 , Supplement 2 ( Part 2 ). Disability Policy : Restoring Published by : Wiley on behalf of Milbank Memorial Fun. \textit{Fund, Milbank Memorial Quarterly, The Milbank, 67}(Part 2), 380–400.

Simmonds, S., Roux, C., \& Avest, I. Ter. (2015). Blurring the boundaries between photovoice and narrative inquiry: A narrative-photovoice methodology for gender-based research. \textit{International Journal of Qualitative Methods, 14}(3), 33–49. \url{https://doi.org/10.1177/160940691501400303}

Stage, F. K., \& Wells, R. S. (2014). Critical Quantitative Inquiry in Context. \textit{New Directions for Institutional Research, 2013}(158), 1–7. \url{https://doi.org/10.1002/ir.20041}

Thurston, L. P., Shuman, C., Middendorf, B. J., \& Johnson, C. (2017). Postsecondary STEM Education for Students with Disabilities: Lessons Learned from a Decade of NSF Funding. \textit{Journal of Postsecondary Education and Disability. 30}(1), 49-60. \url{https://files.eric.ed.gov/fulltext/EJ1144615.pdf}

Wang, C., \& Burris, M. A. (1997). Photovoice: Concept, methodology, and use for participatory needs assessment. \textit{Health Education \& Behavior, 24}(3), 369–387. \url{https://doi.org/10.1177/109019819702400309}

Wendell, S. (1989). \textit{Towards a Feminist Theory of Disability. 4}(2) (pp. 104-124). Cambridge University Press. (2020). \url{https://doi.org/10.1111/j.1527-2001.1989.tb00576.x}

Wendell, S. (1996). \textit{The rejected body: feminist philosophical reflections on disability}. New York: Routledge. 

Wu, J., Cropps, T., Phillips, C. M. L., Boyle, S., \& Pearson, Y. E. (2023) Applicant qualifications and characteristics in STEM faculty hiring: an analysis of faculty and administrator perspectives. International Journal of STEM Education, 10(41), 1-20. \url{https://doi.org/10.1186/s40594-023-00431-w}

\end{document}
