\documentclass[11.5pt]{sig-alternate}
\usepackage{tabularx}
\usepackage{graphicx}
\usepackage{blindtext}
\usepackage[utf8]{inputenc}
\usepackage[english]{babel}
\usepackage{lastpage}
\usepackage{comment}
\usepackage{dirtytalk}
\usepackage{xcolor}
\usepackage{hanging}
\usepackage{wrapfig}
\usepackage[backend=biber,style=apa]{biblatex}
\addbibresource{notation.bib}
\usepackage{authblk}
\usepackage{caption}
\usepackage{longtable}
\usepackage{graphicx,subfigure}
\usepackage{authblk}
\usepackage{enumitem}
\usepackage[utf8]{inputenc}
\usepackage{cuted}
\usepackage{fancyhdr}
\usepackage{makecell}
\usepackage{xurl}
\pagestyle{fancy}
\usepackage{lipsum}
\usepackage{hyperref}
\renewcommand{\headrulewidth}{0pt}
\renewcommand{\footrulewidth}{0pt}
\setlength\headheight{80.0pt}
\addtolength{\textheight}{-80.0pt}
\usepackage{amssymb}
\setcounter{tocdepth}{2}
\chead{%
  \ifcase\value{page}
  % empty test for page = 0
  \or \includegraphics[width=\textwidth]{headerImage.png}% page = 1
  \or \includegraphics[width=\textwidth]{headerImage.png}% page = 2
  \or \includegraphics[width=\textwidth]{headerImage.png}% page = 3
  \or \includegraphics[width=\textwidth]{headerImage.png}% page = 4
  \or \includegraphics[width=\textwidth]{headerImage.png}% page = 5
  \else
  \includegraphics[width=\textwidth]{headerImage.png}
  \fi
}
%\chead{\includegraphics[width=\textwidth]{headerImage}}
\fancyfoot[LE,LO]{  }
\fancyfoot[CE,CO]{{ }}
\fancyfoot[RE,RO]{\thepage}
\pagenumbering{arabic}
\hypersetup{
    colorlinks=true,
    urlcolor=blue
}

\let\oldabstract\abstract
\let\oldendabstract\endabstract
\makeatletter
\renewenvironment{abstract}
{\renewenvironment{quotation}%
               {\list{}{\addtolength{\leftmargin}{1em} % change this value to add or remove length to the the default
                        \listparindent 1.5em%
                        \itemindent    \listparindent%
                        \rightmargin   \leftmargin%
                        \parsep        \z@ \@plus\p@}%
                \item\relax}%
               {\endlist}%
\oldabstract}
{\oldendabstract}
\makeatother

% Left align captions
\captionsetup{justification   = raggedright,
              singlelinecheck = false}
       
    \begin{document}
\title{From the Editors...}

\author[1]{\large \color{blue} Michele Hollingsworth Koomen}
\author[2]{\large \color{blue} Thomastine A. Sarchet-Maher}
\author[2]{\large \color{blue} Jessica Williams}


\affil[1]{Gustavus Adolphus College}
\affil[2]{Rochester Institute of Technology/National Technical Institute for the Deaf}

\toappear{}

\maketitle

\vspace{5mm}
\begin{large}
\section*{Dear \textit{\textbf{JSESD}} Authors, Readers, and Supporters:}

Happy New Year and thank you for your continued support of the \textit{Journal of Science Education for Students with Disabilities} (\textit{JSESD})! 

\textit{JSESD} remains a venue for the dissemination of research and practice related to the education of students with disabilities in the science classroom and laboratory since 1998. Volumes \#1 through 11 were published in a print format. Starting with Volume \#12, the journal has been published online and Open Access. Having \textit{JSESD} in the Open Access format maximizes access for readers and authors and allows the journal to remain economically sustainable. \textit{JSESD} is proud to now be publishing articles in both PDF and HTML formats (the HTML versions can be accessed through a link from the main articles’ web-page).

The journal enthusiastically seeks new manuscript submissions. We are especially interested in articles on science education for students with varying types of disabilities at a full range of grade levels (K-12 and postsecondary). We are also eager to include articles that represent the full research-to-practice continuum, including articles representative of inclusive pedagogies and research in general science classrooms. While most manuscripts submitted to \textit{JSESD} have historically focused on research, we have recently seen an increase in practitioner, or “Teaching Techniques” articles. We are delighted to see these articles, as they provide practical, ready-to-implement approaches for educators at all levels. We also seek referees who can peer-review \textit{JSESD} manuscript submissions. If you are interested in peer reviewing, please contact us (\href{mailto:michelejkoomen@gmail.com}{michelejkoomen@gmail.com}, \href{mailto:tasbka@rit.edu}{tasbka@rit.edu}, \href{mailto:jwtnmp@rit.edu}{jwtnmp@rit.edu}).

The journal is currently hosted by bepressTM Digital Commons. The journal’s management and production are led by the talented group at Rochester Institute of Technology’s Scholarly Publishing group. A few reminders:

\begin{itemize}
    \item[–] The journal resides online.
    \item[–] Manuscripts should be submitted online at \url{http://scholarworks.rit.edu/jsesd/}. 
    \item[–] There are currently no fees charged to authors for publication in \textit{JSESD}. As an open access journal, articles are also free for anyone to read.
    \item[–] \textit{JSESD} uses a double-blind review of manu\-scripts.
\end{itemize}
\newpage
We are pleased to welcome Drs. Christin Monroe (Landmark College) and Jason Nordhaus (Rochester Institute of Technology/ National Technical Institute for the Deaf) as Associate Editors of \textit{JSESD}. Both bring significant expertise to the journal and will be valuable members of the growing editorial team at \textit{JSESD}.

This year, Dr. Todd Pagano, who has served as the editor of \textit{JSESD} for over 15 years, is passing the baton to new editorial leadership. Todd was instrumental in overhauling the journal to an online and Open Access format (while keeping it free of charge for both authors and readers), increasing indexing of the journal, and revitalizing types of article submissions. We plan to build off the strong foundation he created and look forward to moving the journal in new directions. We thank him for his many years of leadership and service to \textit{JSESD}.

We remain proud of the relationship that \textit{JSESD} has with its partner organization, Science Education for Students with Disabilities (SESD), which is an associated group of the National Science Teaching Association (NSTA). Each year, SESD holds a pre-conference on science and disability at NSTA’s national conference. For further information, please contact Rachel Zimmerman Brachman, SESD Conference Coordinator, at: \href{mailto:Rachel.Zimmerman-Brachman@jpl.nasa.gov}{Rachel.Zimmerman-Brachman@jpl.\\nasa.gov}.

We know that there is a considerable amount of high-quality scholarship that is being conducted in the field of science education for students with disabilities. \textit{JSESD} is proud to serve as a mechanism for the dissemination of such work. As always, we appreciate your support in maintaining \textit{JSESD} as a quality peer-reviewed journal.\\
\newpage
Sincerely,\\

\textit{Michele, Thomastine, and Jessica}\\

\textbf{Michele Hollingsworth Koomen, Ph.D.}\\
Editor, \textit{JSESD}\\
Gustavus Adolphus College\\

\textbf{Thomastine A. Sarchet-Maher, Ed.D.}\\
Editor, \textit{JSESD}\\
Rochester Institute of Technology/\\
National Technical Institute for the Deaf

\textbf{Jessica Williams, Ph.D.}\\
Editor, \textit{JSESD}\\
Rochester Institute of Technology/\\
National Technical Institute for the Deaf

\end{large}
\end{document}
