\documentclass[11.5pt]{sig-alternate} % sets document style to sig-alternate
% packages
% typesetting
%\usepackage{dirtytalk} % typset quotations easier (\say{stuff})
\usepackage{hanging} % hanging paragraphs
\usepackage[defaultlines=3,all]{nowidow} % avoid widows
\usepackage[pdfpagelabels=false]{hyperref} % produce hypertext links, includes backref and nameref
\usepackage{xurl} % defines url linebreaks, loads url package
\usepackage{microtype}
%\usepackage[super]{nth} % easily create superscript ordinal numbers with \nth{x}
\usepackage{textcomp}
\newcommand{\texttildemid}{\raisebox{0.4ex}{\texttildelow}}
% layout
%\usepackage{enumitem} % control layout of itemize, enumerate, description
\usepackage{fancyhdr} % control page headers and footers
\usepackage{float} % improved interface for floating objects
%\usepackage{multicol} % intermix single and multiple column pages
% language
\usepackage[utf8]{inputenc} % accept different input encodings
\usepackage[english]{babel} % multilanguage support
% misc
\usepackage{graphicx} % builds upon graphics package, \includegraphics
%\usepackage{lastpage} % reference number of pages
%\usepackage{comment} % exclude portions of text (?)
\usepackage{xcolor} % color extensions
\usepackage[backend=biber, style=apa]{biblatex} % sophisticated bibliographies % necessary for HTML to display author info and date on abstract page
\usepackage{csquotes} % advanced quotations, makes biblatex happy
\usepackage{authblk} % support for footnote style author/affiliation
% tables and figures
\usepackage{tabularray}
%\usepackage{array} % extend array and tabular environments
\usepackage{caption} % customize captions in figures and tables (rotating captions, sideways captions, etc)
%\usepackage{cuted} % allow mixing of \onecolumn and \twocolumn on same page
\usepackage{multirow} % create tabular cells spanning multiple rows
%\usepackage{subfigure} % deprecated, support for manipulation of small figures
%\usepackage{tabularx} % extension of tabular with column designator "x", creates paragraph-like column whose width automatically expands
%\usepackage{wrapfig} % allows figures or tables to have text wrapped around them
%\usepackage{booktabs} % better rules
% dummy text
%\usepackage{blindtext} % blind text dummy text
%\usepackage{kantlipsum} % Kant style dummy text
\usepackage{lipsum} %lorem ipsum dummy text
% other helpful packages may be booktabs, longtable, longtabu, microtype

\pagestyle{fancy} % sets pagestyle to fancy for fancy headers and footers

% header and footer
% modern way to set header image
\renewcommand{\headrulewidth}{0pt} % defines thickness of line under header
\renewcommand{\footrulewidth}{0pt} % defines thickness of line above header
\setlength\headheight{80.0pt} % sets height between top margin and header image, effectively moves page contents down
\addtolength{\textheight}{-80.0pt} % seems to affect the lower height. maybe only works properly if footer numbers enabled?
\fancyhf{}
\fancyhead[CE, CO]{\includegraphics[width=\textwidth]{headerImage.png}}
% footer
%\fancyfoot[LE,LO]{Article Title Here \\ DOI: }% left footer article title and doi
%\fancyfoot[CE,CO]{{}} % center footer empty
%\fancyfoot[RE,RO]{\thepage} % right footer page numbers
%\pagenumbering{arabic} % arabic (1, 2, 3) numbering in footer

\hypersetup{colorlinks=true,urlcolor=blue} % sets link color to blue
\urlstyle{same} % sets url typeface to same as rest of text

% set caption and figure to italics, label bold, left align captions, does not transfer to HTML
\captionsetup{labelfont=bf, font={large, it}, justification=raggedright, singlelinecheck=false}
\renewcommand\theContinuedFloat{\alph{ContinuedFloat}}

%this next bit is confusing, but essentially changes the width of the abstract. Seems to have been copied from this https://tex.stackexchange.com/questions/151583/how-to-adjust-the-width-of-abstract
\let\oldabstract\abstract
\let\oldendabstract\endabstract
\makeatletter %changes @ catcode to enable modification (in parsep)
\renewenvironment{abstract} %alters the abstract environment
{\renewenvironment{quotation}%
               {\list{}{\addtolength{\leftmargin}{1em} % change this value to add or remove length to the the default ?
                        \listparindent 1.5em%
                        \itemindent    \listparindent%
                        \rightmargin   \leftmargin%
                        \parsep        \z@ \@plus\p@}%
                \item\relax}%
               {\endlist}%
\oldabstract}
{\oldendabstract}
\makeatother %changes @ catcode to disable modification

% checks
% italics -
% links -
% dashes -
% tildes -
% dollars -
\begin{document}

\title{Approaching Undergraduate Research with Students who are Deaf and Hard-of-Hearing}

\author[1]{\large \color{blue}Austin U. Gehret}
\author[1]{\large \color{blue}Jessica W. Trussell}
\author[1]{\large \color{blue}Lea V. Michel}
\affil[1]{Rochester Institute of Technology}

\toappear{}
%% ABSTRACT
\maketitle
\begin{@twocolumnfalse} 
\begin{abstract}
\item 
\textit{An undergraduate research experience can provide a unique opportunity for students to learn and grow as scientists; when positive, this experience is often transformative and motivates students to pursue science, technology, engineering and mathematics (STEM) graduate degrees or careers. Conversely, negative research experiences can sour a student’s opinion of research, propagate misconceptions of graduate school, and lead to attrition from STEM fields. Negative research experiences can be equally devastating for faculty mentors and may result in reluctance to mentor future research students. Using a mentoring approach, which has traditionally translated to positive research experiences for hearing students, may not be as efficacious for mentoring d/Deaf and hard-of-hearing (DHH) research students, particularly when a communication mismatch is at play. Up until recently, most research has focused on how to understand and improve the learning environments for DHH students in the classroom. Here, we present several challenges and strategies associated with the undergraduate research experience for DHH students. The challenges and strategies outlined were derived from a pilot survey administered to DHH students who previously took part in undergraduate research. The preliminary strategies put forth by respondents will inform future mentoring and training efforts with the goals of enriching DHH students’ research experiences and their pursuit of graduate STEM degrees or postgraduate careers in STEM.}
\\ \\
Keywords: Deaf, hard of hearing, bioscience, biochemistry, undergraduate research
\end{abstract}
\end{@twocolumnfalse}

%% AUTHOR INFORMATION

\textbf{*Corresponding Author, Austin U. Gehret}\\
\href{mailto: augnts@rit.edu }{(augnts@rit.edu)} \\
\textit{Submitted Dec 7 2016 }\\
\textit{Accepted Mar 29 2017} \\
\textit{Published online May 22 2017} \\
\textit{DOI:10.14448/jsesd.08.0002} \\
\pagebreak
\clearpage
\begin{large}
  
\section*{INTRODUCTION}

Despite the fact that d/Deaf and hard-of-hearing (DHH) individuals comprise roughly 13\% of the US population, Reilly and Qi (2011) reported a significant difference in college graduation rates between hearing (12.8\% of the hearing population graduated college) and DHH people (5.1\% of the DHH population graduated college) (Reilly \& Qi, 2011). This difference is mirrored in graduate education, with 9.2\% of the hearing population and 4.8\% of the DHH population receiving at least some graduate education (Reilly \& Qi, 2011). While the percentages are lower for the DHH population, a large number of DHH college graduates receive at least some graduate education, suggesting that those who graduate college are highly likely to receive at least some graduate training. The correlation between income and education has been well established (U.S. Department of Labor, n.d.), with particular attention to the economic benefit for DHH individuals with a Bachelor’s degree (Schley et al., 2011). In the interests of furthering workforce diversity and improving the financial welfare of DHH individuals, new initiatives are needed to ensure inclusion of DHH individuals in the professional sector and improve their graduation rates with advanced degrees (e.g. M.S., Ph.D., M.D., etc.). 

\section*{STEM}

New initiatives are desperately needed to increase the U.S. workforce in science, technology, engineering and mathematics (STEM) in order to maintain global competitiveness (President’s Council of Advisors on Science, 2012). Currently, DHH people’s representation in STEM careers is less than that of their hearing counterparts (15.5\% vs. 17.9\%) (Walter, 2010). However, these statistics are somewhat misleading given that DHH representation skews more towards blue collar occupations (e.g., agriculture, construction) while hearing representation is more prevalent in emerging STEM fields (e.g., information technology, health care) with higher degree requirements (Walter, 2010). Interestingly, the largest gap in post-secondary STEM degree attainment for DHH individuals resides at the bachelor’s level (15.5\% DHH vs. 24.9\% hearing) (Walter, 2010), suggesting interventions are desperately needed at this academic level.

DHH students experience barriers to entering STEM fields that are systemic at both the social and educational level. Some DHH students lack access to hearing family conversations delaying incidental life learning (Hauser, O’Hearn, McKee, Steider, \& Thew, 2010; Hopper, 2011), have limited exposure to spoken English translating into struggles with English vocabulary, English sentence structure, and overall world knowledge (Convertino, Borgna, Marschark, \& Durkin, 2014; Sarchet et al., 2014; Wolbers, Dostal, \& Bowers, 2010), and experience a complicated interplay of hearing threshold, socioeconomic status, and language fluency with other variables (Marschark, Shaver, Nagle, \& Newman, 2015). Secondary general education school teachers lack training in deaf education practices while teachers of the deaf often lack sufficient STEM training (Kelly, Lang, \& Pagliaro, 2003). At the postsecondary level, access to accommodations in the classroom to facilitate communication and foster inclusion are not universally available (Powell, Hyde, \& Punch, 2014). This combination of factors results in an “accumulated disadvantage” uniquely experienced by DHH students (Listman, 2013). In comparison, hearing students do not encounter these particular barriers yet nearly half of them switch to non-STEM majors driven in significant part by the intensity of first year STEM coursework coupled with a lack of success in those courses (Chen, 2013). DHH students are disproportionately apt to suffer this same consequence owing to the additional educational barriers they face; consequently, their recruitment and retention in STEM remains a significant hurdle to overcome.

\subsection*{The Undergraduate Research Experience}

The undergraduate research experience is one avenue that may help students to overcome those barriers. Undergraduate research experiences have been shown to increase students’ understanding and awareness of graduate school opportunities, confidence in applying for graduate school, and likelihood of acceptance (Eagan et al., 2013; Russell, Hancock, \& McCullough, 2007; Seymour, Hunter, Laursen, \& DeAntoni, 2004). The “inculcation of enthusiasm” was noted as a key element of undergraduate research that led to a substantial increase in student interest for obtaining a PhD (Russell et al., 2007). Similarly, undergraduate research experiences have been shown to instill higher learning gains in scientific writing, working independently, and self-confidence for underrepresented students compared with other students (Lopatto, 2007).

Of significant value to the undergraduate research experience is mentoring. In particular, students have reflected on how undergraduate research mentoring places faculty members in the role of partner rather than simply instructor (Hunter, Laursen, \& Seymour, 2007). This shared responsibility for inquiry turns the dynamic into one of collaboration (Hunter et al., 2007) and allows the mentor ample opportunity to serve as a role model (Bliska, 2016). Further, role model exposure has been shown to cultivate students’ perceived compatibility with STEM (Shin, Levy, \& London, 2016). DHH students often lack role models in mainstreamed educational settings (Kreimeyer, Crooke, Drye, Egbert, \& Klein, 2000) owing to the systemic educational problems that ultimately yield a lack of DHH STEM faculty (National Science Foundation, National Center for Science and Engineering Statistics, 2017). This could be of particular importance for DHH students who view science as a profession that is not communal and scientists as stereotypes that are incompatible with their own science identities (Gormally \& Marchut, 2017). For this underrepresented population, perceived identity incompatibility with STEM stereotypes is a significant factor in attrition (Good, Rattan, \& Dweck, 2012; Rosenthal, London, Levy, \& Lobel, 2011). Enhancing the number of undergraduate research opportunities afforded to the DHH student population might help alleviate these trends.

\subsection*{Including DHH Students in Laboratory Research}

Literature exists on DHH students’ assimilation into the research lab. DHH high school students were recruited into an assistive technologies lab to assist in the development of American Sign Language (ASL) animations (Huenerfauth, 2010). An undergraduate from Gallaudet University, also skilled in the use of ASL, trained and mentored all students. James Madison University (JMU) has an established summer chemistry research program that specifically targets DHH students, but the program includes an interpreter training component to facilitate communication with the mentees and build the interpreter’s familiarity with chemistry terminology (MacDonald, Seal, \& Wynne, 2002). 

There are key bottlenecks in logistically expanding summer research experiences like JMU’s to the 85\% of DHH students at “mainstream” colleges (Marschark, Sapere, Convertino, \& Pelz, 2008) that lack most of these access services (Solomon, Braun, Kushalnagar, Ladner, \& Painter, 2012). In particular, these DHH students do not receive the necessary advising to complete timely applications to summer programs that, in turn, lowers the overall DHH applicant pool. The lower numbers of DHH applicants thus nullifies any cost-effective economy of scale for the summer program to provide interpreting services (Solomon et al., 2012). Hearing loss is categorized as a low-incidence disability (Individuals with Disabilities Education Improvement Act, 2004) that may require varying accommodations (e.g., sign language interpreters, oral interpreters, FM systems) or strategies (e.g., writing, texting information, ensuring face to face communication) to provide access to spoken communication. Thiry and colleagues (2011) noted that students who worked in isolation commonly reflected negatively on their research experience (Thiry, Laursen, \& Hunter, 2011). Thus, it remains imperative to identify successful methods for advising DHH students in the research lab without these access services in order to increase the number of research opportunities afforded to this geographically diffuse population. 

Some descriptions currently exist (Pagano, Ross, \& Smith, 2015; Smith, Ross, \& Pagano, 2016) for mentoring DHH students within “heterogeneous” communication groups. More specifically, “heterogeneous” situations are defined as DHH students in research laboratories working side-by-side with hearing students, irrespective of the communication mode (e.g., spoken, sign language, spoken and sign language) of the DHH student (Pagano et al., 2015). 

(Is this a new paragraph?)The inclusion of DHH students in “heterogeneous” bioscience research experiences can appear particularly daunting, owing in no small part to the resources available to research advisors or peers to communicate bioscience principles effectively. One primary challenge in learning bioscience is the fact that scientific phenomena are occurring simultaneously at multiple levels of organization (Bahar, Johnstone, \& Hansell, 1999). Additionally, bioscience fields are laced with acronyms and abbreviations, designed to manage technical jargon and to explain scientific phenomena. At the same time, fundamental concepts and ideas are continually changing (Tibell \& Rundgren, 2010). In the classroom, sign language interpreters mitigate the communication barrier, but conceptual accuracy of the content varies widely depending on the interpreter’s expertise, experience, and educational background (Schick, Williams, \& Kupermintz, 2006). Even with highly-qualified sign language interpreters, one study showed that DHH college students learned 59\% of the information compared to 87\% learned by their hearing peers when assessed using written and signed tests (Marschark, Sapere, Convertino, Seewagen, \& Maltzen, 2004). At present, there are no studies to indicate if DHH students’ learning and assimilation into the research lab resembles that of the classroom. In an effort to describe mentoring of DHH students in undergraduate bioscience research, we formulated a pilot study, keeping in mind several questions. What challenges do DHH students face during training in a bioscience research lab? What attitudinal/pedagogical approaches might be needed to effectively mentor DHH students who are involved in bioscience research?

\section*{METHOD}

DHH students who previously participated in undergraduate research were given electronic surveys to describe the challenges they faced as undergraduates and strategies they recommend. Preliminary challenges highlighted from survey responses were further supported through the observations of the faculty advisor who mentored these DHH students. Preliminary strategies detailed by students were based on survey feedback.  

\subsection*{Participants}

Two hearing undergraduate advisors (one knowledgeable in sign language) were involved in this study, but did not take the survey; both were formally trained in bioscience research and have mentored DHH undergraduate research students. Student survey participants (\textit{N} = 4) were advised by the undergraduate advisor who does not sign, had documented hearing loss, one or more years of undergraduate research experience, including at least one full-time summer undergraduate research experience, and secured enrollment in STEM or medical graduate programs. 

\subsection*{Pilot Study Survey Questions}

Upon approval by the Human Subjects Review Office, the students consented to complete an online survey administered post-graduation that asked six questions relating to the duration of their undergraduate research experience, how it has prepared them for graduate school, challenges they faced during undergraduate research, and strategies they would propose to confront those challenges. This paper focuses on the following survey questions: 
\begin{enumerate}
    \item What were the major challenges you faced as part of your undergraduate research experience?
    \item What strategies would you suggest for working with DHH students during an undergraduate research experience?
\end{enumerate}

\section*{RESULTS}

Survey responses from recently graduated DHH students described communication practices as both the source of most challenges and a strategy for how they coped with those challenges. Additional strategies for working with DHH students in the lab were communication-based and directed towards research advisors and hearing peers. Interestingly, some recommendations were also directed to DHH advisees, particularly as it pertained to choosing graduate school advisors. Faculty observations supported the preliminary themes put forth from survey responses.

\subsection*{Challenges in the Bioscience Research Lab: Preliminary Student Responses}

In regards to working in a heterogeneous communication environment, one DHH student specified how missing out on “ambient knowledge” caused the feeling of isolation to persist as an undergraduate researcher. Another DHH student described the challenge of developing a connection with the other students in the lab group.   
\begin{itemize}
    \item \textit{This “isolation” bothered me the most when a person comes to me and say, “Remember how so-and-so did this over the weekend – it was so funny! What, you didn’t know?”, making my “isolation” glaringly obvious and painful.}
    \item \textit{To truly feel like you belong to the lab, you have to develop a connection with the other students. This was hard for me as the students all spoke as (a) group, making it impossible for me to follow along . . . I just didn’t become good friends with my lab partners. I was there to do research and that’s what I did. I had friends outside of lab.}
    \item \textit{The hardest part was probably more related to socialization and incidental learning in the lab. The hearing students are in no way openly negative towards Deaf students, and I don’t want to ever suggest that. However, there is a very real disconnect! All the pleasant conversations, incidental learning, and even aid in running an experiment is lost due to the communication barrier.}
\end{itemize}

Another student commented on the lack of access services and its impact on the training received in the lab.
\begin{itemize}
    \item \textit{It was unrealistic to have an interpreter on standby for every minute of lab work. Having that would have helped me catch all of the “little whys” and would have enriched my experience.}
    \item \textit{I would have liked to get more casual training to pick up on these pesky little whys.}
\end{itemize}

\subsection*{Challenges in the Bioscience Research Lab: Preliminary Faculty Observations}

Similar to students, faculty advisors observed a lack of access to information in the lab. Lack of access manifested in a number of different scenarios in the research lab setting. These manifestations posed additional challenges (i.e., lack of access services, group communication, and peer mentoring) for mentors of DHH students who were engaged in independent research.

\subsubsection*{Lack of access services}
Even at a university that has served as a model provider of technical curricula and support services for DHH students (e.g. interpreting, real-time captioning, peer note-taking), priority for these services is given to classrooms and course-related laboratories. As a result, support services are seldom available in the research lab and present a common scenario for most institutions: a non-signing mentor needing to communicate with a DHH student with little or no means of additional support. For several of the non-signing mentor’s DHH students, the challenge in reserving access services in the research lab was largely a byproduct of the research schedule. Research time was often scheduled ad hoc and varied week to week depending on the experiments that needed to be performed and how they could fit into the students’ schedules. Research time could range from five minutes to initiate bacterial growth to several hours to perform more complex protocols, such as multi-step protein purification. This lack of support in the research lab could also be a source of anxiety for DHH students with strong communication preferences. In one scenario, a student decided not to work in the lab because the research advisor was unable to guarantee an interpreter for every research session. 

\subsubsection*{Group setting communication} Participation during group meetings is an important component to undergraduate research that provides students the opportunity to learn how their project fits into the greater research objectives of the lab. However, the communication dynamics of these meetings have been observed to be extremely isolating. Often, multiple people began talking at once, using both informal and technical language to describe their projects. Even when an experienced interpreter or captioning specialist was available, it was challenging to accurately convey all of the information contained within these simultaneous conversations. It was observed by one faculty advisor that a DHH student was distracted during group meetings when too many simultaneous conversations were taking place. That student would sometimes disengage due to the “chaos” of the room and engage in direct conversation with the interpreter. The advisor’s attention to the DHH student may also be compromised during group discussions. In one observation, the non-signing mentor was trying to differentiate between the concepts of natural protein unfolding and protein denaturation using chemicals or increased temperature. The interpreter initially used the same sign to describe both phenomena but quickly realized the mistake and asked for clarification on the difference between the two concepts. New signs were created to better communicate the difference between the two terms, but this situation would not have happened without the interpreter’s initiative. 

In the heterogeneous research lab, group setting communication posed an additional barrier that is not easily appreciated. Students have extemporaneous discussions to troubleshoot technical issues and often gain valuable insight from these conversations. In the absence of access services, many DHH students were not privy to these valuable learning opportunities. In one anecdotal example, a DHH student performed an enzyme-linked immunosorbent assay (ELISA) from start to finish. The student felt confident in their hands-on training, but the results achieved were highly variable and not reproducible. The student later revealed that a critical procedural detail was omitted. While adding the quenching solution to the ELISA plate, the plate itself was not swirled. Without this movement, the quenching solution did not evenly diffuse throughout the well in a timely manner and the ELISA reaction was artificially prolonged. This small detail was not in the written protocol, but a prime example of how student \textit{ad hoc} discussions in the lab help them troubleshoot technical issues. In another example that demonstrates the value of peer-to-peer discussions, a DHH student in the non-signing mentor’s lab was preparing a polyacrylamide gel for protein sample separation. While the student had the written protocol describing the proper procedure needed to polymerize the gel from its constituent components, the process still needed to be repeated six times before polymerization was achieved. Due to the frustrating circumstances surrounding communicating with other students, the DHH student tended to work and troubleshoot independently. However, the student would have likely benefitted from troubleshooting discussions with other students in the lab had common language or communication modality been possible.

\subsubsection*{Peer mentoring} Peer mentoring is a valuable commodity and time-saving strategy employed by many research advisors. However, in addition to the misconceptions and knowledge gaps introduced by the communication barrier, resentment has also been observed to emerge from peer mentoring. One faculty advisor switched the non-signing, hearing peer mentor of a DHH student because the hearing mentor complained that the mentee was either distracted or incapable of understanding what they were teaching. The DHH mentee also complained that the relationship was strained and that training was suffering as a result. This peer mentor/mentee pair used spoken English to communicate, which meant that the student with hearing loss did not necessarily receive all of the information delivered by the peer mentor. The advisor could sense the resentment building between the two students and decided that a new peer mentor-mentee match would help to alleviate the tension.   

\subsection*{Strategies for Working with DHH Students: Preliminary Student Responses} 

Strategies suggested by DHH students focused on the need to investigate both the personality traits of an advisor and the communication environment of a lab. In particular, one student highlighted that the use of whiteboards should be part of the lab’s communication strategy. Several students echoed this sentiment more generally by underscoring the value of lab groups that are open to communication strategies and thus more amicable to DHH research students.
\begin{itemize}
    \item \textit{Find a lab that is more open to uncanonical (sic) methods of communication. The labs that are open to these modes of communication tend to welcome the deaf/HOH student and allow him/her to participate more.}
    \item \textit{I noticed the “isolation” was significantly less when the lab members assertively participate more in sign language, body language or writing on paper/typing on computer with me.}
\end{itemize}

Feelings of isolation were also addressed by a student who suggested that research advisors take on more than one deaf student at a time.
\begin{itemize}
    \item \textit{I would suggest that you accept as many Deaf persons at a time so that they receive opportunities to do research, find their passion, and achieve their dreams. This is also so they have each other to communicate with in their first language if they want to while socializing.}
\end{itemize}

In addition to communication, students placed emphasis on the personality traits of the research advisor and good advising practices they should consider for DHH advisees.
\begin{itemize}
    \item \textit{Pick an advisor...who is organized and explained/listed the expectations CLEARLY… arranges meetings ahead of time (enough time to request interpreters or transcribers).}
    \item \textit{Extra patience from the advisor is MANDATORY! It will NOT work out if the deaf/HOH student picks an advisor who is brilliant but refuses to give extra time to the student to work on something.}
    \item \textit{Make sure to keep the (communication) line open with all deaf students as they all have very different needs.}
    \item \textit{It is the DHH student’s responsibility to request access services when they need them, but make sure there are no impromptu meetings with critical information about their project. These discussions need to be planned so they can get access services and receive all the information discussed about their project . . . a written email after every lab meeting was perfect for DHH students so that they are informed even if there was an issue with communication during the meeting.}
\end{itemize}

\section*{DISCUSSION}

The gains of undergraduate research identified by Hunter et al. (2007) place an emphasis on the professional socialization of students into the scientific community. Implicit in this development is the ability of faculty advisors to effectively train and mentor student researchers closely when they begin, but also to transition them to become more independent thinkers later on (Hunter et al., 2007). Training DHH research students to become professional scientists ostensibly proceeds in a similar manner, but lack of access services and a mentor’s inability to communicate in a common language present as additional barriers to this development. Based on this pilot study, there are some important challenges and potential strategies to keep in mind that might help inform an approach to mentoring DHH research students. 

Access services are unlikely to be availed under most circumstances owing to the \textit{ad hoc} nature of independent research (Pagano et al., 2015). In one case, we observed this to influence a student’s decision not to join a research project. However, we have witnessed far more often that DHH students rise to the challenge. Though a small sample, DHH survey respondents consistently stated that missed information, in the form of social topics or technical discussions, is the most common challenge they encountered during their undergraduate research experiences. It is important for research advisors to recognize the limitations their traditional training techniques may have when working with DHH students. Simultaneously explaining and demonstrating a protocol to a DHH student, even when speaking loudly and annunciating clearly, may be ineffective owing to the student’s need to focus their attention on either the mentor’s lips or the demonstration itself. Students working in homogeneous communication environments have the benefit of troubleshooting with peers to help fill in their knowledge gaps of protocols. DHH students seldom have the opportunity to work in homogeneous communication environments and lack full access to experimental information. Systemic errors have been observed to manifest in students’ techniques, even when written protocols were provided. A survey respondent commented about getting additional training on “these pesky little whys” and how having access services could have helped “catch all of the little whys.” These comments point to a shared frustration by DHH students that these technical details are the hardest to access in a lab with heterogeneous communication.

\subsection*{Implications for Mentors and Peers}

Survey respondents highlighted flexible communication and research mentors that maintain a high degree of organization and clarity with expectations as valuable strategies. Though the following strategies have yet to be formally assessed, they are practices currently used by either or both mentors that may be supportive of mitigating some of the barriers identified in the current survey. 

\subsubsection*{Flexible communication in the lab environment} Flexible communication can take many forms ranging from simple practices to more involved strategies that are planned in advance. Regardless, there ought to be a pre-agreement on what communication strategies will be utilized that best fit the mentor and mentee working together. First, any advisor or peer that is willing to communicate more expressively using body language (e.g., gesturing, mimicking a procedure) or visual cues is making oneself more accessible to the DHH student. A set of survival signs for repeatedly used techniques or terms could also be obtained from a number of online ASL STEM dictionaries compiled by Solomon et al. (2012). Second, written forms of communication (e.g., whiteboards, laptops, notepads, etc.) allow greater discussion and invite more questions from the DHH student. Illustration using whiteboards to facilitate interactions between hearing and DHH students in academic settings has been shown as an effective practice (Marchetti, Foster, Long, \& Stinson, 2012). Its use in the research environment also provides another mechanism for two-way written communication as well as the opportunity for the mentor to fashion and revise proposed protocols with the DHH advisee in real-time. Dictation software might offer another mechanism to facilitate real-time communication with DHH students, but user calibration requirements have been shown to be cumbersome and time consuming in these situations (Kheir \& Way, 2007). 

Planned communication strategies need to occur ahead of experimentation. If detailed, written protocols can be provided ahead of time, DHH students have more opportunity to familiarize themselves with the procedures. An advisor should understand that DHH students may require more time to learn certain procedures and concepts owing to the mismatch in communication modes (e.g., signing DHH student and non-signing mentor or DHH student whose first language is American Sign Language and a mentor whose first language is English). While written protocols can help alleviate this mismatch, technical nuances within experiments are often still overlooked. The creation of captioned video tutorials allows DHH students to learn the techniques at their own pace, replaying sections as needed. These videos could be viewed repeatedly before and after students learn new techniques in the research lab to reaffirm that new knowledge. Additionally, video tutorials hold potential to provide DHH mentees access to a level of technical detail not easily attainable through written means; potentially, this resource may help alleviate some of the troubleshooting issues observed. While both of these proposed strategies would be time consuming to develop, they would likely benefit the training of hearing students and could be time-saving resources in the long term. 

\subsubsection*{Necessary qualities of potential research advisors and peer mentors} DHH students express a preference to work with advisors that are well organized, clear with expectations, and, most critically, patient. A mentoring approach that might reflect these attributes involves structuring a research project with additional time for the student to review objectives associated with each step. While the effectiveness of this strategy has yet to be assessed through survey, it has been implemented to mentor a DHH student through the completion of a molecular cloning project (see Table 1). As outlined previously, designing research projects with clear objectives and a measurable end product can help DHH students develop confidence in the lab (Pagano et al., 2015). While this practice adds more time to a project, it allows both parties the opportunity to assess for the student’s knowledge gaps in the scientific phenomena invoked at each step. An undergraduate research advisor should encourage their DHH students to ask potential graduate advisors about their mentoring practices during their graduate school interviews. Our survey findings indicated that DHH students likely value these personality traits in an advisor as much as they do common research interests.

\begin{table*}[th]
\caption{Schedule of Sample Semester Research Project}
\begin{tabular}{ccc}
\hline
\textbf{Week\textsuperscript{a}} & Activity & Learning objective \\ \hline
1 & Reviewed DNA mutagenesis strategy; designed mutagenic DNA primers & Reading DNA sequences \\
2 & Performed mutagenesis reaction & \multirow{2}{=}{Dilution calculations} \\
3 & Made bacterial growth plates for mutagenic DNA clone isolation & \\
4 & Introduced mutagenic DNA clones into bacteria (Transformation) for isolation & \multirow{2}{=}{Bacterial transformation} \\
5 & Grew bacterial liquid cultures of isolated mutagenic DNA clones & \\
6-7 & Extraction and purification of mutagenic DNA clones & Alkaline lysis for plasmid DNA purification \\
7 & Reviewed DNA plasmid maps; ran enzymatic digestion of mutagenic DNA clones & Restriction digestion analysis, selection of experimental controls \\
8 & Made agarose gel; electrophoresed restriction digests & Percent calculations, agarose gel electrophoresis, DNA quantitation \\
9 & Repeat Week 7 procedure with experimental modifications & \multirow{2}{=}{Adjusting experimental design based on preliminary results} \\
10 & Repeat Week 8 procedure with experimental modifications & \\ \hline
\end{tabular}
\textsuperscript{a}The weekly activities did not always occur contiguously. Research project was conducted over a 15-week semester schedule.
\end{table*}

It is unrealistic to expect that all undergraduate research advisors can devote the amount of time that might be ideal to fully mitigate the information gaps created by communication barriers; rather, peer mentoring will remain a strategy most research advisors need to employ. As articulated by Pagano et al. (2015), peer mentors of DHH students should be experienced and confident with procedures, but we would add patience is equally important. Peer mentors are less experienced in teaching and likely to have greater misconceptions about what they feel has been effectively communicated to their mentee. An impatient peer mentor might become frustrated by the additional time needed to communicate effectively with a DHH student. From the DHH student perspective, particularly when they are the lone DHH person in the lab, outwardly projected impatience can further fuel feelings of isolation or segregation that perpetuate doubt, insecurity, and their own incompatibility with STEM. An advisor should ensure peer mentors are equipped to use the communication strategies previously described and plan to meet regularly with both students to ensure a positive training environment persists for both parties. 

\section*{LIMITATIONS AND FUTURE DIRECTIONS}

The present study has several limitations. First, the survey had a low number of participants. Although four out of five (Nate really didn’t qualify as he is not in a graduate program, but I suppose we are stuck leaving this in, right?) possible participants participated in the survey, the limited number of responses gathered only gives us an initial look at the experiences of undergraduate DHH students in a research lab at a single institution. Second, the small sample size precluded anonymity. Potentially, the DHH participants may have been less or more forthcoming if the number of participants allowed for anonymity. Also, our findings may be heavily biased by particular attitudes or personality traits of the individual students and mentors. Future researchers may decide to contact mentors or lab advisors at other universities across the United States to determine if there are potential DHH participants and mentors willing to share information about their experiences. Third, the survey responses initiated ideas for proposed strategies to improve the undergraduate research experience for DHH individuals. Future researchers may empirically test these strategies to determine if the strategies do or do not improve undergraduate research experiences. 

\section*{CONCLUSIONS}

Increasing DHH student representation in the STEM fields is critical to diversify the STEM workforce in the United States and to provide lucrative career opportunities for DHH graduates. One strategy toward accomplishing this goal is to create positive undergraduate research experiences for DHH students. DHH students report feeling isolated and frustrated in the lab setting when working with non-signing hearing peers. They also describe missing out on “ambient knowledge” when hearing peers do not make the effort to diversify their communication methods. Research mentors have observed numerous scenarios that highlight these instances and made recommendations based on survey feedback to improve the current situation. For example, mentors suggest using flexible communication methods in the lab, such as white boards, and describe the importance of prior planning of experiments and thoughtful matching of peer mentors with DHH mentees. Future research will empirically test these recommendations to determine their effectiveness in labs that include DHH and hearing students.

\section*{ACKNOWLEDGEMENTS}

This study was supported by the Camille and Henry Dreyfus Special Grant Program in the Chemical Sciences and the Rochester Institute of Technology. 

\end{large}
\include{} 
\section*{ References}\par 

\leftskip 0.25in
\parindent -0.25in 
Bahar, M., Johnstone, A. H., \& Hansell, M. H. (1999). Revisiting learning difficulties in biology. \textit{Journal of Biological Education, 33}(2), 84–86. \url{https://doi.org/10.1080/00219266.1999.9655648}

Bliska, J. B. (2016). The importance of role models in research. \textit{PLOS Pathog, 12}(6), e1005426. \url{https://doi.org/10.1371/journal.ppat.1005426}

Chen, X. (2013). \textit{STEM attrition: College students’ paths into and out of STEM fields} (No. NCES 2014-001). Washington, DC: National Center for Education Statistics, Institute of Education Sciences, U.S. Department of Education. Retrieved from \url{http://nces.ed.gov/pubs-2014/2014001rev.pdf}

Convertino, C., Borgna, G., Marschark, M., \& Durkin, A. (2014). Word and world knowledge among deaf learners with and without cochlear implants. \textit{Journal of Deaf Studies and Deaf Education, 19}(4), 471–483. \url{https://doi.org/10.1093/deafed/enu024}

Eagan, M. K., Hurtado, S., Chang, M. J., Garcia, G. A., Herrera, F. A., \& Garibay, J. C. (2013). Making a difference in science education: The impact of undergraduate research programs. \textit{American Educational Research Journal, 50}(4), 683–713. \url{https://doi.org/10.3102/0002831213482038}

Good, C., Rattan, A., \& Dweck, C. S. (2012). Why do women opt out? Sense of belonging and women’s representation in mathematics. \textit{Journal of Personality and Social Psychology, 102}(4), 700–717. \url{https://doi.org/10.1037/a0026659}

Gormally, C., \& Marchut, A. (2017). “Science is not my thing”: Exploring deaf non-science majors’ science identities. \textit{Journal of Science Education for Students with Disabilities, 20}(1). Retrieved from \url{http://scholarworks.rit.edu/jsesd/vol20/iss1/1}

Hauser, P. C., O’Hearn, A., McKee, M., Steider, A., \& Thew, D. (2010). Deaf epistemology: Deafhood and deafness. \textit{American Annals of the Deaf, 154}(5), 486–492. \url{https://doi.org/10.1353/aad.0.0120}

Hopper, M. J. (2011). \textit{Positioned as bystanders: Deaf students’ experiences and perceptions of informal learning phenomena} (Doctoral dissertation). Retrieved from \url{http://gradworks.umi.com/34/58/3458642.html}

Huenerfauth, M. (2010). Participation of high school and undergraduate students who are deaf in research on American Sign Language animation. \textit{SIGACCESS Accessibility and Computing}, (97), 14–24. \url{https://doi.org/10.1145/1873532.1873534}

Hunter, A.-B., Laursen, S. L., \& Seymour, E. (2007). Becoming a scientist: The role of undergraduate research in students’ cognitive, personal, and professional development. \textit{Science Education, 91}(1), 36–74. \url{https://doi.org/10.1002/sce.20173}

Individuals with Disabilities Education Improvement Act, Pub. L. No. 108–446 (2004).

Kelly, R. R., Lang, H. G., \& Pagliaro, C. M. (2003). Mathematics word problem solving for deaf students: A survey of practices in grades 6-12. \textit{The Journal of Deaf Studies and Deaf Education, 8}(2), 104–119. \url{https://doi.org/10.1093/deafed/eng007}

Kheir, R., \& Way, T. (2007). Inclusion of deaf students in computer science classes using real-time speech transcription. In \textit{Proceedings of the 12th Annual SIGCSE Conference on Innovation and Technology in Computer Science Education} (pp. 261–265). New York, NY, USA: ACM. \url{https://doi.org/10.1145/1268784.1268860}

Kreimeyer, K. H., Crooke, P., Drye, C., Egbert, V., \& Klein, B. (2000). Academic and social benefits of a co-enrollment model of inclusive education for deaf and hard-of-hearing children. \textit{Journal of Deaf Studies and Deaf Education, 5}(2), 174–185. \url{https://doi.org/10.1093/deafed/5.2.174}

Listman, J. (2013). \textit{Nature of deaf mentoring dyads: Role of subjugated knowledge} (Doctoral dissertation). Retrieved from \url{http://fisherpub.sjfc.edu/education\_etd/149}

Lopatto, D. (2007). Undergraduate research experiences support science career decisions and active learning. \textit{CBE Life Sciences Education, 6}(4), 297–306. \url{https://doi.org/10.1187/cbe.07-06-0039}

MacDonald, G., Seal, B. C., \& Wynne, D. H. (2002). Deaf students, teachers, and interpreters in the chemistry lab. \textit{Journal of Chemical Education, 79}(2), 239–243. \url{https://doi.org/10.1021/ed079p239}

Marchetti, C., Foster, S., Long, G., \& Stinson, M. (2012). Crossing the communication barrier: Facilitating communication in mixed groups of deaf and hearing students. \textit{Journal of Postsecondary Education and Disability, 25}(1), 51–63.

Marschark, M., Sapere, P., Convertino, C., \& Pelz, J. (2008). Learning via direct and mediated instruction by deaf students. \textit{The Journal of Deaf Studies and Deaf Education, 13}(4), 546–561. \url{https://doi.org/10.1093/deafed/en-n014}

Marschark, M., Sapere, P., Convertino, C., Seewagen, R., \& Maltzen, H. (2004). Comprehension of sign language interpreting: Deciphering a complex task situation. \textit{Sign Language Studies, 4}(4), 345–368.

Marschark, M., Shaver, D. M., Nagle, K. M., \& Newman, L. A. (2015). Predicting the academic achievement of deaf and hard-of-hearing students from individual, household, communication, and educational factors. \textit{Exceptional Children, 81}(3), 350–369. \url{https://doi.org/10.1177/0014402914563700}

National Science Foundation, National Center for Science and Engineering Statistics. (2017). Women, minorities, and persons with disabilities in science and engineering: 2017 (No. NSF 17-310). Arlington, VA. Retrieved from \url{www.nsf.gov/statistics/wmpd/}

Pagano, T., Ross, A., \& Smith, S. B. (2015). Undergraduate research involving deaf and hard-of-hearing students in interdisciplinary science projects. \textit{Education Sciences, 5}(2), 146–165. \url{https://doi.org/10.3390/educsci5020146}

Powell, D., Hyde, M., \& Punch, R. (2014). Inclusion in postsecondary institutions with small numbers of deaf and hard-of-hearing students: Highlights and challenges. \textit{Journal of Deaf Studies and Deaf Education, 19}(1), 126–140. \url{https://doi.org/10.1093/deafed/ent035}

President’s Council of Advisors on Science. (2012). \textit{Engage  to  excel:  Producing  one  million additional college graduates with degrees in  science,  technology,  engineering,  and mathematics}. Retrieved from \url{https://www.whitehouse.gov/sites/default/files/microsites/ostp/pcast-engage-to-excel-final_2-25-12.pdf}

Reilly, C., \& Qi, S. (2011). \textit{Snapshot of deaf and hard of hearing people, postsecondary attendance and unemployment}. Gallaudet Research Institute. Retrieved from \url{https://research.gallaudet.edu/Demographics/deaf-employment-2011.pdf}

Rosenthal, L., London, B., Levy, S. R., \& Lobel, M. (2011). The roles of perceived identity compatibility and social support for women in a single-sex STEM program at a co-educational university. \textit{Sex Roles, 65}(9–10), 725–736. \url{https://doi.org/10.1007/s11199-011-9945-0}

Russell, S. H., Hancock, M. P., \& McCullough, J. (2007). Benefits of undergraduate research experiences. \textit{Science, 316}(5824), 548–549. \url{https://doi.org/10.1126/science.1140384}

Sarchet, T., Marschark, M., Borgna, G., Convertino, C. M., Sapere, P., \& Dirmyer, R. (2014). Vocabulary knowledge of deaf and hearing postsecondary students. \textit{Journal of Postsecondary Education and Disability, 27}(2), 161–178. \url{https://doi.org/10.1055/s-0029-1237430}

Schick, B., Williams, K., \& Kupermintz, H. (2006). Look who’s being left behind: Educational interpreters and access to education for deaf and hard-of-hearing students. \textit{Journal of Deaf Studies and Deaf Education, 11}(1), 3–20. \url{https://doi.org/10.1093/deafed/enj007}

Schley, S., Walter, G. G., Weathers, R. R., Hemmeter, J., Hennessey, J. C., \& Burkhauser, R. V. (2011). Effect of postsecondary education on the economic status of persons who are deaf or hard of hearing. \textit{Journal of Deaf Studies and Deaf Education, 16}(4), 524–536. \url{https://doi.org/10.1093/deafed/enq060}

Seymour, E., Hunter, A.-B., Laursen, S. L., \& DeAntoni, T. (2004). Establishing the benefits of research experiences for undergraduates in the sciences: First findings from a three-year study. \textit{Science Education, 88}(4), 493–534. \url{https://doi.org/10.1002/sce.10131}

Shin, J. E. L., Levy, S. R., \& London, B. (2016). Effects of role model exposure on STEM and non-STEM student engagement. \textit{Journal of Applied Social Psychology, 46}(7), 410–427. \url{https://doi.org/10.1111/jasp.12371}

Smith, S. B., Ross, A. D., \& Pagano, T. (2016). Chemical and biological research with deaf and hard-of-hearing students and professionals: Ensuring a safe and successful laboratory environment. \textit{Journal of Chemical Health and Safety, 23}(1), 24–31. \url{https://doi.org/10.1016/j.jchas.2015.03.002}

Solomon, C. M., Braun, D., Kushalnagar, R., Ladner, R. E., \& Painter, R. (2012). \textit{Workshop for emerging deaf and hard of hearing scientists} (White paper). Washington, D.C.: Gallaudet University.

Thiry, H., Laursen, S. L., \& Hunter, A.-B. (2011). What experiences help students become scientists?: A comparative study of research and other sources of personal and professional gains for STEM undergraduates. \textit{The Journal of Higher Education, 82}(4), 357–388. \url{https://doi.org/10.1353/jhe.2011.0023}

Tibell, L. A. E., \& Rundgren, C.-J. (2010). Educational challenges of molecular life science: Characteristics and implications for education and research. \textit{CBE Life Sciences Education, 9}(1), 25–33. \url{https://doi.org/10.1187/cbe.08-09-0055}

U.S. Department of Labor. (n.d.). Earnings and unemployment rates by educational attainment. Retrieved October 6, 2016, from \url{http://www.bls.gov/emp/ep\_chart\_001.htm}

Walter, G. G. (2010). \textit{Deaf and hard-of-hearing students in transition: Demographics with an emphasis on STEM education}. Retrieved from National Technical Institute for the Deaf, Center on Access Technology website: \url{http://www.ntid.rit.edu/sites/default/files/cat/Transition demographic report 6-1-10.pdf}

Wolbers, K., Dostal, H., \& Bowers, L. (2010). “I was born full deaf.” Written language outcomes after 1 year of strategic and interactive writing instruction. \textit{Journal of Deaf Studies and Deaf Education, 17}(1), 19–38. \url{https://doi.org/10.1093/deafed/enr018}

\end{document}