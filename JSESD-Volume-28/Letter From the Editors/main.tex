\documentclass{sig-alternate} % sets document style to sig-alternate
% packages
% typesetting
% \usepackage{dirtytalk} % can be used to typset quotes easier, automatically sets correct quotation marks with \say{content}
% \usepackage{hanging} % hanging paragraphs with \hanging, like in references. doesn't translate to HTML
% \usepackage[defaultlines=3,all]{nowidow} % avoid widows
\usepackage[pdfpagelabels=false]{hyperref} % produce hypertext links, includes backref and nameref
\usepackage{xurl} % defines url linebreaks, loads url package
\usepackage{microtype} % better typography
% \usepackage{textcomp} % for better tildes
% \newcommand{\texttildemid}{\raisebox{0.4ex}{\texttildelow}} % creates a text mid tilde, a low tilde moved up
% layout
\usepackage{calc} % so we can do inline math within \setlength
\usepackage{enumitem} % control layout of itemize, enumerate, description
\usepackage{fancyhdr} % control page headers and footers
% \usepackage{float} % improved interface for floating objects, adds H float
% \usepackage{multicol} % intermix single and multiple column pages
% \textgreek % typeset greek letters in text mode
% language
\usepackage[utf8]{inputenc} % utf8 encoding, wider character set
\usepackage[english]{babel} % multilanguage support
% misc
\usepackage{graphicx} % builds upon graphics package, \includegraphics
%\usepackage{lastpage} % reference number of pages
\usepackage{xcolor} % color extensions
\usepackage[backend=biber, style=apa]{biblatex} % sophisticated bibliographies % necessary for HTML to display author info and date on abstract page
\usepackage{csquotes} % advanced quotations, makes biblatex happy
\usepackage{authblk} % support for footnote style author/affiliation
% tables and figures
%\usepackage{array} % extend array and tabular environments
\usepackage{caption} % customize captions in figures and tables (rotating captions, sideways captions, etc)
%\usepackage{cuted} % allow mixing of \onecolumn and \twocolumn on same page
\usepackage{multirow} % create tabular cells spanning multiple rows
%\usepackage{subfigure} % deprecated, support for manipulation of small figures
\usepackage{tabularray} % better table construction, does not translate to HTML
%\usepackage{wrapfig} % allows figures or tables to have text wrapped around them
% dummy text
%\usepackage{blindtext} % blind text dummy text
%\usepackage{kantlipsum} % Kant style dummy text
\usepackage{lipsum} % lorem ipsum dummy text

\setlength{\paperheight}{11in}

\pagestyle{fancy} % sets pagestyle to fancy for fancy headers and footers. remember to change the header depending on article type
% allows the header to take the full width of the page https://www.reddit.com/r/LaTeX/comments/awtrb2/how_to_you_make_the_headerfooter_extend_the/
\newlength{\oddmarginwidth}
\setlength{\oddmarginwidth}{1in+\hoffset+\oddsidemargin}
\newlength{\evenmarginwidth}
\setlength{\evenmarginwidth}{\evensidemargin+1in}
\fancyhfoffset[LO,RE]{\oddmarginwidth}
\fancyhfoffset[LE,RO]{\evenmarginwidth}

% header and footer
% modern way to set header image
\renewcommand{\headrulewidth}{0pt} % defines thickness of line under header
\renewcommand{\footrulewidth}{0pt} % defines thickness of line above header
\setlength\headheight{80.0pt} % sets height between top margin and header image, effectively moves page contents down
\addtolength{\textheight}{-80.0pt} % seems to affect the lower height. maybe only works properly if footer numbers enabled?
\fancyhf{}
\fancyhead[CE, CO]{\includegraphics[width=\pdfpagewidth]{headerImage.png}}

\hypersetup{colorlinks=true,urlcolor=blue} % sets link color to blue
\urlstyle{same} % sets url typeface to same as rest of text

% set caption and figure to italics, label bold, left align captions, does not transfer to HTML
\captionsetup{labelfont=bf, font={large, it}, justification=raggedright, singlelinecheck=false}
\renewcommand\theContinuedFloat{\alph{ContinuedFloat}} % has something to do with subfigures... don't remember why i used it

%this next bit is confusing, but essentially changes the width of the abstract. Seems to have been copied from this https://tex.stackexchange.com/questions/151583/how-to-adjust-the-width-of-abstract
\let\oldabstract\abstract
\let\oldendabstract\endabstract
\makeatletter %changes @ catcode to enable modification (in parsep)
\renewenvironment{abstract} % alters the abstract environment
{\renewenvironment{quotation}% alters the quotation environment in the abstract environment ?
               {\list{}{\addtolength{\leftmargin}{1em} % change this value to add or remove length to the the default ?
                        \listparindent 1.5em%
                        \itemindent    \listparindent%
                        \rightmargin   \leftmargin%
                        \parsep        \z@ \@plus\p@}%
                \item\relax}%
               {\endlist}%
\oldabstract}
{\oldendabstract}
\makeatother %changes @ catcode to disable modification

\newenvironment{references} % creates a new environment that uses hanging indentation for sources. handy when there's content after the references
    {
    \leftskip 0.25in
    \parindent -0.25in
    }
    {
    \leftskip 0in
    \parindent 0in
    }
\begin{document}
\title{From the Editors...}

\author[1]{\large \color{blue} Michele Hollingsworth Koomen} % make sure there are no spaces after the author's name
\author[2]{\large \color{blue} Thomastine A. Sarchet-Maher}
\author[ ]{\large \color{blue} Jessica Williams}

\affil[1]{Gustavus Adolphus College}
\affil[2]{Rochester Institute of Technology/National Technical Institute for the Deaf}
\toappear{} % the sig.alternate document type includes a copyright warning that appears at the bottom of the first page. This makes that not appear/be empty. Don't ask my why it's there in the first place /shrug

\maketitle % prints article title

\begin{large}
\section*{Dear \textit{\textbf{JSESD}} Authors, Readers, and Supporters:}
Happy New Year and thank you for your continued support of the \textit{Journal of Science Education for Students with Disabilities (JSESD)}!

\textit{JSESD} remains a venue for the dissemination of research and practice related to the education of students with disabilities in the science classroom and laboratory since 1998. Volumes \#1 through 11 were published in a print format. Starting with Volume \#12, the journal has been published online and Open Access. Having \textit{JSESD} in the Open Access format maximizes access for readers and authors and allows the journal to remain economically sustainable.

The journal enthusiastically seeks new manuscript submissions. We are especially interested in articles on science education for students with varying types of disabilities at a full range of grade levels (K-12 and postsecondary). We are also eager to include articles that represent the full research-to-practice continuum, including articles representative of inclusive pedagogies and research in general science classrooms. While most manuscripts submitted to \textit{JSESD} have historically focused on research, we have recently seen an increase in practitioner, or “Teaching Techniques” articles. We are delighted to see these articles, as they provide practical, ready-to-implement approaches for educators at all levels. We also seek referees who can peer-review \textit{JSESD} manuscript submissions. If you are interested in peer reviewing, please contact us (\href{mailto:michelejkoomen@gmail.com}{michelejkoomen@gmail.com}, \href{mailto:tasbka@rit.edu}{tasbka@rit.edu}, \href{mailto:jessicaw.phd@gmail.com}{jessicaw.phd@gmail.com}).

The journal is currently hosted by bepress\texttrademark Digital Commons. The journal’s management and production are led by the talented group at Rochester Institute of Technology’s Scholarly Publishing group. A few reminders:

\begin{itemize}[label=--]
    \item The journal resides online.
    \item Manuscripts should be submitted online at \url{http://scholarworks.rit.edu/jsesd/}.
    \item There are currently no fees charged to authors for publication in \textit{JSESD}. As an open access journal, articles are also free for anyone to read.
    \item \textit{JSESD} uses a double-blind review of manuscripts
\end{itemize}

We are pleased to welcome Drs. Millicent Carmouche Burks (University of South Alabama) and Laura N. Sarchet (Niagara University) as Associate Editors of \textit{JSESD}. Both bring significant expertise to the journal and will be valuable members of the growing editorial team at \textit{JSESD}.

We remain proud of the relationship that \textit{JSESD} has with its partner organization, Science Education for Students with Disabilities (SESD), which is an associated group of the National Science Teaching Association (NSTA). Each year, SESD holds a pre-conference on science and disability at NSTA’s national conference. For further information, please contact Rachel Zimmerman Brachman, SESD Conference Coordinator, at: \href{mailto:Rachel.Zimmerman-Brachman@jpl.nasa.gov}{Rachel.Zimmerman-Brachman@jpl.nasa.gov}.

We know that there is a considerable amount of high-quality scholarship that is being conducted in the field of science education for students with disabilities. \textit{JSESD} is proud to serve as a mechanism for the dissemination of such work. As always, we appreciate your support in maintaining \textit{JSESD} as a quality peer-reviewed journal.

Sincerely,

Michele, Thomastine, and Jessica

Michele Hollingsworth Koomen, Ph.D.

Editor, \textit{JSESD}

Thomastine A. Sarchet-Maher, Ed.D.

Editor, \textit{JSESD}

Jessica Williams, Ph.D.

Editor, \textit{JSESD}

\end{large}
\end{document}
