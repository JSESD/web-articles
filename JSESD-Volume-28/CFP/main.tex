\documentclass{sig-alternate} % sets document style to sig-alternate
% packages
% typesetting
% \usepackage{dirtytalk} % can be used to typset quotes easier, automatically sets correct quotation marks with \say{content}
% \usepackage{hanging} % hanging paragraphs with \hanging, like in references. doesn't translate to HTML
% \usepackage[defaultlines=3,all]{nowidow} % avoid widows
\usepackage[pdfpagelabels=false]{hyperref} % produce hypertext links, includes backref and nameref
\usepackage{xurl} % defines url linebreaks, loads url package
\usepackage{microtype} % better typography
% \usepackage{textcomp} % for better tildes
% \newcommand{\texttildemid}{\raisebox{0.4ex}{\texttildelow}} % creates a text mid tilde, a low tilde moved up
% layout
\usepackage{calc} % so we can do inline math within \setlength
\usepackage{enumitem} % control layout of itemize, enumerate, description
\usepackage{fancyhdr} % control page headers and footers
% \usepackage{float} % improved interface for floating objects, adds H float
\usepackage{multicol} % intermix single and multiple column pages
% \textgreek % typeset greek letters in text mode
% language
\usepackage[utf8]{inputenc} % utf8 encoding, wider character set
\usepackage[english]{babel} % multilanguage support
% misc
\usepackage{graphicx} % builds upon graphics package, \includegraphics
%\usepackage{lastpage} % reference number of pages
\usepackage{xcolor} % color extensions
\usepackage[backend=biber, style=apa]{biblatex} % sophisticated bibliographies % necessary for HTML to display author info and date on abstract page
\usepackage{csquotes} % advanced quotations, makes biblatex happy
\usepackage{authblk} % support for footnote style author/affiliation
% tables and figures
%\usepackage{array} % extend array and tabular environments
\usepackage{caption} % customize captions in figures and tables (rotating captions, sideways captions, etc)
%\usepackage{cuted} % allow mixing of \onecolumn and \twocolumn on same page
\usepackage{multirow} % create tabular cells spanning multiple rows
%\usepackage{subfigure} % deprecated, support for manipulation of small figures
\usepackage{tabularray} % better table construction, does not translate to HTML
%\usepackage{wrapfig} % allows figures or tables to have text wrapped around them
% dummy text
%\usepackage{blindtext} % blind text dummy text
%\usepackage{kantlipsum} % Kant style dummy text
\usepackage{lipsum} % lorem ipsum dummy text

\setlength{\paperheight}{11in}

\pagestyle{fancy} % sets pagestyle to fancy for fancy headers and footers. remember to change the header depending on article type
% allows the header to take the full width of the page https://www.reddit.com/r/LaTeX/comments/awtrb2/how_to_you_make_the_headerfooter_extend_the/
\newlength{\oddmarginwidth}
\setlength{\oddmarginwidth}{1in+\hoffset+\oddsidemargin}
\newlength{\evenmarginwidth}
\setlength{\evenmarginwidth}{\evensidemargin+1in}
\fancyhfoffset[LO,RE]{\oddmarginwidth}
\fancyhfoffset[LE,RO]{\evenmarginwidth}

% header and footer
% modern way to set header image
\renewcommand{\headrulewidth}{0pt} % defines thickness of line under header
\renewcommand{\footrulewidth}{0pt} % defines thickness of line above header
\setlength\headheight{80.0pt} % sets height between top margin and header image, effectively moves page contents down
\addtolength{\textheight}{-80.0pt} % seems to affect the lower height. maybe only works properly if footer numbers enabled?
\fancyhf{}
\fancyhead[CE, CO]{\includegraphics[width=\pdfpagewidth]{headerImage.png}}

\hypersetup{colorlinks=true,urlcolor=blue} % sets link color to blue
\urlstyle{same} % sets url typeface to same as rest of text

% set caption and figure to italics, label bold, left align captions, does not transfer to HTML
\captionsetup{labelfont=bf, font={large, it}, justification=raggedright, singlelinecheck=false}
\renewcommand\theContinuedFloat{\alph{ContinuedFloat}} % has something to do with subfigures... don't remember why i used it

%this next bit is confusing, but essentially changes the width of the abstract. Seems to have been copied from this https://tex.stackexchange.com/questions/151583/how-to-adjust-the-width-of-abstract
\let\oldabstract\abstract
\let\oldendabstract\endabstract
\makeatletter %changes @ catcode to enable modification (in parsep)
\renewenvironment{abstract} % alters the abstract environment
{\renewenvironment{quotation}% alters the quotation environment in the abstract environment ?
               {\list{}{\addtolength{\leftmargin}{1em} % change this value to add or remove length to the the default ?
                        \listparindent 1.5em%
                        \itemindent    \listparindent%
                        \rightmargin   \leftmargin%
                        \parsep        \z@ \@plus\p@}%
                \item\relax}%
               {\endlist}%
\oldabstract}
{\oldendabstract}
\makeatother %changes @ catcode to disable modification

\newenvironment{references} % creates a new environment that uses hanging indentation for sources. handy when there's content after the references
    {
    \leftskip 0.25in
    \parindent -0.25in
    }
    {
    \leftskip 0in
    \parindent 0in
    }
\begin{document}
\begin{large}
\twocolumn[
\begin{@twocolumnfalse}
\section*{Call For Manuscripts!}
\subsection*{Journal of Science Education for Students with Disabilities (JSESD)}

The \textit{Journal of Science Education for Students with Disabilities} is a peer-reviewed, open access, multi-disciplinary online journal with an international focus. We publish articles that reflect the best of research and practice related to inclusive science education, including works submitted by science and special education researchers, teacher educators, and teachers. Interesting topics have included innovative curricular ideas, instructional adaptations, research-based modifications, best practices, and management strategies in science education. \textit{JSESD} adopts the philosophical perspective that \textbf{all} students can achieve in science, and that science benefits from the full participation of \textbf{all} students.

Below are types of articles that we are interested in publishing (this is not an exhaustive list):
\begin{itemize}
    \item Research Articles (Theoretical or Empirical)
    \item Position Papers
    \item Teaching Techniques Reflecting Classroom Best Practices
    \item Product Reviews
    \item Perspectives
    \item Conference Proceedings
\end{itemize}

\subsection*{Submission:}
\begin{itemize}
    \item Manuscripts should be submitted online at \url{http://scholarworks.rit.edu/jsesd/}.
    \item Title page with author information, including names, titles, affiliations, addresses, email, phone.
    \item Electronic copy of the manuscript in Microsoft Word™.
    \item Style must follow American Psychological Association (APA) 6th guidelines.
    \item Maximum page length: 25 double-spaced pages including references.
    \item The manuscript submitted to \textit{JSESD} must not have been submitted, either in part or wholly, elsewhere.
\end{itemize}

\subsection*{Contact Information:}

Michele Hollingsworth Koomen, Ph.D.\\
Editor, \textit{JSESD}\\
\href{mailto:michelejkoomen@gmail.com}{michelejkoomen@gmail.com}

Thomastine A. Sarchet-Maher, Ed.D.\\
Editor, \textit{JSESD}\\
\href{mailto:tasbka@rit.edu}{tasbka@rit.edu}

Jessica Williams, Ph.D.\\
Editor, \textit{JSESD}\\
\href{mailto:jwtnmp@rit.edu}{jwtnmp@rit.edu}

\end{@twocolumnfalse}]

\end{large}
\end{document}
