\documentclass[11.5pt]{sig-alternate} % sets document style to sig-alternate
% packages
% typesetting
%\usepackage{dirtytalk} % typset quotations easier (\say{stuff})
\usepackage{hanging} % hanging paragraphs
\usepackage[defaultlines=3,all]{nowidow} % avoid widows
\usepackage[pdfpagelabels=false]{hyperref} % produce hypertext links, includes backref and nameref
\usepackage{xurl} % defines url linebreaks, loads url package
\usepackage{microtype}
%\usepackage[super]{nth} % easily create superscript ordinal numbers with \nth{x}
\usepackage{textcomp}
\newcommand{\texttildemid}{\raisebox{0.4ex}{\texttildelow}}
% layout
%\usepackage{enumitem} % control layout of itemize, enumerate, description
\usepackage{fancyhdr} % control page headers and footers
\usepackage{float} % improved interface for floating objects
%\usepackage{multicol} % intermix single and multiple column pages
% language
\usepackage[utf8]{inputenc} % accept different input encodings
\usepackage[english]{babel} % multilanguage support
% misc
\usepackage{graphicx} % builds upon graphics package, \includegraphics
%\usepackage{lastpage} % reference number of pages
%\usepackage{comment} % exclude portions of text (?)
\usepackage{xcolor} % color extensions
\usepackage[backend=biber, style=apa]{biblatex} % sophisticated bibliographies % necessary for HTML to display author info and date on abstract page
\usepackage{csquotes} % advanced quotations, makes biblatex happy
\usepackage{authblk} % support for footnote style author/affiliation
% tables and figures
\usepackage{tabularray}
%\usepackage{array} % extend array and tabular environments
\usepackage{caption} % customize captions in figures and tables (rotating captions, sideways captions, etc)
%\usepackage{cuted} % allow mixing of \onecolumn and \twocolumn on same page
\usepackage{multirow} % create tabular cells spanning multiple rows
%\usepackage{subfigure} % deprecated, support for manipulation of small figures
%\usepackage{tabularx} % extension of tabular with column designator "x", creates paragraph-like column whose width automatically expands
%\usepackage{wrapfig} % allows figures or tables to have text wrapped around them
%\usepackage{booktabs} % better rules
% dummy text
%\usepackage{blindtext} % blind text dummy text
%\usepackage{kantlipsum} % Kant style dummy text
\usepackage{lipsum} %lorem ipsum dummy text
% other helpful packages may be booktabs, longtable, longtabu, microtype

\pagestyle{fancy} % sets pagestyle to fancy for fancy headers and footers

% header and footer
% modern way to set header image
\renewcommand{\headrulewidth}{0pt} % defines thickness of line under header
\renewcommand{\footrulewidth}{0pt} % defines thickness of line above header
\setlength\headheight{80.0pt} % sets height between top margin and header image, effectively moves page contents down
\addtolength{\textheight}{-80.0pt} % seems to affect the lower height. maybe only works properly if footer numbers enabled?
\fancyhf{}
\fancyhead[CE, CO]{\includegraphics[width=\textwidth]{headerImage.png}}
% footer
%\fancyfoot[LE,LO]{Article Title Here \\ DOI: }% left footer article title and doi
%\fancyfoot[CE,CO]{{}} % center footer empty
%\fancyfoot[RE,RO]{\thepage} % right footer page numbers
%\pagenumbering{arabic} % arabic (1, 2, 3) numbering in footer

% deprecated way
%\renewcommand{\headrulewidth}{0pt}
%\renewcommand{\footrulewidth}{0pt}
%\setlength\headheight{80.0pt}
%\addtolength{\textheight}{-80.0pt}
%\chead{%
%  \ifcase\value{page}
%  % empty test for page = 0
% \or \includegraphics[width=\textwidth]{headerImage.png}% page=1
%  \or \includegraphics[width=\textwidth]{headerImage.png}% page = 2
%  \or \includegraphics[width=\textwidth]{headerImage.png}% page = 3
%  \or \includegraphics[width=\textwidth]{headerImage.png}% page = 4
%  \or \includegraphics[width=\textwidth]{headerImage.png}% page = 5
%  \else
%  \includegraphics[width=\textwidth]{headerImage.png}
%  \fi
%}
%\chead{\includegraphics[width=\textwidth]{headerImage.png}}

\hypersetup{colorlinks=true,urlcolor=blue} % sets link color to blue
\urlstyle{same} % sets url typeface to same as rest of text

% set caption and figure to italics, label bold, left align captions, does not transfer to HTML
\DeclareCaptionFormat{custom}
{
    \textbf{\textit{\large #1#2}}\textit{\large #3} % #1 is the "Table 1" or "Figure 1" part, #2 is the separator (":"), #3 is the caption
}
\captionsetup{format=custom}
\captionsetup{justification = raggedright, singlelinecheck = false}

%this next bit is confusing, but essentially changes the width of the abstract. Seems to have been copied from this https://tex.stackexchange.com/questions/151583/how-to-adjust-the-width-of-abstract
\let\oldabstract\abstract
\let\oldendabstract\endabstract
\makeatletter %changes @ catcode to enable modification (in parsep)
\renewenvironment{abstract} %alters the abstract environment
{\renewenvironment{quotation}%
               {\list{}{\addtolength{\leftmargin}{1em} % change this value to add or remove length to the the default ?
                        \listparindent 1.5em%
                        \itemindent    \listparindent%
                        \rightmargin   \leftmargin%
                        \parsep        \z@ \@plus\p@}%
                \item\relax}%
               {\endlist}%
\oldabstract}
{\oldendabstract}
\makeatother %changes @ catcode to disable modification

% checks
% italics 
% links
% dashes
% tildes
\begin{document}

\title{Publishing Successful Practitioner Manuscripts for the Journal of Science Education for Students with Disabilities}

\author[1]{\large \color{blue}Jonte C. Taylor}

\affil[1]{Pennsylvania State University}

\toappear{}
%% ABSTRACT
\maketitle
\begin{@twocolumnfalse} 
\begin{abstract}
\item 
\textit {The Journal of Science Education for Students with Disabilities (JSESD) is the premier journal focusing on the intersections of science education for students with disabilities. JSESD provides valuable content and context for teachers and researchers on what works in advancing science access, practices, and knowledge for all students across settings, grades, ages, and exceptionality. One way in which JSESD supports teachers and researchers is through publication of practitioner manuscripts also referred to as Teaching Techniques. These manuscripts focus on the how-to portion of science education. That is, JSESD practitioner publications give detailed information on how-to provide science instruction or how-to implement instructional strategies or supports, hence Teaching Techniques. The purpose of this paper is the provide guidance to authors on what to include (or not include) in Teaching Techniques practitioner manuscript submissions to JSESD for successful publication.} \\ \\
Keywords: science, students with disabilities, practitioners, instruction, strategies, supports
\end{abstract}
\end{@twocolumnfalse}

%% AUTHOR INFORMATION

\textbf{*Corresponding Author, Jonte C. Taylor }\\
\href{mailto: jct215@psu.edu }{(jct215@psu.edu)} \\
\textit{Submitted November 15, 2020 }\\
\textit{Accepted January 4, 2021} \\
\textit{Published online  April 23 2021} \\
\textit{DOI:0.14448/jsesd.13.0002} \\
\pagebreak
\clearpage
\begin{large}

\section*{INTRODUCTION}
The Journal of Science Education for Students with Disabilities (JSESD) is the premier journal focusing on the intersections of science education for students with disabilities. The purpose of the JSESD is to provide a forum for presentation of research and exemplary practice in the field of science education as it relates to teaching students with disabilities.  JSESD provides valuable content and context for teachers and researchers on what works in advancing science access, practices, and knowledge for all students across settings, grades, ages, and exceptionality.  A main goal of the JSESD is to publish reports of research. While JSESD mainly publishes research, other types of manuscripts are encouraged and accepted.  One way in which JSESD supports teachers and researchers is through publication of practitioner manuscripts or “Teaching Techniques” article as they are referred to in JSESD.  These manuscripts focus on the \textit{how-to} portion of science education.  That is, JSESD’s Teaching Techniques are practitioner publications that give detailed information on \textit{how-to provide science instruction or how-to implement instructional strategies or supports.}  The purpose of this paper is the provide guidance to authors on what to include (or not include) in Teaching Technique practitioner manuscript submissions to JSESD for successful publication.  Journals with similar goals (i.e., practitioner supports) have provided detailed guidance for potential authors including \textit{Teaching Exceptional Children} (Ludlow \& Dieker, 2013; Sayeski, 2018), \textit{Beyond Behavior} (Mooney \& Ryan, 2018), and \textit{Science Scope} (National Science Teaching Association [NSTA], n.d.). This guide will focus on three core areas:
\begin{enumerate}
\item considerations for writing practitioner man-uscripts for JSESD,
\item 	types of practitioner manuscripts for JSESD, and
\item 	formatting practitioner manuscripts for JSESD.
\end{enumerate}

\section*{CONSIDERATIONS}
When writing a practitioner piece for JSESD, there are a number of considerations that need to be taken into account.  Practitioner paper should be grounded in solid theory and evidence of effectiveness.  That is, the core elements of the suggested practice should have previous studies or research with operational procedures and that showed improvement with students with disabilities (Cook et al., 2013). Mooney and Ryan (2018) identified multiple elements of writing practitioner manuscripts including:
\begin{itemize}
    \item a strong rationale for the intervention or process being described;
    \item a brief summary (with references) that support the intervention or process; and
    \item a detailed description of the steps or process for implementation or teaching.
\end{itemize}

Practitioner manuscripts for JSESD need a firm foundation in research along with actionable procedures that can followed by practitioners regardless of setting or student population (Sayeski, 2018).  

\section*{TYPES OF MANUSCRIPTS}

JSESD is interested in two types of Teaching Techniques or \textit{how-to} practitioner articles: a) how-to provide science instruction in a particular topic (e.g., environmental education, analyzing data, water cycle) and b) how-to implement evidence-based strategies for science learning (e.g., using graphic organizers, peer-assisted learning, discrete trial teaching). Manuscripts that focus on a particular topic should include specifics about that topic and details that can provide high fidelity of instruction and/or implementation.  Manuscript that detail strategies to apply during science instruction should include a detailed example of the strategy use and detailed step-by-step implementation instructions.  Each type of manuscript should be developed from and reflect current research regarding the teaching and learning of science or the strategic method that is the focus of the manuscript (NSTA, n.d.).  

\section*{FORMATTING MANUSCRIPTS}

The main body of your manuscript should be between 8-20 double space pages using 12-point Arial or Times New Roman font.  References, captions, and other supplementary text are not included in the word count. The remainder of formatting should be followed as described on the JSESD’s author guidelines (\url{https://scholar-works.rit.edu/jsesd/author\_guidelines.pdf}). Sayeski (2018) identified a number of elements (high-utility elements) that should be included in practitioner papers that increases the usability of a published manuscript (checklist, forms, guiding questions, screenshots, step-by-step directions, examples, and/or list of resources). Practitioner manuscripts submitted to JSESD should include a table that should indicate the following: grade level, disability specific focus (as appropriate), subject or content focus, Next Generation Science Standards (NGSS) content: standard, science and engineering practices, disciplinary core ideas, crosscutting concepts.  See Table 1 for an example.  Manuscripts can optionally include a vignette that helps readers understanding. Scenarios or vignettes, if included, should describe how “practices might be implemented with one or more individuals or in different contexts” (Sayeski, 2018, p. 117). If scenarios or vignettes are included, they should be woven throughout the manuscript and formatted in italics at every point they are revisited (See Table 2 for a brief vignette example).  Scenarios and vignettes should help introduce the lesson, activity, or strategy; provide context to what is being discussed in the manuscript; and provide resolution of success for the students and/or teacher.  Authors should use headings and subheadings as appropriate with all tables and figures placed after References section.  It is also recommended that authors use the JSESD Teaching Techniques Practitioner Manuscript Checklist (See Table 3) to provide additional guidance prior to submitting manuscript for review.  The JSESD Teaching Techniques Practitioner Manuscript Checklist is not necessary to complete, but is suggested to be used as a guide to ensure that all components of the manuscript are included to increase the likelihood of a quality submission and thus increased likelihood of possible publication.  At publication of your Teaching Techniques manuscript, your publication will be designated as such (as opposed to Research Manuscript).

\section*{CONCLUDING REMARKS}
JSESD strives to be an excellent source for research and practice focused on science education for students with exceptionalities.  By publishing Teaching Techniques practitioner manuscripts in JSESD, you can contribute to increases access and inclusion in science classrooms and beyond.  You are also providing support to others that and even those you may never meet.  As stated by Ludlow \& Dieker (2013) regarding publishing practitioner journals, you can “have a far-ranging effect on how learners with exceptionalities are taught for years to come (p. 65).
\clearpage

\begin{table*}[!hbp]
\caption{Manuscript Information and NGSS Connection Table Example}
\begin{tabular}{|l|l|l|l|l|l|}
\hline
\textbf{Grade Level:} & \multicolumn{2}{l|}{Middle School (7th grade)} & \textbf{Subject:} & \multicolumn{2}{l|}{Physical Science} \\ \hline
\textbf{Disability Focus:} & \multicolumn{5}{l|}{Emotional/Behavioral Disorders (EBD)} \\ \hline
\textbf{Strategy Focus:} & \multicolumn{5}{l|}{Classwide Peer Tutoring (CWP)} \\ \hline
\textbf{Safety Precautions:} & \multicolumn{5}{l|}{N/A} \\ \hline
\textbf{NGSS Standard:} & \multicolumn{5}{l|}{MS-PS3-2. Develop a model to describe that when the arrangement of objects interacting at a distance changes, different amounts of potential energy are stored in the system} \\ \hline
\multicolumn{2}{|c|}{\textbf{Science and Engineering Practices}} & \multicolumn{2}{|c|}{\textbf{Disciplinary Core Ideas}} & \multicolumn{2}{|c|}{\textbf{Crosscutting Concepts}} \\ \hline
\multicolumn{2}{|l|}{{\textit{Engaging in Argument from Evidence} Engaging in argument from evidence in 6–8 builds on K–5 experiences and progresses to constructing a convincing argument that supports or refutes claims for either explanations or solutions about the natural and designed worlds}} & \multicolumn{2}{|l|}{{\textit{PS3.C: Relationship Between Energy and Forces} When two objects interact, each one exerts a force on the other that can cause energy to be transferred to or from the object. (MS-PS3-2)}} & \multicolumn{2}{|l|}{{\textit{Systems and System Models} Models can be used to represent systems and their interactions – such as inputs, processes, and outputs – and energy and matter flows within systems. (MS-PS3-2)}} \\ \hline
\end{tabular}
\end{table*}

\begin{table*}
\caption{Brief Vignette Example}
\begin{tabular}{|l|}
\hline
\textit{Mrs. James reviewed the concept maps of the students in her Behavior Support class.  All of her students seemed to have misconceptions about energy and types of energy.  As she wondered what the best ways were to help her students understand energy, she decided to speak with the Physical Science teacher I her unit Mrs. Treatman.  After a discussion with Mrs. Treatman, Mrs James felt like she had a game plan and some great activities and opportunities for research and exploration for her students to learn about energy.} \\ \hline
\end{tabular}
\\ \\
\textit{Note}. The above vignette is for introducing the context for the intervention, process, strategy, or activity
\end{table*}

\begin{table*}
\caption{JSESD Teaching Techniques Practitioner Manuscript Checklist}
\begin{tabular}{l}
\hline
\multicolumn{1}{c}{\textbf{JSESD Teaching Techniques Practitioner Manuscript Checklist}} \\ \hline
\textit{Manuscript includes the following:} \\ \hline
focus \textbf{specifically} on students with disabilities or exceptionalities \\
focus on how-to teach specific content, process, or subject (i.e., Earth science)* \\
focus on how-to implement strategy or technique (i.e., graphic organizers)* \\
\textbf{all} of the following: grade level, subject/content, disability focus \\
\textbf{all} required NGSS information \\
(\textit{optional}) essential pre-existing knowledge, time required, cost** \\
(\textit{optional}) intervention/strategy focus** \\
information regarding safety precautions (if applicable)** \\
use of \textbf{one or more} of the following: table, chart, picture, graph, figure, drawing, etc. \\ \hline
Note. * = must identify one or the other; ** = optional for the information table. \\
\end{tabular}
\end{table*}
\end{large}
\clearpage

\section*{REFERENCES}\par 
\leftskip 0.25in
\parindent -0.25in 

Cook, B. G., Cook, L., \& Landrum, T. L. (2013). Moving research into practice: Can we make dissemination stick? \textit{Exceptional Children, 79}, 163–180.

Ludlow, B. L., \& Dieker, L. (2013). How to write for teaching exceptional children. \textit{Teaching Exceptional Children, 45}(6), 58-65.

Mooney, P., \& Ryan, J. B. (2018). Editor recommendations for successfully writing for beyond behavior. \textit{Beyond Behavior, 27}(2), 116-123.

National Science Teaching Association (n.d.). Guidelines for authors: Science scope. \url{https://www.nsta.org/guidelines-authors-science-scope}

Sayeski, K. L. (2018). How (and Why) to write for teaching exceptional children. \textit{Teaching Exceptional Children, 50}(3), 115-122.

\end{document}