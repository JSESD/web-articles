\documentclass[11.5pt]{sig-alternate}
\usepackage[defaultlines=3,all]{nowidow}
\usepackage{hyperref}
\usepackage[utf8]{inputenc}
\usepackage[english]{babel}
\usepackage{dirtytalk}
\usepackage{hanging}
\usepackage{wrapfig}
\usepackage[backend=biber, style=apa]{biblatex}
\addbibresource{notation.bib}
\usepackage{caption}
\usepackage{graphicx,subfigure}
\usepackage{authblk}
\usepackage{enumitem}
\usepackage[utf8]{inputenc}
\usepackage{fancyhdr}
\usepackage{xcolor}
\pagestyle{fancy}
\usepackage{microtype}
\renewcommand{\headrulewidth}{0pt}
\renewcommand{\footrulewidth}{0pt}
\setlength\headheight{80.0pt}
\addtolength{\textheight}{-80.0pt}
\chead{%
  \ifcase\value{page}
  % empty test for page = 0
  \or \includegraphics[width=\textwidth]{headerimage.png}% page = 1
 \or \includegraphics[width=\textwidth]{headerimage.png}% page = 2
  \or \includegraphics[width=\textwidth]{headerimage.png}% page = 3
  \or \includegraphics[width=\textwidth]{headerimage.png}% page = 4
 \or \includegraphics[width=\textwidth]{headerimage.png}% page = 5
  \else
  \includegraphics[width=\textwidth]{headerimage.png}
  \fi
}

%\chead{\includegraphics[width=\textwidth]{headerimage.png}}
\hypersetup{
    colorlinks=true,
    urlcolor=blue
}
 
\let\oldabstract\abstract
\let\oldendabstract\endabstract
\makeatletter
\renewenvironment{abstract}
{\renewenvironment{quotation}%
               {\list{}{\addtolength{\leftmargin}{1em} % change this value to add or remove length to the the default
                        \listparindent 1.5em%
                        \itemindent    \listparindent%
                        \rightmargin   \leftmargin%
                        \parsep        \z@ \@plus\p@}%
                \item\relax}%
               {\endlist}%
\oldabstract}
{\oldendabstract}
\makeatother

% Left align captions
\captionsetup{justification   = raggedright,
              singlelinecheck = false}


\begin{document}

\title{Overview of the proceedings of the 2020 Inclusion in
Science, Learning a New Direction, Conference on
Disability (ISLAND)}

\author[1]{\large \color{blue}Cary A. Supalo}
\author[1]{\large \color{blue}Jasodhara Bhattacharya}
\author[1]{\large \color{blue}Daniel Steinberg}

\affil[1]{Princeton University}

\toappear{}
%% ABSTRACT
\maketitle

\begin{large}
    

 The 11th annual Inclusion in Science, learning a New Direction, Conference on Disability was hosted by Princeton Center for Complex Materials a National Science Foundation funded Materials Research Science and Engineering Center (MRSEC) and Princeton University on Saturday, September 19, 2020 in a virtual format due to the Covid19 pandemic. This annual conference included presentations that featured innovative research done by science teachers, science education researchers, access technology developers, and policy makers, other disability conference organizers, and others interested in the full inclusion of persons with disabilities into the Science, Technology, Engineering, and Mathematics (STEM) workforce. Due to the virtual conference format, we were able to include closed captioning for all presentations. Princeton University is also known as an Aira Access point. Aira is a visual interpreter service to which blind and vision impaired (BVI) persons can subscribe, that assists them in their daily lives with navigation, and reading print materials in the home or in the community among numerous other purposes. 
 \end{large}
 
 \textbf{*Corresponding Author, Cary A. Supalo }\\
\href{mailto:csupalo@independencescience.com}{(csupalo@independencescience.com)} \\
\textit{Submitted  March 29, 2021}\\
\textit{Accepted March 29, 2021} \\
\textit{Published online April 20, 2021 } \\
\textit{DOI: 10.14448/jsesd.13.0005} \\

\begin{large}
     The ISLAND conference, with the support of Princeton University, was able to offer the Aira service to any blind patron of the conference that requested it, to provide them with visual descriptions of PowerPoint slides that were being discussed by each speaker. According to Aira, the 2020 ISLAND conference was the first conference to attempt this innovative new accessibility addition.
\\
\\
     Most of the ISLAND 2020 presenters agreed to allow their presentations to be video recorded and are now available on YouTube for archival viewing in their entirety. Due to conditions in the world at the time of this writing, the 2021 ISLAND conference will also be conducted in a virtual environment on Friday, September 17 and Saturday, September 18, 2021. More details will be forthcoming on the ISLAND Conference website. The opportunity of an on-line virtual environment allowed ISLAND 2020 to have its largest participation since its founding, both in terms of a record registration count and actual participant presence.
\end{large}


%% AUTHOR INFORMATION



\vspace{5mm}
\section*{\vspace{140mm}}
\begin{large}
  
Leading off the conference was a presentation titled, \textbf{Independent laboratory learning for students who are blind or low vision }by Dr. Siegbert Schmid  from the University of Sydney in Australia, with co-authors Dr. Alice E. Motion, Dr. Peter J. Rutledge, Dr. Stephen George-Williams, Dr. Reyne Pullen, and Dr. David Evans from the University of Sydney, as well as Dr. Cary A. Supalo of the Educational Testing Service, Princeton, New Jersey and Jasodhara Bhattacharya of Princeton University, Princeton, New Jersey. This presentation discussed how researchers at the School of Chemistry in the University of Sydney are working to develop a general chemistry lab curriculum that is fully accessible to a blind student, using a comprehensive suite of current STEM (science, technology, engineering and mathematics) access methodologies and technologies. This initiative aligns with the culture of the school, which aims to be inclusive for all staff and students and addresses the current inaccessibility gap of blind and low vision (BLV) students independently carrying out first year chemistry laboratory experiments. In doing so, this research effort is poised to be the first college level general chemistry lab curriculum that is fully accessible to BLV learners, and serves as an exemplar of using access technologies and multi-sensory laboratory techniques for participating in the higher education chemistry laboratory and observing chemistry phenomena.

Next, Dr. Jason White from the Educational Testing Service in Princeton, New Jersey, presented \textbf{Making Scientific and Technical Materials Pervasively Accessible.} This presentation provided a detailed illustration of what is necessary to make math content more seamlessly accessible to blind and vision impaired (BVI) persons on the worldwide web. Currently, math and STEM content are not typically born accessible despite calls to do so from many quarters. Rather it is idiosyncratically implemented by authors, publishers, and other content producers. Ensuring consistent born accessible content requires that design and implementation practices are in place which reliably and consistently result in materials that are well constructed for cross-disability access. Solving this problem can be accomplished through education, policy, advocacy, and technological and methodological innovations including assistive technologies and research and software development. Creating the requisite public awareness can result from suitable education and training for relevant participants in the creation of scientific and technical materials, sufficient to equip them with appropriate knowledge and skills. Dr. White asserts that it is possible to achieve the objective of born-accessible STEM content by establishing standards and policies, including laws, regulations, organizational processes, and technical specifications, and enacting them into practice. His remarks were forward-thinking and thought provoking for all in attendance.

Building on this was Dr. Stefan M. Kilyanek’s paper, titled \textbf{Chemistry for ALL!: Developing Web Standards for the Interaction of Chemical Notation and Screen Reader Software.} Dr. Kilyanek, Associate Professor at the University of Arkansas, Fayetteville, Arkansas, remarks addressed the work currently being done by the W3C’s chemistry community to make chemistry symbology accessible to persons with BLV, by developing a common language for learning through screen-readers. The current lack of accessibility intersects with other narratives of inequities such as the problem of problematizing students with disabilities whether as colleagues or as students. In seeking to making chemistry content accessible so that persons with BLV too can access this STEM subject area, the W3C’s standardization effort therefore may help to shift discourse in these intersecting narratives. The W3C chemistry community consists of volunteer participants from the United States and around the world. Participation is open to all who register to participate that express an interest in making chemistry content more accessible on the web.This standardization effort adds to the historical narrative of standardization of symbols in chemistry by relevant actors such as IUPAC and ACS, amongst others. This presentation discussed on-going work to disambiguate chemical symbology terminology and how this content can be rendered leveraging the MathML authoring tools. It is the expectation of this group that these proposed conventions will be adopted soon.

The morning break was followed by a presentation titled, \textbf{Promoting Diversity and Accessibility and Developing BVI STEM Outreach Activities with Volunteer Research Scientists} delivered by Dr. Daniel Steinberg Education Director from the Princeton Center for Complex Materials (PCCM) at Princeton University, Princeton, New Jersey, in addition to co-presenters Shafayet Hossain, Nikita dutta and Nick Garcia. This presentation discussed how early-career scientists (graduate and post-doctoral level researchers) can be incorporated into outreach activities that were designed for blind and vision impaired students. Three early career scientists - Nikita Dutta, Nick Garcia and Shafayat Hossain, had to become familiar with the needs of these low incidence populations and thus had to adapt their instruction accordingly. These activities were conducted at several regional gatherings of blind and vision impaired student events where the participants varied in age from as young as fourth graders to undergraduate college students. Participants were both sighted and blind and were actively engaged in the hands-on science activities that were developed at Princeton University. It is the hope that these and other similar activities can be developed and shared with the BVI community.

Next were a series of three presentations from Metropolitan State University in Denver, Colorado. The first was titled,\textbf{ An Artificial Intelligence Tool for Accessible Science Education} by Jacob Watters, along with co-authors Dr. Melissa Weinrich of University of Northern Colorado, and Dr. April Hill and Dr. Feng Jiang of Metropolitan State University. According to the speaker, Jacob Watters, 6.7 million students in the US have additional learning needs, with vision impairment impacting 4.6\% of all US adults and 1.2\% of all US children under 18. The presentation commenced with an overview of existing accessible science tools used today, such as text-to-speech tools, talking balances, notched syringes, text dictation on smartphones, human assistants, and the Sci-VoiceTM Talking LabQuest (TLQ) developed by Independence Science. The limitations of current tools resulted in an exploration of using artificial intelligence (AI) tool with natural language processing (NLP) techniques and the Amazon Alexa Skills Kit (ASK), to control STEM laboratory access equipment through the spoken word.  This futuristic proof of concept work shows how AI tools can be incorporated in a unique way to promote science accessibility in the laboratory classroom.

This was followed by a presentation titled, \textbf{Digital Plant Growing System for Accessible Biology Education} by Alexander Lyubomirov (Sasha) and co-authors Dr. Feng Jiang and Dr. Ranjidha Rajan from the Metropolitan State University. The presenter Alexander Lyubomirov, discussed the use of digital cameras along with a custom software interface that measured visual analysis of plant growth over time. The set-up featured in this presentation uses various Arduino sensors, which may be programmed to be text-to-speech, to detect relative humidity temperature and plant growth without human supervision. Computer vision analysis was used on photos. A user of this interface could determine the rate of plant growth over specific time intervals. This presentation clearly shows how innovative applications for commercially available technologies can be used to promote STEM accessibility in contexts that may not have been thought to be possible.

Our morning session concluded with the presentation titled \textbf{Making analytical chemistry accessible to blind and vision impaired students} by Dr. Alycia Palmer from Metropolitan State University. This presentation was a case study on a specific blind student who is enrolled in Dr. Palmer’s analytical chemistry course at Metro State University in the Chemistry Department. To conduct analytical chemistry laboratory experimentation, this student used a series of STEM access technology tools such as the Ohaus Talking balance, the Sci-VoiceTM Talking LabQuest, a voice recorder, and the PENfriend voice labeler to tag and identity various reagents, as well as methodological approach- es including laboratory bench-top management and a personal computer with a text-to-speech screen reader to access lab procedures. This presentation shows how the synergistic creativity of an instructor and student can devise STEM accessibility approaches that previously may not have been attempted in this type of context. Dr. Palmer asserted that many of the approaches devised for this course may be transferred to other science learning contexts.

During the lunch break at this year’s conference, it was with great pleasure that we had a keynote address titled \textbf{Science Without Sight: Research from Another Perspective}, delivered by Maureen J. Hayden from the Texas A\&M University, in College Station, Texas. Ms. Hayden is currently a graduate student who is vision impaired. She discussed her journey to science research as a person with low vision, detailing how she overcame her disability, and the experiences that informed and inspired her career in marine biology. She spoke to how she uses low vision tools, some commercially available and others custom designed, that allow her to perform all her required tasks in this field that she loves. Her remarks were very inspirational to the conference attendees, with many participants asking specific questions about the field of marine biology for persons with blindness and low vision.

For the first time in the ISLAND conference history, two concurrent sessions were offered, with each track featuring three papers for a total of six different presentations. Conference attendees could attend either or both tracks depending on their preferences. 

Concurrent session 1 commenced with a presentation by paralympian Dr. Vincent Martin, titled \textbf{Efficiency in STEM: The intersection of Usability and Accessibility.} Dr. Martin currently works for Regions Bank and is based in Atlanta, Georgia. This presentation was delivered from an ethnocentric perspective, commencing with how Dr. Martin self-diagnosed his deteriorating vision as blindness. This diagnosis did not deter him from the path of becoming an engineer, and today he synergistically leverages five advanced degrees, including two doctoral degrees, for his work in rehabilitation engineering. Dr. Martin deconstructed three dimensions of accessibility in STEM including access (“the ability to get access to it”), usability (“how well we can use it”) which is influenced by learning modalities and human factors such as decision-making processes, and design thinking. He felt that inclusive design, barrier-free design, and accessible design are synonymous. Dr. Martin concluded with the network of experts he has been coalescing for the past three years including university faculty, upper management in industry, and others to develop a comprehensive accommodations handbook analogous to what exists in other fields such as engineering psychology and industrial engineering. It is his hope that such a handbook will inform and improve accessibility in education and employment for all. 

This presentation was followed by a talk titled \textbf{An Accessible CURE: Course- Based Undergraduate Research Experiences for Deaf/Hard-of-Hearing Students} by Dr. Todd Pagano, Dr. Annemarie Ross, and Dr. Susan Smith Pagano, of the Rochester Institute of Technology (RIT). Dr. Todd Pagano discussed the innovation in research-based programs, CURE, to be inclusive of students who are deaf or hard-of-hearing. CURE - Course Based Undergraduate Research Experiences - is a program format that provides research opportunities for entire courses of students to participate in the research process. This logistically accommodates many more students than would be possible with one-on-one research projects, thus scaling the research experience for a larger population. NTID’s inaugural CURE program for deaf and hard of hearing students centered around food chemistry themes (nutmeg, a spice with antioxidant properties, and honey, one of the world’s most counterfeited products) to develop students’ technical and laboratory research skills. While the full data-set from this research continues to be collated, the initial qualitative data clearly indicate benefits to student learning 

The final presentation in concurrent session 1 was titled \textbf{The PLAYground: Deaf and Hard of Hearing Virtual STEM Summer Camp}, presented by Emma Monson, Krista Schumacher and Dr. AnnMarie Thomas from the University of St. Thomas in St. Paul, Minnesota. This presentation discussed an informal learning program, the PLAYground summer camp, which was developed by the Playful Learning Lab, an undergraduate research group at the University of St. Thomas with a focus on play-based learning. According to the presenters, the innovative format of this STEM enrichment program arose due to the Covid-19 pandemic requiring program modification for in-home instruction. The PLAYground is an 8-week virtual summer camp consisting of STEAM workshops for students, including those with deafness and who are hard of hearing (DHH), in elementary and middle schools. The program was designed at point-of-inception to be inclusive of students with DHH by incorporating American Sign Language (ASL), as well as addressing economic accessibility barriers through the use of kits. The 85 participating students in the program worked with kits that were delivered to each participant at their homes by program staff. Virtual meeting technology was used to provide student interactions with their instructors. This approach is a beautiful illustration of innovative thinking that resulted in a highly successful program in the global pivot to remote learning. 

There were three presentations in \textbf{Concurrent Session 2}. The first was titled \textbf{SciAccess: Making Space for Everyone} by Anna Voelker, Caitlin O’Brien and Michaela Deming from The Ohio State University in Columbus, Ohio. This presentation discussed innovative methodologies to teach students with vision impairments astronomy concepts by means of 3D models and tactile graphics when appropriate. This presentation also discussed the Sci-Access conference which features presentations from STEM experts interested in promoting the full inclusion in science learning from all disability groups. This conference is a complement for attendees of the ISLAND conference who wish to broaden their interests further in the STEM access arena.

Next, Gina Fugate from the Maryland School for the Blind (Baltimore, Maryland) presented \textbf{Quorum Lego Robotics}. The Maryland School for the Blind sponsors two First Lego League teams - the DOT5UDOGS and 180 Optimum. This presentation discussed how the Quorum programming platform can be used to develop accessible Lego-based robotics activities. The Quorum platform is viewed by many in the blind community as one of the most accessible programming platforms that can be easily used in the K12 space today. There have been concerns with First Lego League in recent years regarding the accessibility of their robotics curriculum.  This presentation shows how some key aspects to that curriculum can be made accessible.

The final presentation in concurrent session 2 was titled, see 3D \textbf{– 3D Printing for People who are Blind} and was delivered by Caroline Karbowski founder of See 3D and an undergraduate student at The Ohio State University. This presentation discussed how See3D can provide free 3D model services to teachers and blind students who make on-line requests. This organization is staffed by volunteers and operates primarily on donations from individuals and corporations. This innovative organization shows how good work can be scaled with low overhead, and how this can and does result in high impact on the blind community.

Once the afternoon break concluded, the final series of presentations was led off by the paper titled \textbf{STEM Access and Ambiguity in Speaking Math}, by Dr. Mick Isaacson from Ivy Tech Community College from Lafayette, Indiana. Dr. Isaacson discussed the challenges with the instruction of mathematics, which can consist of many spoken ambiguities that make comprehension difficult for students with vision impairments. Simply reading what is written is not enough to make math content accessible. Dr. Isaacson promotes the use of the MathSpeak rules of disambiguation for spoken mathematics. These rules were originally developed by Dr. Abraham Nemeth, the original author of the Nemeth Code for Mathematics and Science Notation, and a blind mathematician who felt that any blind student could learn any aspect of mathematics if the instruction could be offered using the MathSpeak rules. These MathSpeak rules provide clear enunciation of numerators and denominators, breaks in terms that go with specific quantities, and what terms are impacted by specific math operators. Dr. Isaacson’s research has found that once math teachers received professional development training in the MathSpeak rules, the comprehension of students with vision impairments increased as a result of disambiguation. If more math teachers can learn to teach while speaking mathematics in a non-ambiguous fashion, it can enhance learning experiences for blind and vision impaired students.

This presentation was followed by a talk titled \textbf{Accessing and Understanding Graphical Information for the Blind: How to}, by Dr. Ashley Nashleanas from Morningside College, located in Sioux City, Iowa. This presentation, based on doctoral dissertation, discussed the efficacy of effective tactile graphics being used as part of math instruction for students with vision impairments (SVI). This research found that when tactile graphics are developed in a way that proves to be efficacious to the braille learner, comprehension of mathematics concepts is increased. Dr. Nashleanas asserts that based on this research study, tactile graphics are a valuable tool in the instruction of the blind in mathematics. Her research indicates that while teachers perceive that SVI with previous vision were more likely to perform successfully on graphical tasks than SVI with minimal visual experience, with appropriate guided instruction SVI who are congenitally blind can perform in a comparable manner. However, additional research is necessary to conduct further investigations in other branches of mathematics. The presentation concludes with implications for teaching practice.

The next presentation was titled \textbf{Intelligent Assistants for More Effective Multisensory Exploration of Digital Images} by Dr. Bradley Duerstock from the School of Biomedical Engineering and School of Industrial Engineering at Purdue University in West Lafayette, Indiana. The exploration of digital images is not readily available for individuals who are blind with current solutions centering on tactile papers. Dr. Duerstock presented an educational research study which explored the solution of haptic virtual graphics in combination with artificial intelligence (AI) to provide real-time access to digital images, with intelligent synchro-\\nous assistance to support image exploration. The three main questions investigated by this study were:1) how to determine what exploration procedures performed by users who are blind and vision impaired (BVI); 2) what kind of help are effective for assistance; and 3) how to measure the performance of the proposed system. The study population comprised blind students and blind-folded students. The data-set from this study indicates that AI is as efficient, if not more, than human assistance using this methodology. 

The penultimate presentation was titled, \textbf{Design + Disability: Building the next generation of inclusive designers} and was delivered by Kate Ganim and Anaiss Arreola, representatives from Born Just Right/Make Just Right, based in Pawtucket, Rhode Island. This presentation incorporated both program-level and community-level dimensions.  At the program-level, the presenters discussed BOOST workshops, where persons with limb-difference  can leverage maker spaces to design and prototype prosthetics that can assist with their ability to perform a specific task. Examples of the impact of BOOST workshop include students who have spoken at TEDx, and presented at the White House Week of Making. Born Just Right has seeded a new organization Make Just Right as a response to the demand for pairing engineering with experience-informed expertise in accessibility. Anaiss Arreola, an alumnus of Born Just Right and one of the cofounders of this new organization spoke about the importance of shifting hearts and minds by being an actor in reframing traditional deficit views of disability to one of empowerment. Make Just Right has consulted with Mattel in their design of a Barbie that is inclusive of disability. This presentation shows the importance of working in a synergistic way to effectively produce solutions that are desirable and useful, including co-creation and long-term view of partnering with communities with disabilities. 

The final presentation of the 2020 ISLAND conference was titled \textbf{Applying Universal Design Principles in Identifying Accessibility Solutions for Students in STEM} and was delivered by Dr. Mahadeo A. Sukhai, Head of Research and Chief Accessibility Officer from the Canadian National Institute for the Blind (CNIB), Toronto, Ontario and Ainsley R. LaTour, IDEA-STEM, Kingston, Ontario. The mission of IDEA-STEM is to make STEM and healthcare disciplines accessible to people with disabilities, and to improve the trainee experience of students and trainees with disabilities in these fields. This presentation overviewed several policy perspectives that background accessible STEM education, and the intersection of universal design principles with different dimensions of accessibility and inclusion in STEM. The presenters gave several illustrative examples of how accessibility and inclusion in STEM may be accomplished. They also discussed how more collaboration within the STEM access community is necessary to advance the science access goals of students with disabilities.

The 2020 ISLAND conference clearly had one of the most diverse programs ever offered in the history of the conference. The organizers wish to acknowledge all the presenters, participants, and conference sponsors who helped make this year’s conference possible. Additionally, readers are encouraged to consult other ISLAND conference papers from the 2020 proceedings that have published in JSESD. As a result of the global pivot to virtual and remote environments, the 2020 ISLAND conference presentations that were recorded and made available on YouTube will also provide more details about the perspectives of our different presenting experts. The summaries included as part of this discussion were very short and are intended to serve as an entree to the other subsequent papers and video recordings that are available on-line.
The authors of this overview paper of the 2020 ISLAND conference hope you found this year’s proceedings to be informative and useful to your work. Many of the multi-sensory approaches were unique in their own right and serve as more illustrative examples of how educator innovation can move the needle forward in promoting science access and the full inclusion in the STEM courses of study.
 
This research was partially supported by NSF through the Princeton University’s Materials Research Science and Engineering Center DMR-2011750. Additional sponsors whose support made possible an inclusive virtual ISLAND 2020 for a wide audience include Princeton University’s Office of Information Technology, Princeton University’s Campus Conversation on Identities, and Independence Science.

\end{large}
\end{document}
