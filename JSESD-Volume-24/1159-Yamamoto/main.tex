\documentclass[11.5pt]{sig-alternate}
\usepackage[defaultlines=3,all]{nowidow}
\usepackage{hyperref}
\usepackage{tabularx}
\usepackage{graphicx}
\usepackage{blindtext}
\usepackage[utf8]{inputenc}
\usepackage[english]{babel}
\usepackage{lastpage}
\usepackage{comment}
\usepackage{dirtytalk}
\usepackage{xcolor}
\usepackage{hanging}
\usepackage{wrapfig}
\usepackage[backend=biber, style=apa]{biblatex}
\addbibresource{notation.bib}
\usepackage{authblk}
\usepackage{caption}
\usepackage{graphicx,subfigure}
\usepackage{authblk}
\usepackage{enumitem}
\usepackage[utf8]{inputenc}
\usepackage{cuted}
\usepackage{fancyhdr}
\pagestyle{fancy}
\usepackage{lipsum}
\usepackage{xurl}
\usepackage{tabu}
\usepackage{longtable}
\usepackage{ragged2e}
\usepackage{microtype}
\renewcommand{\headrulewidth}{0pt}
\renewcommand{\footrulewidth}{0pt}
\setlength\headheight{80.0pt}
\addtolength{\textheight}{-80.0pt}
\chead{%
  \ifcase\value{page}
  % empty test for page = 0
  \or \includegraphics[width=\textwidth]{headerimage.png}% page = 1
 \or \includegraphics[width=\textwidth]{headerimage.png}% page = 2
  \or \includegraphics[width=\textwidth]{headerimage.png}% page = 3
  \or \includegraphics[width=\textwidth]{headerimage.png}% page = 4
 \or \includegraphics[width=\textwidth]{headerimage.png}% page = 5
  \else
  \includegraphics[width=\textwidth]{headerimage.png}
  \fi
}

%\chead{\includegraphics[width=\textwidth]{headerimage.png}}
\hypersetup{
    colorlinks=true,
    urlcolor=blue
}
 
\let\oldabstract\abstract
\let\oldendabstract\endabstract
\makeatletter
\renewenvironment{abstract}
{\renewenvironment{quotation}%
               {\list{}{\addtolength{\leftmargin}{1em} % change this value to add or remove length to the the default
                        \listparindent 1.5em%
                        \itemindent    \listparindent%
                        \rightmargin   \leftmargin%
                        \parsep        \z@ \@plus\p@}%
                \item\relax}%
               {\endlist}%
\oldabstract}
{\oldendabstract}
\makeatother

\captionsetup[figure]{font = it, labelfont = bf, font = large}
\captionsetup[table]{font = it, labelfont = bf, font = large}
\captionsetup{justification = raggedright, singlelinecheck = false}

\begin{document}

\title{STEM and high school students with disabilities:\\
A qualitative review of the research literature 
}

\author[1]{\large \color{blue}Scott Yamamoto Ph.D}
\author[1]{\large \color{blue}Charlotte Y. Alverson Ph.D.}
\author[1]{\large \color{blue}Laura McCoid-Goudy M.A.T.}
\author[1]{\large \color{blue}Hannah Castle B.A.}
\author[1]{\large \color{blue}Jacquelyn Burr M.Ed.}
\affil[1]{University of Oregon}

\toappear{}
%% ABSTRACT
\maketitle
\begin{@twocolumnfalse} 
\begin{abstract}
\item 
 \textit {We conducted a qualitative review of the research literature on STEM (science, technology, engineering, mathematics) related to high school students with disabilities (SWD). We selected and analyzed 53 articles to answer two questions: (1) How are high-school SWD prepared for careers in STEM? (2) How are educators prepared to support high-school SWD for opportunities in STEM? In answering the first question, four qualitative themes emerged: (a) barriers to STEM, (b) increasing STEM opportunities, (c) STEM readiness in college and career, and (d) STEM identity. In answering the second question, three qualitative themes emerged: (a) individualizing learning and supports for SWD, (b) using technology and collaboration among educators, and (c) professional development for educators. Limitations of this review related to search terms and inclusion criteria. Implications of this review related to the need for more research on STEM enrichment programs, STEM identity, and long-term outcomes.}
     \\
     \\
     Keywords: high school students with disabilities, STEM and special education, STEM careers, hidden STEM, qualitative literature review
\end{abstract}
\end{@twocolumnfalse}

%% AUTHOR INFORMATION

\textbf{*Corresponding Author, Scott Yamamoto Ph.D}\\
\href{mailto: syamamo1@uoregon.edu }{(syamamo1@uoregon.edu)} \\
\textit{Submitted  Oct 3 2021 }\\
\textit{Accepted Jun 21 2022} \\
\textit{Published online  Dec 7 2022} \\
\textit{DOI:10.14448/jsesd.14.0006} \\

\clearpage
\begin{large}
\section*{INTRODUCTION}
The National Science Foundation (NSF) began using an acronym ‘SMET’ in the 1990s, combining science, mathematics, engineering and technology (McComas, 2014; Sanders, 2009). As the Assistant Director for Education and Human Resources Division at the National Science Foundation (NSF) in 2001, Dr. Judith Ramaley rearranged the letters to ‘STEM’. In an interview several years later she explained the acronym change, stating that ‘STEM’ emphasized the connection between the four individual subject areas, rather than implying that any one or two were more important than the others (Christenson, 2011; Chute, 2009). 
While some have viewed STEM as eluding a single straightforward definition (see Gerlach, 2012), others have posited that one is unnecessary (Holmlund et al., 2018). Regardless of whether such a definition will ever be established, the last twenty years has seen STEM grow from classrooms and research centers to mainstream culture. That growth, however, has not occurred evenly nor experienced similarly across different groups of people. Young adults and students with disabilities (SWD), especially, have encountered more barriers to STEM opportunities and their benefits than their peers without disabilities (National Science Foundation, 2021). Thus, we conducted a qualitative review of the research literature over the last twenty years in order to understand what that growth in STEM has meant in terms of the educational and career goals and opportunities for high school SWD. 

\subsection*{Stem Careers and Students with Disabilities}
The Bureau of Labor Statistics (2017) reported that in 2015, there were nearly 9 million jobs in science, technology, engineering, and mathematics (STEM) fields with an average annual wage of \$87,570 in STEM occupations and \$45,700 in non-STEM occupations. In April 2020, the Bureau projected 8.8\% growth in STEM occupations in the U.S. from 2018 to 2028, with a median wage of \$86,890, and 5.0\% growth in non-STEM occupations with a median wage of \$38,160. These government reports focus on STEM occupations requiring at least a bachelor’s degree and clustering in metropolitan areas, such as San Francisco and New York. That focus, however, limits the consideration of and access to STEM careers for millions of people who do not fit into either category. In response, the Brookings Institution analyzed STEM occupations by coding the “O*NET Knowledge Statements” used to define occupations in the labor market based on the amount of STEM knowledge required (see Rothwell, 2013). That process resulted in expanding the STEM designation to include occupations requiring less than a bachelor’s degree and existing outside of metropolitan areas. This expanded designation comprises what is now generally known as the ‘hidden’ STEM economy. 
Major initiatives by the National Science and the U.S. Department of Education, among others, have emphasized the preparation of all youth for college and careers STEM fields. Despite these efforts, research and data continue to show that SWD in early grades are falling behind their peers without disabilities in science achievement (National Center for Education Statistics, 2015). SWD are also (a) less likely to graduate from college or university in a STEM major (National Science Foundation, 2019), (b) more likely to be unemployed or under-employed, and (c) more likely to live in poverty (Semega et al., 2019). Even adults with disabilities who have STEM degrees have experienced (a) fewer opportunities in internships and research assistantships, (b) less funding from scholarships and grants, and (c) higher unemployment rates in STEM fields than their peers without disabilities (National Science Foundation, 2021). In a reviewing a decade of NSF-funded research aimed at broadening the participation of SWD in STEM, Thurston et al. (2017) reported that structural barriers  encountered by people with disabilities in STEM have persisted. These include discrimination, lack of accommodations, lower expectations, lack of access to facilities and adaptive technologies, and lack of knowledge/skills of faculty. With career awareness and planning, SWD can take advantage of opportunities in high school to prepare for a STEM career. This is critical for SWD, who are legally entitled by the Individuals with Disabilities Education Act (IDEA 2004, P.L. 108-446) to receive transition services that prepare them for life after high school. 
We intended this qualitative literature review to inform the field, particularly high-school educators and transition specialists. Aside from reporting our findings, we also had practical goals of increasing awareness of the different pathways to a STEM career. We specifically focused this review on high-school SWD and educators as they are at the core of the special-education transition process (i.e., IDEA Indicator 13) that prepare SWD for post-high school education/training or employment (i.e., IDEA Indicator 14). Thus we posed two main questions and corresponding sub-questions (see Bogdan \& Biklen, 2007) to frame our review: 
\begin{enumerate}
    \item  	How are high school SWD prepared for careers in STEM?
   \begin{enumerate}
    \item   What barriers do SWD identify relative to STEM coursework or careers?
    \item   What supports do SWD need in order to engage in STEM opportunities?
    \item   What contributes to SWD developing a STEM identity?
   \end{enumerate}
    \item 	How are educators prepared to support high school SWD for opportunities in STEM?
 \begin{enumerate}
    \item    How do educators individualize instruction for SWD in STEM?
    \item   What contributes to educators’ confidence in teaching SWD in STEM?
    \item   What professional development do educators need to support SWD in STEM?
\end{enumerate}
\end{enumerate}
\section*{METHODS}
We chose to conduct a qualitative review of the research literature related to STEM and high school SWD. Although a common criticism of the type of literature review is that it limits the generalization of cumulative knowledge (see Paré et al., 2015), we chose it for two reasons. One, we believed the field (i.e., both researchers and practitioners in education) would benefit from a broad coverage of articles that provide a sense of the scope of the current state of knowledge regarding STEM and high school SWD and how that knowledge has been derived. Two, we recognized that the extant literature in special education and related fields would contain a variety of articles and different methods (see Snyder, 2019). Being able to compare across these different articles (e.g., research reports, position papers) and methods (e.g., quantitative, qualitative) to discover common themes with a well-established research methodology (see Brantlinger et al., 2005), rather than only assessing measured quantitative effects, was essential (see Onwuegbuzie, Leech, \& Collins, 2012). We followed three steps in conducting the review: (a) searching multiple electronic research databases, (b) applying inclusion criteria for article selection, and (c) coding and analyzing selected articles. 

\subsection*{Searching Multiple Electronic Research Databases}
We started with a broad definition of STEM as referring to any one of the four fields – science, technology, engineering, and mathematics – as well as the integration of two or more fields (Honey et al., 2014). We conducted a search of electronic research databases, including EBSCO Host, Academic Search Premiere, ERIC, Social Science Database, and Sociological Abstracts, by applying combinations of ‘high school students with disabilities’ and the STEM terms. We also enabled the database search engine to use related words and terms. In this step, we applied two search filters: peer-reviewed articles in English and published in the year 2000 or later. We recognized that much has changed in the U.S. related to STEM and high school SWD since 2000, particularly with the passage of two federal education laws – No Child Left Behind (2001) and Individuals with Disabilities Education Improvement Act (IDEA, 2004). 

\subsection*{Applying Inclusion Criteria For Article Selection}
We downloaded articles from the databases and sorted them into two categories, relevant and not relevant. Relevant articles met any one of these inclusion criteria: (a) addressed high-school SWD preparing for or engagement in STEM careers, (b) high-school educator professional development for STEM, or (c) high school-level STEM program or curriculum. This process yielded 70 articles that were deemed relevant by a consensus of all the authors. We independently read each of the 70 articles applying the same inclusion criteria. By consensus we excluded 17 additional articles and selected a final set of 53 articles to review. These 53 articles are marked with an asterisk in the References section.

\section*{Coding And Analyzing Articles}
The first and second authors took the 53 selected articles and imported the PDF files into NVivo 12 (QSR, 2019), a software commonly used for qualitative data analyses. Because we utilized qualitative methodology for conducting this literature review, we followed best practices in qualitative research in education for ensuring trustworthiness and credibility: (a) reaching data saturation to include different perspectives and enhance richness of information, (b) triangulating different sources of data, (c) acknowledging how researcher perspectives, beliefs, and biases influence data collection and findings (i.e., reflexivity), (d) coding independently for initial review and then conducting consensus coding to develop final codes, and (e) minimizing reactivity through neutral stances and questions (Brantlinger et al., 2005).  
The first and second authors read and reviewed each article independently and applied start codes (Miles et al., 2014) on all 53 articles in the NVivo software. To ensure thorough and consistent coding, they defined the codes using examples and non-examples (Rossman \& Rallis, 1998) culled from the articles. Next, the authors extracted 'node reports’ from the NVivo software (a feature of the software) in order to inspect and identify main codes and sub-code extensions. The authors then selected 30 articles at random to conduct interrater agreement for coding using Cohen’s Kappa (see Cohen, 1960). Corrected for chance agreement (see McHugh, 2012), the computed Kappa coefficient was .73. In the final step of the analysis, the authors conducted consensus coding, and derived the themes to answer the two questions (above).  

\section*{FINDINGS}
A summary of all 53 articles selected for this qualitative literature review is provided in Table 1. It includes information about authors, publication year, journal, study design, sample, and demographics, and topics covered. These articles are also marked with an asterisk in the References section. These 53 articles were published between 2000 and 2020 in 36 peer-reviewed journals, and a plurality (\textit{n}=21) were reports of research using qualitative, quantitative, or mixed methods. The remaining articles were literature reviews (\textit{n}=12), essays (\textit{n}=6), position papers (\textit{n}=6), meta-analysis (\textit{n}=4), and practitioner papers (\textit{n}=4). As we had expected prior to the literature search, these articles span- ned a wide range of studies and methods.  

In answering the two questions framing this review, several qualitative themes emerged from our analysis of the selected articles. For the first question, “How are high school SWD prepared for careers in STEM?” there were four emergent themes: (a) barriers to STEM, (b) increasing STEM opportunities, (c) STEM readiness in college and career, and (d) STEM identity. For the second question, “How are educators prepared to support high school SWD for opportunities in STEM?” there were three emergent themes: (a) individualizing learning and supports for SWD, (b) using technology and collaboration among educators, and (c) professional development for educators. Each of these themes is described in order below. 

\subsection*{First Question: Preparing Secondary SWD for STEM Careers}
\subsubsection*{Barriers to STEM}
We found that this particular area of the literature receives the most research attention. Three specific barriers were most evident: (a) lack of STEM experiences, (b) inaccessible classroom or school environments, and (c) lack of access to STEM curriculum.

\textbf{Lack of STEM Experiences}. The lack of appropriate school STEM experiences for SWD limits their opportunities for STEM learning. Scruggs et al. (2008) noted that traditional approaches to science learning, which rely on textbook-centric heavy memorization and recall of facts and learning of science vocabulary, pose significant challenges for most SWD. Research over the last decade has indicated that inquiry-based learning, also sometimes referred to as experiential learning, hands-on learning, or project-based instruction (see Rizzo \& Taylor, 2016), is more appropriate for the individualized needs of SWD and their differing classroom and school environments (Brigham et al., 2011). Villanueva and Hand (2011) asserted that the emphasis should not be on memorization as “learning science cannot be a mere transmission of facts stemming from teacher-centered instruction” (p.235).      

Barriers for SWD in STEM can also be based on the type or severity of disability. For example, Isaacson and Michaels (2015) reported that students with print disabilities had difficulty acquiring knowledge through reading, especially with complex and dense STEM content, such as the ambiguity of spoken mathematics. Their research also showed that students with reading disabilities can encounter barriers resulting from the demands reading STEM texts places on working memory and attention. Socioeconomic status (SES), gender, race-ethnicity, and variables may also affect SWD overall STEM experiences and outcomes, involving educators and school systems at every level and affecting equity and underrepresentation of groups, such as female SWD (Mau \& Li, 2018; Wang \& Degol, 2017). This also has implications for STEM in terms of available education and career pathways, as diverse SWD become an increasingly larger share of the postsecondary education population and the workforce in STEM occupations (Byars-Winston, 2014).  

\textbf{Inaccessible Classroom or School Environment}. Supalo et al. (2011, 2014) noted that even when middle and high school students with blindness or low vision had expressed interest in STEM, particularly in the laboratory sciences, they had inadequate opportunities to explore these interests or were discouraged from those pursuits. Over time this would have a cumulative effect on the trajectory of career paths SWD choose and pursue, and can perpetuate biases of peers and educators. Lower expectations and negative stereotypes attributed to SWD abilities and or capabilities (see Dunn et al., 2012) are further complicated by biases or stereotypes based on other characteristics such as race/ethnicity and gender that can emerge in early youth (Wang \& Degol, 2017). Even when SWD have STEM opportunities, they can also encounter limited and inadequate accommodations in schools (Rule \& Stefanich, 2012).

In terms of instruction or curriculum, Gottfried et al. (2016) noted that while mathematics can be considered a gateway curriculum for STEM learning and more advanced coursework, for SWD at the secondary level there has been inadequate teaching of math, leading to fewer SWD moving on to advanced STEM coursework. Israel et al. (2013) posited that STEM instruction often does not sufficiently ensure accessibility or appropriateness for SWD, and special educators often face difficulties balancing individualized needs and curriculum expectations.     

\textbf{Lack of Access to STEM Curriculum}. Basham \& Marino (2013) found that as early as middle school, SWD can fall behind their peers without disabilities on standardized measures and develop negative attitudes toward STEM content. This can result in SWD becoming discouraged, and this feeling can progress through grade levels as scientific concepts become increasingly complex and even more inaccessible or difficult to comprehend (Basham et al., 2010). In turn, these factors can become linked to reduced or limited secondary and post-secondary STEM education and employment options, and ultimately result in lowered quality of life in adulthood (White \& Massiha, 2015). These types of barriers can overlap and become reinforced over time, which may result not only in perpetuating stigma about SWD in STEM, but also in reducing or limiting STEM opportunities and poorer outcomes for SWD.     

\subsubsection*{Increasing STEM Opportunities}
For this theme we found that there were two main points of focus: (a) expanding STEM programs by program and setting, and (b) recruiting and supporting SWD in STEM.   

\textbf{Expanding STEM by Program and Setting}. One of the most unique STEM programs specifically designed with accommodations for SWD was created by Dr. Supalo at Towson University in Maryland. Having been frustrated with the lack of opportunities and resources for students who were blind or had other visual impairments to engage in STEM, Dr. Supalo created a one-week summer camp called “Camp Can Do” in 2009 focusing on chemistry. The goal for the camp was to provide participants opportunities for hands-on learning about chemistry and other related science and technology experiences that could translate to STEM opportunities and learning in postsecondary settings and beyond. Supalo et al. (2011) reported that participants expected to also learn more about themselves, how they can utilize certain technologies to engage with science and chemistry and more broadly with the environment around them. Camp activities included multi-sensory, learning activities to build confidence, empower participants to seek further STEM opportunities, and learn problem solving skills that apply to every aspect of their lives, education, employment, and civic life. Participants also reported that the problem-solving skills they learned and their approach to lab activities both translated to academic success and greater personal independence. These types of programs (i.e., STEM summer camp) are vital because they supplement and build on other successful transition-focused (i.e., high school to college) models for SWD in STEM. 

Plasman \& Gottfried (2018) examined SWD and STEM looking at three interrelated factors: engagement in applied STEM coursework, high school completion, and progression in the STEM ‘pipeline’. They reported that students with a specific learning disability (SLD) who took applied STEM courses had increased math test scores and enrollment in further education and reduced likelihood of dropping out of high school. Subramaniam et al. (2012) advocated a unique approach for increasing opportunities in STEM, the use of school libraries as places and spaces for STEM learning. These authors contended that school libraries were well-situated and structured to run programs for SWD and have them actively participate in STEM learning, and to teach that pursuing STEM as a course of study or as a career would require them to do more than memorize facts or scientific information. The authors called for a deeper engagement and literacy about how to read, conduct, and speak about science, while also acknowledging that such an approach would neither be easy nor straightforward and would require addressing a number of individual, educational, and sociocultural factors.     

\textbf{Recruiting and Supporting SWD in STEM}. Leddy (2010) advocated for SWD to be provided a number of supports in STEM, including financial resources, opportunities to work in STEM labs, off-campus internships in STEM, cooperative learning, mentoring, and participation in STEM clubs. The author believed these supports could also aid SWD to understand their legal rights and improve self-advocacy skills by learning to request accommodations with faculty members. In evaluating the effects of student-learning communities on STEM for SWD, Izzo et al. (2011) reported they were effective in providing ongoing academic and personal supports. Peters-Burton et al. (2014) examined the effects of inclusive STEM high schools. With open enrollment and geared specifically to provide STEM-focused education for traditionally underrepresented youth, these schools were intended to help prepare students to pursue advanced STEM studies.

White \& Massiha (2015) analyzed outcomes of a STEM transition program aimed at increasing SWD in STEM disciplines. They reported that an integrated intervention system of (a) quick-ready access to personal and academic supports, (b) continuous mentoring, (c) degree enhancement, and (d) PD experiences helped SWD to persist in STEM. Dunn et al. (2012) concurred, citing mentoring as an effective strategy for SWD persistence in STEM fields. The issue of SWD persistence in STEM is an emerging area of research in the literature, and its importance and relevance to supports for SWD in STEM is likely to increase with the continued growth and expansion of occupations and careers in traditional and hidden STEM fields.  

 To increase SWD participation in STEM, Marino et al. (2010) recommended (a) raising the awareness of SWD in secondary and postsecondary education, (b) increasing institutional commitment to recruiting, retaining, and graduating SWD into STEM fields, and (c) ensuring secondary teachers and counselors and college/uni- versity faculty help SWD to view STEM as a viable career path or choice. Research has also focused on increasing SWD participation in STEM through a much wider application of the principles of universal design. Generally defined as the deliberate design of products and environments to be usable by all people to the greatest extent possible without the need for adaptation or specialized design, UDL contributes to a learning environment for everyone (see Hall et al., 2012). Not knowing about or ineffectively incorporating UDL principles in rigorous STEM learning activities can create an unnecessary barrier (Moon et al., 2012). Basham and Marino (2013) asserted that UDL can also enhance access to STEM for SWD especially in a general education setting.  
 
\subsubsection*{STEM Readiness in College and Career}
This theme – and area of research literature – appeared to be more sparse and emerging than other areas. Cease-Cook et al. (2015) have posited that in order for high school SWD to be ready for STEM in college or career they need a complete and rigorous high-school curriculum in the specific disciplines combined with work-based learning experiences. Further, they believed this preparation involved partnerships with organizations in the community for SWD to gain valuable experience concretely learning about various STEM careers and work styles. Job sampling, internships, apprenticeships, and service learning experiences could help identify jobs SWD might like after leaving high school. Gottfried et al. (2016) assessed that the traditional approach to preparing students for STEM beyond high school, via a rigorous curriculum supplemented by school-based experiential activities, is insufficient for preparing SWD. 

Sublett and Plasman (2017) submitted that applied STEM coursework, which includes career and technical education (CTE), can support and promote college and career readiness for SWD. When SWD have more experiences in applied STEM courses, they are also more likely to persist in STEM through high school and to connect STEM with post-high school opportunities (Plasman \& Gottfried, 2018). Beyond access to rigorous STEM coursework in school to prepare SWD for college or career pathway, there also needs to be access and opportunities for STEM learning after school or during out-of-school time, such as summer programs, field trips, mentoring, and or tutoring (Rakich \& Tran, 2016). 

\subsubsection*{STEM Identity}

This theme represents the newest area of the literature related to STEM and high school SWD. STEM identity and its often-associated area of social-emotional learning (SEL) have traditionally received the least attention in special education, perhaps, because it is difficult to research due to construct/concept and data complexity. That has begun to change, as exampled by the emphasis the NSF has placed on STEM identity and SEL for its STEM-focused grant programs (see National Science Foundation, 2018). Further, the renewed attention on the foundational work of noted psychologist Dr. Albert Bandura and others (e.g., Durlak et al., 2011) in special education and early childhood research fields can be seen in the literature.    

Whether pertaining to students with or without disabilities, variables such as identity, self-efficacy, and self-confidence are not only interconnected but also complex to analyze and understand, and highly contextualized. They also often become salient practical issues in STEM given the differences in expectations, as well as the stereotypes and bias in education and employment that affect outcomes of SWD. Gregg et al. (2017) studied the effects of virtual mentoring, using devices and platforms such as email, smartphones, and social media, on persistence in STEM for high-school SWD. The authors found that the largest improvements were in their perceptions of self-advocacy and self-deter- mination, although these outcomes differed by student disability type and ethnicity. Likewise, analyzing nationally representative data of high school students, Sublett and Plasman (2017) reported that applied STEM coursework was predictive of self-efficacy increases in science and math for males without disabilities, but not for females or for SWD. 

Mau and Li (2018) suggested school counselors also take family influences into account when working with SWD to prepare them for career or college pathways; and to work with educators and administrators on policies that encourage STEM aspirations and advocate for SWD in their career goals. The active involvement of educators, counselors, and administrators in the development and enhancement of self-efficacy acknowledges their key roles influencing SWD motivation, confidence, and success, and how self-perception affects self-efficacy. This awareness could mean SWD are empowered to focus on practical aspects of pursuing STEM college and career pathways, such as earning postsecondary scholarships or seeking occupations with higher incomes in STEM (Shoffner et al., 2015). Individual differences in self-efficacy and development of a STEM identity are also important issues to address. 

Research has indicated that most individuals begin deciding future careers before postsecondary education, and interests in science and math can take hold as early as middle school (Wang \& Degol, 2017). Women scientists have also reported their school STEM experiences were key to developing their interests. Moreover, research has indicated females with high math achievement but low interest or motivation in pursuing STEM occupations are less likely to obtain a STEM degree than females with average math achievement but high interest or motivation in pursuing STEM occupations (Wang \& Degol, 2017). This underscores the importance of STEM supports especially in the form of STEM role models for SWD.       

\subsection*{Second Question: Preparing Educators to Support SWD in STEM}

\subsection*{Individualizing Learning and Supports}
The National Education Association (2019) defines project-based learning (PBL) as:  
\begin{quotation}
\noindent
    A model for classroom activity that shifts away from teacher-centered instruction and emphasizes student-centered projects… helps make learning relevant to students by establishing connections to life outside the classroom and by addressing real world issues. In the school and beyond, the model further allows teachers opportunities to build relationships among colleagues and with those in the larger community (para.2).
\end{quotation}

Project-based learning (PBL) is often a characteristic of STEM focused schools. Bargerhuff (2013) identified ten STEM schools in the Midwest and reported findings from one that was developed by several stakeholder partners including K-12, higher education, business, government, and local community. School faculty and staff emphasized five key dimensions: communication, creativity, persistence, collaboration, and inquiry. The faculty assessed these areas within their interdisciplinary curriculum throughout the school year. 

As previously noted, PBL is sometimes referred to as inquiry-based learning and instruction. While there is no standard definition in research, Rizzo and Taylor (2016) offered a working definition of inquiry-based instruction: “inquiry-based instruction can operationally be observed as student- conducted experiments with the use of an inquiry-based instructional framework” (p.2). Studies of this type of intervention for SWD in STEM (e.g., Villanueva \& Hand, 2011) show indications of being effective, for example, the Science Writing Heuristic. This involves designing learning templates for SWD and teachers. Students typically write questions related to their work and experience in science classes and reflect on those activities and learning chan- ges over time. Teachers provide support in learning through activities in class or in labs and provide students opportunities to share the knowledge they have gained and how they understand concepts and main ideas, and about how they reflect on their learning. Inquiry-based approaches are increasingly being recommended for use as significant elements of support for SWD in different categories, such as specific learning disability, autism spectrum disorder, and emotional-behavioral disorder (e.g., Dexter et al., 2011; Therrien et al., 2014).

According to the National Center for Education Statistics (2020) the largest group of students in special education receiving services under the Individuals with Disabilities Education Act (IDEA) (P.L. 108-446) is students with a specific learning disability (SLD). Thus it is not surprising to see more research in the literature related to this group of SWD in inquiry-based interventions than for other groups. Seifert and Espin (2012) studied a small group of high-school students with SLD and reported direct instruction had significant positive effects on their ability to read science text and learn vocabulary but not for comprehension. Kaldenberg et al. (2015) conducted a meta-analysis of 20 studies of reading interventions (direct instruction, cognitive strategy instruction, and direction instruction) for middle and high school students with SLD. Across all studies, they reported medium to high effect sizes for explicit vocabulary instruction on students’ comprehension of expository science text. The authors concluded that vocabulary instruction would be effective for SWD in STEM curricula if it incorporated direct instruction, semantic mapping, and or mnemonics.         

Taylor et al. (2020) conducted a meta-analysis of 11 single-subject studies published from 2000 to 2018 of students with ASD (in elementary, middle, and high school) and science achievement. One study with high-school students (see Carnahan et al., 2016) analyzed the effectiveness of an intervention with graphic organizers and text-based strategies; it reported a large effect size. Another study (see Hart \& Whalon, 2012) analyzing self-management and technology-related strategies produced mixed results in science comprehension. The limitations of both studies was the very small numbers of participants across the studies. Reviewing a number of studies examining interventions for students with ASD and math instruction, King et al. (2016) concluded very few could be considered evidence-based, as the majority of approaches were one-to-one (i.e., teacher to student) and unlikely to be feasible in general education settings. Nevertheless, the authors concluded that any approach would have to utilize, at a minimum, explicit instruction with prompts and positive reinforcement. Israel et al. (2013) suggested teachers could effectively support SWD in STEM by utilizing graphic organizers or drawing pictures of concepts that the students are trying to learn using explicit instruction and supporting recall of concepts and vocabulary to improve comprehension. The overall goal with this approach is to facilitate the growth of students’ literacy and language that translates to an improved understanding of complex STEM concepts.     

Hwang and Taylor (2016) emphasized the importance of (a) taking an interdisciplinary approach to supporting SWD in STEM, (b) collaborating across each STEM area, and (c) making direct connections to other disciplines, such as reading and literature. They endorsed an approach that is gaining favor in some pedagogical circles called “STEAM”. Developed at the Rhode Island School of Design, the “A” represents the inclusion of “arts” to the STEM acronym. The authors believed that incorporating the arts strengthens the curriculum for SWD because they: (a) motivate students especially when accessing the difficult aspects of STEM; (b) provide opportunities for self-express- ion, an important element in learning; and (c) serve as the scaffolding needed for SWD to learn abstract and theoretical concepts in STEM. 

Moorehead and Grillo (2013) proposed that teachers could include and accommodate more SWD in general education STEM classrooms by following a co-teaching model. Such a model would allow each teacher to focus on their respective strengths – the general educator as content knowledge specialist and the special educator as differentiation specialist. Reporting on a qualitative meta-synthesis of more than two dozen studies about co-teaching in inclusive classrooms, Scruggs et al. (2007) concluded that co-teaching while favorably viewed by teachers also presents challenges relating to the classroom, students, and school administration (see also Mastropieri \& Scruggs, 2001). There were, however, two particular issues also noted by the researchers: (1) Special educators too often serving more of a subordinate function rather than as a true “co-” teacher; and (2) What is established in the research literature as effective practices for SWD, such as mnemonics, self-monitoring, peer mentoring, were often not being utilized.      

\subsubsection*{Using Technology and Collaboration Among Educators}
The expansion in this particular area of research has followed the advancement of technologies in terms of complexity and sophistication. In particular, computer technology has often been a favorite instructional tool for educators in student cognitive motivation and engagement, to deliver individualized lessons and integrate subjects, and to support students. As a means of enhancing access to STEM fields for SWD and other traditionally underrepresented students, Israel et al. (2015) recommended computing and computational thinking to tailor K-12 instruction in STEM education. The authors pointed to some benefits of this approach, including improvements in (a) collaborative problem solving, (b) attitude about computer science, (c) higher-order thinking skills, and (d) creation of applied, real-world contexts for teaching algorithmic problem solving. In terms of computing instruction and education for SWD, the use of UDL can play a key role through multiple means of representation, expression and action, and student engagement. Research also notes the importance of striking a balance between open inquiry with explicit instruction and student collaboration and cooperative learning, and using various computer hardware and software to help SWD access content (Israel et al., 2015). 

Technology also supports the design of an inclusive learning environment for SWD, by accommodating many different needs. By integrating learning and teaching through UDL, students get to use an array of technological applications and media platforms (e.g., podcasts, YouTube) through differentiated instruction, cooperative learning, embedded assessment, and self-monitor- ing. In terms of using UDL principles, technology, and pedagogical training, Marino et al. (2010) reported that there were some benefits of applying UDL principles to the use of video to deliver curriculum in astronomy class. Isaacson and Michaels (201) evaluated Math Speak, a system for speaking mathematical expressions in a non-ambiguous manner, as a system for teaching math to SWD, and reported that it was generally effective in communicating math and chemistry concepts to students with blindness and visual impairments. 

Marino and Beecher (2010) analyzed a Response-to-Intervention (RTI) approach incorporating video games to support STEM learning for students with SLD. The authors found that video games aided teachers in progress monitoring; the video-game interface provided a way for teachers to collect assessment data in real time. Students had received video tutorials prior to playing games and learned from ‘experts’ who modeled how to play the games. Students were then prompted to figure out answers to sets of challenges, the completion of which would indicate their level of mastery. Students spent about half an hour per week playing video games – used as enhancement and not as a replacement of instruction – to engage them in scientific inquiry and provide opportunities for independent exploration of scientific content.  

\subsubsection*{Professional Development of Educators}
A greater research focus on STEM and SWD has more recently included the professional development (PD) of educators and how they are being trained to keep pace with ongoing changes in STEM learning to support SWD. Yore and Treagust (2006) placed a greater focus and emphasis on teachers’ understanding the importance of students’ vocabulary acquisition in science learning and literacy. Along those lines, Taylor et al. (2020, 2011) recommended inquiry-based instruction for SWD in STEM be an integral part of teachers’ pre-service training and PD. This included explicit instruction in science vocabulary acquisition and retention, and the use of direct instruction, and mnemonics, as these elements have accumulated evidence in the research literature for their effectiveness in instructional supports for SWD.   

In a qualitative study of early-childhood teacher candidates enrolled in science methods and special education courses, Khan et al. (2017) reported that teacher candidates tended to primarily rely on SWD seeking help from other students before and after instruction, rather than being directly involved in designing environments and developing supports for SWD that would more likely foster student autonomy. This perhaps points to the need in teacher training and PD to consider more broadly the adaptations for SWD in science lesson planning and approaches.         

In addition to standard modes of in-service PD, there are other supplemental programs outside of the school environment that can offer PD opportunities especially for STEM and special education teachers. These can be annual programs or more specialized ones, such as “Sci Train”, a project funded by the NSF designed to provide high-school science and math teachers effective methods of instruction, including the understanding and application of modifications and accommodations, and developing a resource library of these methods (Moon et al., 2012).  

Israel et al. (2015) advocated for the incorporation of computational thinking and computer programming into PD for classroom instruction and lessons. They reported that ongoing PD and embedded coaching increased computing-education participation by students; and peer-collaboration, modeling, and scaffolding were beneficial for student learning in computer education. They also pointed out the benefit to SWD in terms of allowing multiple means of expression and representation, to engage students using both explicit instruction and open-inquiry approaches. In that vein, describing a new framework of PD for educators in STEM, Fore, et al. (2015) introduced the concept of subjectivity. “Subjectivity refers to ‘the relation of self – comprising one's emergent truths, desires, practices, and perspectives – to itself, to others, and to the influence [i.e. power] present in a variety of encounters, whether social, political, economic, or religious’” (p. 102). The authors argued for a move away from the traditional classroom approach, which in their view ignored or hindered the essential element of innovative teaching, the ability of teachers to utilize their experiences, knowledge, and other uni- que factors that can be utilized to teach STEM.    

\section*{DISCUSSION}
We conducted a qualitative review of research from the past twenty years to understand what the growth in STEM in the U.S. has meant in terms of the educational and career goals and opportunities for high school SWD. Therefore we searched and selected 53 articles, and through our analyses of those articles several themes emerged to answer two specific questions: (1) How are high-school SWD prepared for careers in STEM? (2) How are educators prepared to support high-school SWD for opportunities in STEM? We now turn to explaining those themes, as well as describing the limitations of this review and the implications of this review for the field. 

\subsection*{Qualitative Themes and Research Gaps}
Taken together, the four themes that answer the first question and the three themes that answer the second question form an interesting, albeit preliminary, narrative regarding high-school SWD and STEM. Half of this narrative focuses on high-school SWD and the types of supports they would need in order to mitigate the effects of barriers in STEM and enable access to STEM opportunities, the development of a STEM identity through those opportunities, and gaining the necessary knowledge and experiences to find a pathway to a STEM career in ‘traditional’ or ‘hidden’ STEM. The other half of the narrative focuses on educators and their pivotal role in the intersection of high-school SWD and STEM through individualizing student learning supports, using technology and collaborating among other educators to provide or augment those supports, and engaging in regular professional development that are specific to STEM and high-school SWD, post-high school transition, and STEM career pathways. Research from the past twenty years has shown what high-school SWD could achieve in STEM, nevertheless, there remain disparities between students with and without disabilities as well as gaps in research knowledge, which can and should be investigated in future research.

\subsection*{Limitations of This Qualitative Review}
In any research study there are limitations; this qualitative review was no different. First, although we used search terms in various combinations and searched multiple databases, it is still possible that our search was too narrow. Second, while we followed an inclusion criteria driven by specific questions, it is likely that additional and or different patterns could be identified by others who reviewed the same literature. Third, the studies that were research reports varied in rigor; only a few of these are likely to meet the strict “What Works Clearinghouse” criteria (\url{https://ies.ed.gov/ncee/wwc/}) for establishing evidence-based practices in education.   

\subsection*{Implications of This Qualitative Review}
Despite the limitations of this qualitative literature review, there are important implications for the field, including researchers and practitioners in education. For researchers, our review has revealed that there is room for much further investigation in a few key areas. One area is research into STEM enrichment programs (e.g., “Camp Can Do’), which provide out-of-school seasonal or periodic STEM learning opportunities for SWD based on disability category, gender, and race/ethnicity. This type of intersectionality research will also show the complexity of STEM learning, which manifests in different growth trajectories for high-school SWD based on these demographic characteristics (see Wei et al., 2012). Second area of research is STEM identity and social-emotional learning, and how they influence SWD readiness to pursue STEM career pathways. These longitudinal studies will be key to understanding how SWD develop the necessary resilience and persistence over time, from high school and into college/university, and career in STEM. The third area involves STEM-focused high-school transition services in special education and their link to post-high school outcomes of SWD (i.e., IDEA requirement Indicator 14). This is an important area of special education that is becoming a greater focus of administrators and policymakers. Because high-school SWD are legally entitled to these services, conducting research to more closely analyze how IEP (Individualized Education Program) are structured for high-school SWD to prepare for post-high school STEM (i.e., college/university, career) could lead to developing best practices in transition services.   

Our review of the research literature has also revealed that there are implications for practitioners. There is clearly a need for pre-service training and in-service professional development of high-school special educators in STEM pedagogy. Because STEM continues to evolve and technology continues to advance, educators need to stay current with their content- area knowledge and pedagogy for high-school SWD in special education and general education classrooms. The continued growth in STEM fields also likely means that individuals’ life-long commitment to STEM learning is necessary given the constant, often daunting competition in these fields. Lastly, our review of the research literature also indicates that for educators in general and special education classrooms, the utilization of technology and active collaboration in STEM instruction are positive ways to support high-school SWD in STEM coursework and increasing opportunities for in-school STEM learning. 

\end{large}
\clearpage
\section*{REFERENCES}\par 

\leftskip 0.25in
\parindent -0.25in 

*Indicates studies reviewed in this paper 

*Bargerhuff, M.E. (2013). Meeting the needs of students with disabilities in a stem school. \textit{American Secondary Education,} 41(3), 3-20.  \url{https://doi.org/10.1002/tea.21437}

*Basham, J.D., \& Marino, M.T. (2013). Understanding STEM education and supporting students through universal design for learning. \textit{Teaching Exceptional Children,} 45(4), 8-15. \url{https://doi.org/10.1177\%2F004005991304500401}

*Basham, J.D., Israel, M., \& Maynard, K. (2010). An ecological model of STEM education: Operationalizing STEM for all. Journal of \textit{Special Education Technology,} 25(3), 9-19. \url{https://doi.org/10.1177\%2F016264341002500303}

Bogdan, R.C., \& Biklen, S.K. (2007). \textit{Qualitative research for education: An introduction to theories and methods} (5th ed.). San Francisco, CA: Pearson Education, Inc. 

*Boyle, J.R. (2012). Note-taking and secondary students with learning disabilities: Challenges and solutions. \textit{Learning Disabilities Research \& Practice,} 27(2), 90-101. \url{https://doi.org/10.1177/0731948711435794}

Brantlinger, E., Jimenez, R., Klingner, J., Pugach, M., \& Richardson, V. (2005). Qualitative studies in special education. \textit{Exceptional Children,} 71(2), 195-207.

*Brigham, F.J., Scruggs, T.E., \& Mastropieri, M.A. (2011). Science education and students with learning disabilities. \textit{Learning Disabilities Research \& Practice,} 26(4), 223-232. \url{https://psycnet.apa.org/doi/10.1111/j.1540-5826.2011.00343.x}

Bureau of Labor Statistics (2020). \textit{Employment projections: Employment in STEM occupations.} \url{https://www.bls.gov/emp/tables/stem-employment.htm}

Bureau of Labor Statistics (2017). \textit{STEM Occupations: Past, Present, and Future} (Spotlight on Statistics by Stella Fayer, Alan Lacey, \& Audrey Watson). Retrieved from \url{https://www.bls.gov/spotlight/2017/science-technology-engineering-and-mathematics-stem-occupations-past-present-and-future/home.htm}

*Byars-Winston, A. (2014). Toward a framework for multicultural STEM-focused career interventions. \textit{The Career Development Quarterly,} 62(4), 340-357.  \url{https://doi.org/10.1002/j.2161-0045.2014.00087.x}

*Carnahan, C.R., Williamson, P., Birri, N., Swoboda, C., \& Snyder, K.K. (2016). Increasing comprehension of expository science text for students with autism spectrum disorder. \textit{Focus on Autism and Other Developmental Disabilities, 31}(3), 208-220. \url{https://doi.org/10.1177/1088357615610539}

*Cease-Cook, J., Fowler, C., \& Test, D.W. (2015). Strategies for creating work-based learning experience in schools for secondary students with disabilities. \textit{Teaching Exceptional Children,} 47(6), 352-358. \url{https://doi.org/10.1177/004005991558003}

Christenson, J. (2011). Ramaley coined STEM term now used nationwide. Winona Daily News, November 13, 2011. \url{https://www.winonadailynews.com/news/local/ramaley-coined-stem-term-now-used-nationwide/article\_457afe3e-0db3-11e1-abe0-001cc4c03286.html}

Chute, E. (2009). STEM education is branching out. \textit{Pittsburgh Post-Gazette,} February 10, 2009. \url{https://www.post-gazette.com/news/education/2009/02/10/STEM-education-is-branching-out/stories/200902100165}

Cohen, J. (1960). A coefficient of agreement for nominal scales. \textit{Educational and Psychological Measurement,} 20, 37–46. 

*Dexter, D.D., Park, Y.J., \& Hughes, C.A. (2011). A meta-analytic review of graphic organizers and science instruction for adolescents with learning disabilities: Implications for the intermediate and secondary science classroom. \textit{Learning Disabilities Research,} 26(4), 204-213. \url{https://doi.org/10.1111/j.1540-5826.2011.00341.x}

*Dunn, C., Rabren, K.S., Taylor, S.L., \& Dotson, C.K. (2012). Assisting students with high-incidence disabilities to pursue careers in science, technology, engineering, and mathematics. \textit{Intervention in School and Clinic,} 48(1), 47-54. \url{https://doi.org/10.1177/1053451212443151}
 
Durlak, J.A., Weissberg, R.P., Dymnicki, A.B., Taylor, R.D., \& Schellinger, K. (2011). The impact of enhancing students’ social and emotional learning: A meta-analysis of school-based universal interventions. \textit{Child Development,} 82 (1), 405-432. \url{https://doi.org/10.1111/j.1467-8624.2010.01564.x}
 
*Falkenheim, J., Burke, A., Muhlberger, P., \& Hale, K. (2017). \textit{Women, Minorities, and Persons with Disabilities in Science and Engineering: 2017.} National Science Foundation, National Center for Science and Engineering Statistics. 2017. Special Report NSF 17-310. Arlington, VA. \url{https://www.nsf.gov/statistics/wmpd/}

*Fore, G.A., Feldhaus, C.R., Sorge, B.H., Agarwal, M., \& Varahramyan, K. (2015). Learning at the nano-level: Accounting for complexity in the internalization of secondary STEM teacher professional development. \textit{Teaching and Teacher Education,} 51, 101-112. \url{http://dx.doi.org/10.1016/j.tate.2015.06.008}

Gerlach, J. (2012). \textit{STEM: Defying a simple definition.} National Science Teachers Association (NSTA) Web Digest. NSTA Reports April 11, 2012. 
 
*Gottfried, M.A., Bozick, R., Rose, E., \& Moore, R. (2016). Does career and technical education strengthen the STEM pipeline? Comparing students with and without disabilities. \textit{Journal of Disability Policy Studies,} 26(4), 232-244. \url{https://doi.org/10.1177/1044207314544369}

*Gregg, N., Galyardt, A., Wolfe, G., Moon, N., \& Todd, R. (2017). Virtual mentoring and persistence in STEM for students with disabilities. Career Development and Transition for Exceptional Children, 40(4), 205-214. \url{http://doi.org/10.1177/2165143416651717}

Hall, T.E., Meyer, A., \& Rose, D.H. (2012). \textit{Universal design for learning in the classroom: Practical applications.} New York, NY: The Guilford Press.  

*Hart, J.E., \& Whalon, K.J. (2012). Using video self-modeling via iPads to increase academic responding of an adolescent with autism spectrum disorder and intellectual disability. \textit{Education and Training in Autism and Developmental Disabilities,} 47(4), 438–446. \url{https://www.learntechlib.org/p/50203/}

Holmlund, T.D., Lesseig, K., \& Slavit, D. (2018). Making sense of “STEM education” in K-12 contexts. \textit{International Journal of STEM Education,} 5(32), 1-18. \url{https://doi.org/10.1186/s40594-018-0127-2}

Honey, M., Pearson, G., \& Schweingruber, H. (2014). \textit{STEM integration in K-12 education: status, prospects, and an agenda for research.} National Academy of Engineering and National Research Council. The National Academies Press: Washington, DC. \url{https://doi.org/10.17226/18612}

*Hwang, J., \& Taylor, J.C. (2016). Stemming on STEM: A STEM education framework for students with disabilities. \textit{Journal of Science Education,} 19(1), 39-49. \url{https://doi.org/10.14448/jsesd.09.0003}

Individuals with Disabilities Education Act, P.L. 108-446, codified as amended at title U.S.C. §20 U.S.C. 1400(d).  

*Isaacson, M.D., \& Michaels, M. (2015). Ambiguity in speaking chemistry and other STEM content: Educational implications. \textit{Journal of Science Education for Students With Disabilities,} 18(1), 1-9. \url{https://doi.org/10.14448/jced.07.0001}

*Israel, M., Maynard, K., \& Williamson, P. (2013). Promoting literacy – Embedded, authentic STEM instruction for students with disabilities and other struggling learners. \textit{Teaching Exceptional Children,} 45(4), 18-25. \url{https://doi.org/10.1177/004005991304500402}

*Israel, M., Wherfel, Q.M., Pearson, J., Shehab, S., \& Tapia, T. (2015). Empowering K-12 students with disabilities to learn computational thinking and computer programming. \textit{Teaching Exceptional Children,} 48(1), 45-53. \url{https://doi.org/10.1177/004005991559479}

*Izzo, M.V., Murray, A., Priest, S., \& McArrell, B. (2011). Using student learning communities to recruit STEM students with disabilities. \textit{Journal of Postsecondary Education and Disability,} 24(4), 301-316. \url{https://files.eric.ed.gov/fulltext/EJ966131.pdf}

*Kaldenberg, E.R., Watt, S.J., \& Therrien, W.J. (2015). Reading instruction in science for students with learning disabilities: A meta-analysis. \textit{Learning Disability Quarterly,} 38(3), 160-173, \url{https://doi.org/10.1177/0731948714550204}

*Kahn, S., Pigman, R., \& Ottley, J. (2017). A tale of two courses: Exploring teacher candidates’ translation of science and special education methods instruction into inclusive science practices. \textit{Journal of Science Education for Students with Disabilities,} 20(1), Article 6, 50-68. \url{http://scholarworks.rit.edu/jsesd/vol20/iss1/6}

*King, S.A., Lemons, C.J., \& Davidson, K.A. (2016). Math interventions for students with autism spectrum disorder: A best-evidence synthesis. \textit{Exceptional Children,} 82(4), 443-462. \url{https://doi.org/10.1177\%2F0014402915625066}

*Leddy, M.H. (2010). Technology to advance high school and undergraduate students with disabilities in science, technology, engineering, and mathematics. \textit{Journal of Special Education Technology,} 25(3), 3-8. \url{https://www.learntechlib.org/p/113978/}

*Marino, M.T., \& Beecher, C.C. (2010). Conceptualizing RTI in 21st-century secondary science classrooms: Video games’ potential to provide tiered support and progress monitoring for students with learning disabilities. \textit{Learning Disability Quarterly,} 33, 299-311. \url{https://doi.org/10.1177\%2F073194871003300407}

*Mason, L. H., \& Hedin, L. R. (2011). Reading science text: Challenges for students with learning disabilities and considerations for teachers. Learning Disabilities: Research \& Practice, 26, 214–222. \url{https://doi.org/10.1111/j.1540-5826.2011.00342.x}

*Mastropieri, M., \& Scruggs, T.E. (2001). Promoting inclusion in secondary classrooms. \textit{Learning Disability Quarterly,} 24(4), 265-274. \url{https://doi.org?10.2307/1511115}

*Mastropieri, M.A., Scruggs, T.E. \& Graetz, J. (2005). “Cognition and learning in inclusive high school chemistry classes”. In \textit{Advances in learning and behavioral disabilities: Vol. 18. Cognition and learning in diverse settings,} Edited by: Scruggs, T. E. and Mastropieri, M. A. 107–118. Oxford, United Kingdom: Elsevier. 

*Mau, W-C. J., \& Li, J. (2018). Factors influencing STEM career aspirations of underrepresented high school students. \textit{The Career Development Quarterly,} 66(3), 246-258. \url{https://doi.org/10.1002/cd.150}

McComas W.F. (2014) STEM: Science, Technology, Engineering, and Mathematics. In: McComas W.F. (eds.). \textit{The Language of Science Education.} SensePublishers, Rotterdam. \url{https://doi.org/10.1007/978-94-6209-497-0\_92}

McHugh, M.L. (2012). Interrater reliability: The Kappa statistic. \textit{Biochemia Medica,} 22(3), 276-282. \url{https://doi.org:10.11613/BM.2012.031}

Miles, M.B., Huberman, A.M., \& Saldaña, J. (2014). \textit{Qualitative data analysis: A methods sourcebook} (3rd ed). Los Angeles, CA: Sage Publications.  

*Moon, N.W., Todd, R.L., Morton, D.L., \& Ivey, E. (2012). \textit{Accommodating students with disabilities in science, technology, engineering, and mathematics (STEM): Findings from research and practice for middle grades through university education.} A publication of SciTrain: Science and Math for All, sponsored by the National Science Foundation (Award No. 0622885). Atlanta, GA: Center for Assistive Technology and Environmental Access, College of Architecture, Georgia Institute of Technology. \url{https://hourofcode.com/files/accommodating-students-with-disabilities.pdf}

*Moorehead, T., \& Grillo, K. (2013). Celebrating the reality of inclusive STEM education: Co-teaching in science and mathematics. \textit{Teaching Exceptional Children,} 54(4), 50-57. \url{https://journals.sagepub.com/doi/pdf/10.1177/004005991304500406}

National Center for Education Statistics. (2015). \textit{National assessment of educational progress 4th grade science assessment.} Washington, D.C.: Institute of Education Sciences.

National Center for Education Statistics. (2020). \textit{Students with Disabilities: Percentage distribution of students ages 3–21 served under the Individuals with Disabilities Education Act (IDEA), by disability type: School year 2018–19.}

National Education Association (2019). Research spotlight on project-based learning: NEA review of the research on best practices in education. \url{http://www.nea.org/tools/16963.htm}

National Science Foundation (2018). \textit{Dear colleague letter: Research to improve STEM teaching and learning, and workforce development for persons with disabilities.} (December 20, 2018). NSF Document 19-033. \url{https://www.nsf.gov/pubs/2019/nsf19033/nsf19033.jsp}

National Science Foundation (2021). \textit{Women, minorities, and persons with Disabilities in science and engineering.} National Center for Science and Engineering Statistics. Special Report NSF 21-321. Alexandria, VA. \url{https://ncses.nsf.gov/wmpd}

National Science Foundation (2019). \textit{Women, minorities, and persons with disabilities in science and engineering.} National Center for Science and Engineering Statistics, Directorate for Social, Behavioral and Economic Sciences. \url{https://ncses.nsf.gov/pubs/nsf19304/digest}

Onwuegbuzie, A.J., Leech, N.L., \& Collins, K.M.T. (2012). Qualitative analysis techniques for the review of the literature. \textit{Qualitative Report,} 17(56), 1-28. \url{http://www.nova.edu/ssss/QR/QR17/onwuegbuzie.pdf}

Paré G., Trudel, M.C., Jaana, M., \& Kitsiou, S. (2015) Synthesizing information systems knowledge: A typology of literature reviews. \textit{Information \& Management.} 52(2), 183-199. \url{http://dx.doi.org/10.1016/j.im.2014.08.008}

*Peters-Burton, E.E., Lynch, S.J., Behrend, T.S., \& Means, B.B. (2014). Inclusive STEM high school design: 10 critical components. \textit{Theory Into Practice,} 53, 64-71. \url{https://doi.org/10.1080/00405841.2014.862125}

*Plasman, J.S., \& Gottfried, M.A. (2018). Applied STEM coursework, high school dropout rates, and students with learning disabilities. \textit{Educational Policy,} 32(5), 664-696. \url{https://doi.org/10.1177/0895904816673738}

QSR (2019). \textit{NVivo Qualitative Data Analysis Software} (Version 12) [Computer Software]. Burlington, MA: QSR International (Americas) Inc.

*Rakich, S.S., \& Tran, V. (2016) A Balanced Approach to Building STEM College and Career Readiness in High School: Combining STEM Intervention and Enrichment Programs. \textit{European Journal of STEM Education,} 1(3), p.59. \url{http://dx.doi.org/10.20897/lectito.201659}

*Rizzo, K.L., \& Taylor, J.C. (2016). Effects of inquiry-based instruction on science achievement for students with disabilities: An analysis of the literature. \textit{Journal of Science Education for Students with Disabilities,} 19(1), Article 2, 1-16. \url{https://scholarworks.rit.edu/jsesd/vol19/iss1/2/}

Rossman, G.B., \& Rallis, S.F. (1998). \textit{Learning in the Field: An introduction to qualitative research.} Thousand Oaks, CA: Sage Publications, Inc. 

Rothwell, J. (2013). \textit{The hidden STEM economy: The metropolitan policy program at Brookings.} Washington, DC: Brookings Institution. \url{https://www.brookings.edu/research/the-hidden-stem-economy}

*Rule, A.C., \& Stefanich, G.P. (2012). Using a thinking skills system to guide discussions during a working conference on students with disabilities pursuing STEM fields. \textit{Journal of STEM Education,} 13(1), 43-54. \url{ https://gseuphsdlibrary.files.wordpress.com/2013/03/using-a-thinking-skills-system-to-guide-discussions-during-a-working-conference-on-students-with-disabilities-pursuing-stem-fields.pdf}

Sanders, M. (2009). STEM, STEM education, STEMmania. \textit{The Technology Teacher,} 68(4),20-26. 

*Scruggs, T.E., \& Mastropieri, M.A. (2007). Science learning in special education: The case for 
constructed versus instructed learning. \textit{Exceptionality,} 15, 57-74. \url{https://doi.org/10.1080/09362830701294144}

*Scruggs, T.E., Mastropieri, M.A., \& McDuffie, K.A. (2007). Co-teaching in inclusive classrooms: A meta-synthesis of qualitative research. \textit{Exceptional Children, 73}(4), 392-416. \url{https://doi.org/10.1177\%2F001440290707300401}

*Scruggs, T.E., Mastropieri, M.A., \& Okolo, C.M. (2008). Science and social studies for students with disabilities. \textit{Focus on Exceptional Children,} 41(2), 1-24.

*Seifert, K., \& Espin, C. (2012). Improving reading of science text for secondary students with learning disabilities: Effects of text reading, vocabulary learning, and combined approaches to instruction. \textit{Learning Disability Quarterly,} 35(4), 236-247. \url{https://doi.org/10.1177/0731948712444275}

Semega, J., Kollar, M., Creamer, J., \& Mohanty, A. (2019). \textit{Income and poverty in the United States: 2018} (Current Population Reports P60-266). Washington, D.C.: U.S. Census Bureau (U.S. Government Printing Office). \url{https://www.census.gov/content/dam/Census/library/publications/2019/demo/p60-266.pdf}

*Shoffner, M.F., Newsome, D., Minton, C.A., \& Morris, C.A. (2015). A qualitative exploration of the STEM career-related outcome expectations of young adolescents. \textit{Journal of Career Development,} 42(2), 102-116. \url{https://doi.org/10.1177/0894845314544033}

Snyder, H. (2019). Literature review as a research methodology: An overview and guidelines. \textit{Journal of Business Research,} 104, 333-339.

*Sublett, C., \& Plasman, J.S. (2017). How does applied STEM coursework relate to mathematics and science self-efficacy among high school students? Evidence from a national sample. \textit{Journal of Career and Technical Education,} 32(1), 29-50. \url{http://doi.org/10.21061/jcte.v32i1.1589}

*Subramaniam, M.M., Ahn, J., Fleischmann, K.R., \& Druin, A. (2012). Reimagining the role of school libraries in STEM education: Creating hybrid spaces for exploration. \textit{The Library Quarterly: Information, Community, Policy,} 82(2), 161-182. \url{https://doi.org/10.1086/664578}

*Supalo, C.A., Isaacson, M.D., \& Lombardi, M.V. (2014). Making hands-on science learning for students who are blind or have low vision. \textit{Journal of Chemical Education,} 91, 195-199. \url{https://doi.org/10.1021/ed3000765}

*Supalo, C.A., Wohlers, H.D., \& Humphrey, J.R. (2011). Students with blindness explore chemistry at ‘Camp Can Do’. \textit{Journal of Science Education for Students with Disabilities,} 15(1), 1-9. \url{https://doi.org/:10.14448/jsesd.04.0001}

*Taylor, J.C., Rizzo, K.I., Hwang, J., \& Hill, D. (2020). A review of research on science instruction for students with autism spectrum disorder. \textit{School Science and Mathematics,} 120, 116-125. \url{https://doi.org/10.1111/ssm.1238}

*Therrien, W.J., Taylor, J.C., Hosp, J.L., Kaldenberg, E.R., \& Gorsh, J. (2011). Science instruction for students with learning disabilities: A meta-analysis. \textit{Learning Disabilities Research \& Practice,} 26(4), 188-203. \url{https://doi.org/10.1111/j.1540-5826.2011.00340.x}

*Therrien, W.J., Taylor, J.C., Watt, S., \& Kaldenberg, E. (2014). Science instruction for students with emotional and behavioral disorders. \textit{Remedial and Special Education,} 35(1), 15-27. \url{https://doi.org/10.1177/0741932513503557}

Thurston, L.P., Shuman, C., Middendorf, B.J., \& Johnson, C. (2017). Postsecondary STEM education for students with disabilities: Lessons learned from a decade of NSF funding.\textit{ Journal of Postsecondary Education and Disability,} 30(1), 49-60. 
\url{https://files.eric.ed.gov/fulltext/EJ1144615.pdf}

*Villanueva, M.G., \& Hand, B. (2011). Science for all: Engaging students with special needs in and about science. \textit{Learning Disabilities Research \& Practice,} 26(4), 233–240. \url{https://doi.org/10.1111/j.1540-5826.2011.00344.x}

*Wang, M-T., \& Degol, J.L. (2017). Gender gap in science, technology, engineering, and mathematics (STEM): Current knowledge, implications for practice, policy, and future directions. \textit{Educational Psychology Review,} 29(1), 119-140. \url{https://doi.org/10.1007/s10648-015-9355-x}

*Wei, X., Lenz, K. B., \& Blackorby, J. (2012). Math growth trajectories of students with disabilities: Disability category, gender, racial, and socioeconomic status differences from ages 7 to 17. \textit{Remedial and Special Education,} 34(3), 154-165. \url{https://doi.org/10.1177/0741932512 448253}

*Williams Jr., T.O., Ernst, J.V., \& Kaui, T.M. (2015). Special populations at-risk for dropping out of school: A discipline-based analysis of STEM educators. \textit{Journal of STEM Education,} 16(1), 41-45. \url{https://vtechworks.lib.vt.edu/bitstream/handle/10919/76635/ WilliamsSpecialPopulations2015.pdf?sequence=1\&isAllowed=y}

*Yore, L.D., \&Treagust, D.F. (2006). Current realities and future possibilities: Language and science literacy—em\-powering research and informing instruction. \textit{International Journal of Science Education,} 28(2-3), 291-314. \url{https://doi.org/10.1080/09500690500336973}
 
\clearpage

\begin{@twocolumnfalse}
\RaggedRight
\begin{table}
\caption{\textit{Summary of Articles (N=53) Selected for the Qualitative Literature Review}}
\label{tab:my-table}
\begin{tabular}{m{0.72in}m{0.72in}m{0.72in}m{0.72in}m{0.72in}m{0.72in}m{0.72in}m{0.72in}}
\hline
Authors                                                 & Journal                                                         & Design                                & Sample                                                                                       & Disability                                        & Ethnicity                                                                                              & Gender                                                           & Topics                                                                                                   \\ \hline
Bargerhuff (2013)                                       & American Secondary Education                                    & Qualitative                           & education professionals (N=9)                                                                & \centering  \centering Not Available                                    &  \centering Not Available                                                                                          & females (\textit{n}=7), males (\textit{n}=2)                                       & Project- based STEM learning, problem-solving                                                            \\ \hline
Basham \& Marino (2013)                                 & Teaching Exceptional Children                                   & Position paper                        &  \centering Not Applicable                                                                               &  \centering Not Applicable                                    &  \centering Not Applicable                                                                                         &  \centering Not Applicable                                                   & UDL in STEM instruction                                                                                  \\ \hline
Boyle (2012)                                            & Learning Disabilities Research \& Practice                      & Literature review                     &  \centering Not Applicable                                                                               & Learning Disability                               &  \centering Not Applicable                                                                                         &  \centering Not Applicable                                                   & High school content areas, note-taking                                                                   \\ \hline
Brigham, Scruggs, \& Mastropieri (2011)                 & Learning Disabilities Research \& Practice                      & Literature review                     &  \centering Not Applicable                                                                               & Learning disability                               &  \centering Not Applicable                                                                                         &  \centering Not Applicable                                                   & Validated science teaching strategies                                                                    \\ \hline
Byars-Winston (2014)                                    & The Career Development Quarterly                                & Essay                                 &  \centering Not Applicable                                                                               &  \centering Not Applicable                                    &  \centering Not Applicable                                                                                         &  \centering Not Applicable                                                   & STEM \& career, under-represented groups                                                                 \\ \hline
Carnahan, Williamson, Birri, Swoboda, \& Snyder (2016)  & Focus on Autism and Other Developmental Disabilities            & Quantitative (single subject)         & high school students (\textit{n}=3), \& special education teacher (\textit{n}=1)                               & Autism, ASD                                       &  \centering Not Available                                                                                          & males (N=3) ages 15-16                                           & Science text reading                                                                                     \\ \hline
Cease-Cook, Fowler, \& Test (2015)                      & Teaching Exceptional Children                                   & Position paper                        &  \centering Not Applicable                                                                               &  \centering Not Applicable                                    &  \centering Not Applicable                                                                                         &  \centering Not Applicable                                                   & Work-based learning, career exploration                                                                  \\ \hline
Dexter, Park, \& Hughes (2011)                          & Learning Disabilities Research                                  & Meta- analysis                        & 7 studies of high school students (total N=232)                                              & Learning disability                               &  \centering Not Available                                                                                          &  \centering Not Available                                                    & Graphic organizers                                                                                       \\ \hline
Dunn, Rabren, Taylor, \& Dotson (2012)                  & Intervention in School and Clinic                               & Qualitative (case study)              & 1 high school student                                                                        & Learning disability, Other health impairment      &  \centering Not Available                                                                                          & 1 male, transition age                                           & STEM \& transition planning                                                                              \\ \hline
Falkenheim, Burke, Muhlberger, \& Hale (2017)           & National Center for Science and Engineering Special             & Quantitative (extant data analysis)   & Multiple national samples – NSF, Dept. of Labor, Census Bureau, Department of Education      & Disabilities not listed by individual categories  & Black, Native American, Alaska Native, Native Hawaiian, Other Pacific Islander, Hispanic, Asian, White & females \& males                                                 & Women, ethnic minorities, \& persons with disabilities in science \& engineering education and employment \\ \hline
Fore, Feldhaus, Sorge, Agarwal, \& Varahramyan (2015)   & Teacher and Teacher Education                                   & Qualitative                           & high-school teachers (N=13)                                                                  &  \centering Not Available                                     & Black (\textit{n}=1), Hispanic (\textit{n}=1), \& White (\textit{n}=11) teachers                                                  & females (\textit{n}=6), males (\textit{n}=7)                                       & STEM instruction, teacher professional development                                                       \\ \hline
Gottfried, Bozick, Rose, \& Moore (2016)            & Journal of Disability Policy Studies                            & Quantitative (extant data analysis)   & National Longitudinal Survey of Youth                                                        & 2405 parents of students with disabilities        & 24.3\% Black, 16.2\% Hispanic, 1.2\% Mixed Race, \& 58.4\% White                                       & 45.4\% female, 54.6\% male                                       & CTE, STEM, school-based experiential programs                                                            \\ \hline
Gregg, Galyardt, Wolfe, Moon, \&  Todd (2017)        & Career Development and Transition for Exceptional Individuals   & Quantitative (survey)                 & High school students from urban, suburban, and rural schools in Georgia (N=91)               & ASD (\textit{n}=54), ADHD (\textit{n}=15), ASD (\textit{n}=12), Other (\textit{n}=10) & Minority (\textit{n}=61), White (\textit{n}=30)                                                                          & females (\textit{n}=40), males (\textit{n}=51)                                     & Virtual mentoring, STEM persistence, self-advocacy, self-efficacy                                        \\ \hline
Hart \& Whalon (2012)                                & Education and Training in Autism and Developmental Disabilities & Quantitative (single subject)         & 1 high school student                                                                        & ASD, Other Health Impairment                      &  \centering Not Available                                                                                          & Male in 10th grade                                               & Video modeling, science instruction                                                                      \\ \hline
Hwang \&  Taylor (2016)                              & Journal of Science Education                                    & Position paper                        &  \centering Not Applicable                                                                               &  \centering Not Applicable                                    &  \centering Not Applicable                                                                                         &  \centering Not Applicable                                                   & Art \& STEM learning for students with disabilities                                                      \\ \hline
Isaacson \&   Michaels (2015)                     & Journal of Science Education for Students with Disabilities     & Practitioner paper                    &  \centering Not Applicable                                                                               & Blindness, low vision                             &  \centering Not Applicable                                                                                         &  \centering Not Applicable                                                   & STEM, ambiguous speech, Math Speak                                                                       \\ \hline
Israel, Maynard, \& Williamson (2013)                   & Teaching Exceptional Children                                   & Practitioner paper                    &  \centering Not Applicable                                                                               &  \centering Not Applicable                                    &  \centering Not Applicable                                                                                         &  \centering Not Applicable                                                   & STEM \& reading instruction, authentic learning, technology supports                                     \\ \hline
Israel, Wherfel, Pearson, Shehab, \&  Tapia (2015)   & Teaching Exceptional Children                                   & Practitioner paper                    &  \centering Not Applicable                                                                               &  \centering Not Applicable                                    &  \centering Not Applicable                                                                                         &  \centering Not Applicable                                                   & Computer programming \& UDL, explicit instruction \& open inquiry                                        \\ \hline
Izzo, Murray, Priest, \&  McArrell (2011)            & Journal of Postsecondary Education and Disability               & Mixed methods                         & high school students (N=62)                                                                  & ASD, Blindness, ADHD, SLD, Deafness, OHI          &  \centering Not Available                                                                                          & 70\%+ male                                                       & STEM learning communities, transition planning                                                           \\ \hline
Kaldenberg, Watt, \& Therrien (2015)                    & Learning Disability Quarterly                                   & Meta-analysis                         & 4 studies (N=112)                                                                            & Learning disability                               &  \centering Not Available                                                                                          &  \centering Not Available                                                    & Science instruction, inquiry-based learning                                                              \\ \hline
Kahn, Pigman, \&  Ottley (2017)                      & Journal of Science Education for Students with Disabilities     & Qualitative                           & new teaching candidates (N=26)                                                               &  \centering Not Available                                     &  \centering Not Available                                                                                          &  \centering Not Available                                                    & Inclusive science, UDL, content knowledge                                                                \\ \hline
King, Lemons, \& Davidson (2016)                        & Exceptional Children                                            & Literature review                     & 14 studies                                                                                   & ASD                                               &  \centering Not Available                                                                                          & 71\% male                                                        & Evidence-based math practices                                                                           \hline  \\
Leddy (2010)                                            & Journal of Special Education Technology                         & Literature review                     &  \centering Not Available                                                                                &  \centering Not Available                                     &  \centering Not Available                                                                                          &  \centering Not Available                                                    & STEM \& students with disabilities high school to college, STEM workforce                               \hline  \\
Marino \&  Beecher (2010)                            & Learning Disability Quarterly                                   & Position paper                        &  \centering Not Applicable                                                                               & Learning disability                               &  \centering Not Applicable                                                                                         &  \centering Not Applicable                                                   & Response to intervention, science education \& video games                                               \\ \hline
Mason, \& Hedin (2011)                                  & Learning Disabilities Research \& Practice                      & Literature review                     &  \centering Not Applicable                                                                               & Learning disability                               &  \centering Not Applicable                                                                                         &  \centering Not Applicable                                                   & Reading science text, instructional supports \& strategies                                               \\ \hline
Mastropieri \&  Scruggs (2001)                       & Learning Disability Quarterly                                   & Literature review                     &  \centering Not Applicable                                                                               & Learning disability                               &  \centering Not Applicable                                                                                         &  \centering Not Applicable                                                   & Inclusive classrooms, peer tutoring, co-teaching, strategy instruction                                   \\ \hline
Mastropieri, Scruggs, \& Graetz (2005)                  & Cognition \& Learning in Diverse Settings                       & Quantitative (quasi-experiment group) & High school chemistry classroom (N=10)                                                       & Learning disability                               &  \centering Not Available                                                                                          &  \centering Not Available                                                    & Inclusive classroom, peer tutoring versus teacher directed instruction                                   \\ \hline
Mau \&  Li (2018)                                    & The Career Development Quarterly                                & Quantitative (extant data analysis)   & National Longitudinal Survey of Youth (N=21444)                                              &  \centering Not Available                                     & 55.3\% White, 15.4\% Hispanic, 10.4\% Black, 8\% Asian, \& 10.9\% Other                                & 49.1\% Female \& 50.1\% Male                                     & Career Aspirations Model, factors for pursuing STEM career                                               \\ \hline
Moon, Todd, Morton, \&  Ivey (2012)                  & SciTrain: Science and Math for All                              & Literature review                     & High school science and math teachers                                                        &  \centering Not Available                                     &  \centering Not Available                                                                                          &  \centering Not Available                                                    & UDL \& STEM, assistive technology, inclusive teaching                                                    \\ \hline
Moorehead \&    Grillo (2013)                  & Teaching Exceptional Children                                   & Practitioner paper                    &  \centering Not Available                                                                                &  \centering Not Available                                     &  \centering Not Available                                                                                          &  \centering Not Available                                                    & Co-teaching STEM, station teaching                                                                       \\ \hline
Peters-Burton, Lynch, Behrend, \&   Means (2014)  & Theory Into Practice                                            & Literature review                     &  \centering Not Applicable                                                                               &  \centering Not Applicable                                    &  \centering Not Applicable                                                                                         &  \centering Not Applicable                                                   & Inclusive STEM high school, open enrollment, project-based learning                                      \\ \hline
Plasman \&  Gottfried (2018)                         & Educational Policy                                              & Quantitative (extant data analysis)   & Educational Longitudinal Study (N=9410, 10th grade)                                          & Learning disability                               & 66\% White, 11\% Black, 14\% Hispanic, 3\% Asian                                                       & 40\% female \& 60\% male                                         & STEM coursework, school-to-careers, dropout rates, STEM supports                                         \\ \hline
Rakich \&  Tran (2016)                               & European Journal of STEM Education                              & Position paper                        &  \centering Not Applicable                                                                               &  \centering Not Applicable                                    &  \centering Not Applicable                                                                                         &  \centering Not Applicable                                                   & U.S. high school STEM, college \& career pathways                                                        \\ \hline
Rizzo \& Taylor                                         & Journal of Science Education for Students with Disabilities     & Literature review                     &  \centering Not Applicable                                                                               &  \centering Not Applicable                                    &  \centering Not Applicable                                                                                         &  \centering Not Applicable                                                   & Science achievement, inquiry-based learning                                                              \\ \hline
Rule \&  Stefanich (2012)                            & Journal of STEM Education                                       & Qualitative                           & Pre-service teachers, education professionals                                                &  \centering Not Available                                     &  \centering Not Available                                                                                          &  \centering Not Available                                                    & Professional development, supports to students with physical disabilities in STEM                        \\ \hline
Scruggs \& Mastropieri (2007)                           & Exceptionality                                                  & Essay                                 &  \centering Not Applicable                                                                               &  \centering Not Applicable                                    &  \centering Not Applicable                                                                                         &  \centering Not Applicable                                                   & Science learning and special education, constructed versus instructed learning                           \\ \hline
Scruggs, Mastropieri, \& McDuffie (2007)                & Exceptional Children                                            & Literature review                     & Qualitative research studies                                                                 &  \centering Not Available                                     &  \centering Not Available                                                                                          &  \centering Not Available                                                    & Co-teaching, peer mediation, strategy instruction, mnemonics                                             \\ \hline
Scruggs, Mastropieri, \&  Okolo (2018)               & Focus on Exceptional Children                                   & Essay                                 &  \centering Not Applicable                                                                               &  \centering Not Applicable                                    &  \centering Not Applicable                                                                                         &  \centering Not Applicable                                                   & Science and social studies instruction for students with disabilities                                    \\ \hline
Seifert \&  Espin (2012)                             & Learning Disability Quarterly                                   & Quantitative (quasi-experiment group) & 10th high school students (N=20)                                                             & Learning disability                               &  \centering Not Available                                                                                          & females (\textit{n}=9), males (\textit{n}=11)                                      & Reading science text, combined instructional approaches                                                  \\ \hline
Shoffner, Newsome, Minton, \&  Morris (2015)         & Journal of Career Development                                   & Qualitative (focus groups)            & 9th grade high school students (N=28)                                                        &  \centering Not Available                                     &  \centering Not Available                                                                                          & females (\textit{n}=12), males (\textit{n}=16)                                     & Outcome expectations, STEM knowledge, career goals                                                       \\ \hline
Sublett \&  Plasman (2017)                           & Journal of Career and Technical Education                       & Quantitative (extant data analysis)   & High School Longitudinal Study (started at 9th grade, follow up at 11th grade)               & 21\% with an IEP                                  & 11\% Black, 17\% Hispanic, 9\% Asian, 10\% Other                                                       & 49\% female, 51\% male                                           & Math \& science self-efficacy, CTE, applied STEM coursework                                              \\ \hline
Subramaniam, Ahh, Fleischmann, \&  Druin (2012)      & The Library Quarterly                                           & Position paper                        &  \centering Not Applicable                                                                               &  \centering Not Applicable                                    &  \centering Not Applicable                                                                                         &  \centering Not Applicable                                                   & School library programs \& STEM learning                                                                 \\ \hline
Supalo, Isaacson, \&  Lombardi (2014)                & Journal of Chemical Education                                   & Mixed methods                         & High school students at summer program (N=91)                                                & Blindness, low vision                             & Black (\textit{n}=16), Hispanic (\textit{n}=10), Asian (\textit{n}=7), White (\textit{n}=58)                                               & 41 females (\textit{n}=41), males (\textit{n}=50)                                  & STEM academy, hands on learning                                                                          \\ \hline
Supalo, Wohlers, \& Humphrey (2011)                     & Journal of Science Education for Students with Disabilities     & Qualitative (case study)              & High school students at “Camp Can Do” summer program                                         & Blindness, low vision                             & Youth from Caribbean                                                                                   &  \centering Not Available                                                    & Exploring chemistry, adaptive technologies, laboratory classes                                           \\ \hline
Taylor, Rizzo, Hwang, \& Hill (2020)                    & School Science and Mathematics                                  & Literature review                     &  \centering Not Applicable                                                                               & ASD                                               &  \centering Not Available                                                                                          &  \centering Not Available                                                    & Science achievement, science instruction                                                                 \\ \hline
Therrien, Taylor, Hosp, Kaldenberg, \&  Gorsh (2011) & Learning Disabilities Research and Practice                     & Meta-analysis                         & High school students (N=155)                                                                 & Learning disability                               &  \centering Not Available                                                                                          &  \centering Not Available                                                    & Science achievement, inquiry-based instruction                                                           \\ \hline
Therrien, Taylor, Watt, \& Kaldenberg (2014)            & Remedial and Special Education                                  & Meta-analysis                         & High school students (N=19)                                                                  & Emotional \& Behavior Disorder                    &  \centering Not Available                                                                                          &  \centering Not Available                                                    & Science instruction, knowledge \& retention                                                              \\ \hline
Villanueva \&  Hand (2011)                           & Learning Disabilities Research and Practice                     & Essay                                 &  \centering Not Applicable                                                                               &  \centering Not Available                                     &  \centering Not Available                                                                                          &  \centering Not Available                                                    & Science writing heuristic                                                                                \\ \hline
Wang \& Degol (2017)                                    & Educational Psychology Review                                   & Literature review                     &  \centering Not Applicable                                                                               &  \centering Not Applicable                                    &  \centering Not Applicable                                                                                         &  \centering Not Applicable                                                   & Factors in STEM gender gap, racial differences                                                           \\ \hline
Wei, Lenz, \& Blackorby (2012)                          & Remedial and Special Education                                  & Quantitative (extant data analysis)    & Special Education Longitudinal Study (multiple waves)                                        & 12 IDEA disability categories                     & 19.33\% Black, 13.01\% Hispanic, 65.58\% white                                                         & 32\% female, 68\% male                                           & Math growth trajectories, achievement gaps between groups                                                \\ \hline
Williams, Ernst, \&  Kaui (2015)                     & Journal of STEM Education                                       & Quantitative (extant data analysis)    & Schools \&  Staffing Survey (N=559300); teachers in science, math, \&  tech education) &  \centering Not Available                                     &  \centering Not Available                                                                                          & 62\% female in science; 65\% female in math; 25\% female in tech & School drop-out risk,                                                                                    \\ \hline
Yore \&   Treagust (2006)                         & International Journal of Science Education                      & Essay                                 &  \centering Not Applicable                                                                               &  \centering Not Applicable                                    &  \centering Not Applicable                                                                                         &  \centering Not Applicable                                                   & Science literacy, models of learning, teacher professional development               \hline     
\end{tabular}
\end{table}
\end{@twocolumnfalse}
\end{document}