\documentclass[11.5pt]{sig-alternate}
\usepackage[defaultlines=3,all]{nowidow}
\usepackage{hyperref}
\usepackage{tabularx}
\usepackage{graphicx}
\usepackage{blindtext}
\usepackage[utf8]{inputenc}
\usepackage[english]{babel}
\usepackage{lastpage}
\usepackage{comment}
\usepackage{dirtytalk}
\usepackage{xcolor}
\usepackage{hanging}
\usepackage{wrapfig}
\usepackage[backend=biber, style=apa]{biblatex}
\addbibresource{notation.bib}
\usepackage{authblk}
\usepackage{caption}
\usepackage{graphicx,subfigure}
\usepackage{authblk}
\usepackage{enumitem}
\usepackage[utf8]{inputenc}
\usepackage{cuted}
\usepackage{fancyhdr}
\pagestyle{fancy}
\usepackage{lipsum}
\usepackage{xurl}
\usepackage{ragged2e}
\usepackage{microtype}
\raggedbottom
\renewcommand{\headrulewidth}{0pt}
\renewcommand{\footrulewidth}{0pt}
\setlength\headheight{80.0pt}
\addtolength{\textheight}{-80.0pt}
\chead{%
  \ifcase\value{page}
  % empty test for page = 0
  \or \includegraphics[width=\textwidth]{headerimage.png}% page = 1
 \or \includegraphics[width=\textwidth]{headerimage.png}% page = 2
  \or \includegraphics[width=\textwidth]{headerimage.png}% page = 3
  \or \includegraphics[width=\textwidth]{headerimage.png}% page = 4
 \or \includegraphics[width=\textwidth]{headerimage.png}% page = 5
  \else
  \includegraphics[width=\textwidth]{headerimage.png}
  \fi
}

%\chead{\includegraphics[width=\textwidth]{headerimage.png}}

\hypersetup{
    colorlinks=true,
    urlcolor=blue
}
 
\let\oldabstract\abstract
\let\oldendabstract\endabstract
\makeatletter
\renewenvironment{abstract}
{\renewenvironment{quotation}%
               {\list{}{\addtolength{\leftmargin}{1em} % change this value to add or remove length to the the default
                        \listparindent 1.5em%
                        \itemindent    \listparindent%
                        \rightmargin   \leftmargin%
                        \parsep        \z@ \@plus\p@}%
                \item\relax}%
               {\endlist}%
\oldabstract}
{\oldendabstract}
\makeatother

% Left align captions
\captionsetup{justification   = raggedright,
              singlelinecheck = false}


\begin{document}

\title{Supporting Students with an Autism Spectrum Disorder in Engineering: K-12 and Beyond }

\author[1]{\large \color{blue}Jennifer L. Kouo}
\author[1]{\large \color{blue}Alexis E. Hogan}
\author[1]{\large \color{blue}Sarah Morton}
\author[2]{\large \color{blue}Jay Gregorio}
\affil[1]{Department of Special Education, Towson University}
\affil[2]{The Children's Guild School of Prince George’s County}
\toappear{}
%% ABSTRACT
\maketitle
\begin{@twocolumnfalse} 
\begin{abstract}
\item 
 \textit {Individuals with disabilities, including individuals with an autism spectrum disorder (ASD), are underrepresented in science, technology, engineering, and mathematics (STEM) fields. With the importance of STEM skills in future employment and other disciplines, effective instructional strategies must be identified to enhance early and sustained access to STEM for students with ASD. However, the literature identifying effective STEM-specific supports and practices for this population of students is sparse and regarding engineering, there are no empirical studies that focus on teaching engineering skills to students with ASD. Therefore, the article aims to provide an overview of the available literature on the perspectives of engineering educators and suggested strategies aimed at supporting students with ASD in K-12 instruction and higher education. Additionally, recommendations regarding employment preparation and shifting the workplace environment to support individuals with ASD are presented. The available literature reveals limitations and implications for future research including the presentation of the voices of individuals with ASD across the spectrum. Furthermore, there continues to be work that must be done to prepare educators, employers, peers, and colleagues to better understand the disability and support individuals with ASD in all contexts.}
     \\
     \\
     Keywords: autism spectrum disorder, engineering, instructional strategies
\end{abstract}
\end{@twocolumnfalse}




%% AUTHOR INFORMATION


\textbf{*Corresponding Author,Jennifer L. Kouo  }\\
\href{mailto: Jennifer.Kouo@jhu.edu }{(Jennifer.Kouo@jhu.edu)} \\
\textit{Submitted January 21, 2021 }\\
\textit{Accepted June 23, 2021} \\
\textit{Published online October 21, 2021} \\
\textit{DOI:0.14448/jsesd.13.0011} \\
\pagebreak
\pagebreak

\vspace{5mm}
\section*{\vspace{140mm}}
\begin{large}

\section*{SUPPORTING STUDENTS WITH AN AUTISM SPECTRUM DISCORDER IN ENGINEERING: K-12 AND BEYOND}

The science, technology, engineering, and mathematics (STEM) workforce demand continues to strengthen. According to the U.S. Bureau of Labor Statistics (BLS) 2019–29 employment projections, occupations in the STEM field are expected to increase 8.0 percent by 2029, compared with 3.7 percent for all occupations (Zilberman \& Ice, 2021). There were approximately 2.4 million job vacancies in the STEM field between 2008 to 2018 (Carnevale et al., 2011) and a projected 3.5 million job vacancies by 2025 in the United States according to the National Association of Manufacturing and Deloitte report (Giffi et al., 2015). It is expected that more than 2 million of these vacancies will go unfilled due to a shortage of individuals entering the workforce at a rate required for continued globalization and technological advancement (Carnevale et al., 2011). The U.S. government has identified underrepresented groups such as women, minorities, and persons with disabilities as critical in filling these positions related to science and engineering (Carnevale et al., 2011; National Science Foundation, 2013). Advancing Excellence in P-12 Engineering and The American Society for Engineering Education (2020) states that such efforts to diversify the field will lead to advancements in technology and innovation, and “a more robust and democratic community” (p. 40). 

Ehsan et al. (2018) emphasize that engaging individuals with disabilities in STEM education is necessary for preparing this population for the growing workforce. Specifically, individuals with an autism spectrum disorder (ASD) represent one of the untapped talent pools and may contribute to these societal goals (Wei et al., 2013). The Centers for Disease Control and Prevention (CDC) and Autism and Developmental Disabilities Monitoring (ADDM) Network indicate that 1 in 54 children are diagnosed with ASD in the United States (Maenner et al., 2020). As a spectrum disorder, no one individual with ASD is alike. This heterogeneous population varies in severity in communication, dependence on routines or sameness, restrictive and repetitive behavior, and sensitivity to environmental stimuli. Individuals with ASD may also have particular difficulties with social interactions, verbal or nonverbal communication, and developing and maintaining interpersonal relationships (Schroeder et al., 2014). Furthermore, individuals with ASD may have other comorbidity diagnoses, including speech-language impairment and/or intellectual disability, and other health impairment. 

However, according to Wei et al. (2012), individuals with ASD are more likely than other disability groups to gravitate towards and participate in STEM fields. Evidence suggests that the reason for this may be related to the Empathizing– Systemizing (E-S) theory, which suggests that individuals with ASD may have a greater aptitude toward systemizing, which is conducive for STEM fields. \textit{Systemizing} refers to constructing or analyzing rule-based systems to understand the surrounding world. \textit{Empathizing} relates to social and emotional reactions to the thoughts and feelings of other individuals (Baron-Cohen, 2009). These predilections for predicting and controlling systems and analyzing and/or constructing rule-based systems through the identification of input–function–ou-tput rules may be critical in STEM-related fields (Baron-Cohen, 2002).

Many have called for and supported increased accessibility to and diversity of STEM education to include marginalized and underrepresented people, including those with ASD (Baybee, 2010; National Academy of Engineering and National Research Council, 2009). Such initiatives include the Inventing, Designing, and Engineering for All Students (IDEAS) Maker Program and the NSF-funded Building Capacity for Inclusive Informal STEM Learning Opportunities for People with Autism Spectrum Disorder research project. However, despite efforts to include individuals with ASD and despite their interest in STEM and an aptitude toward systemizing, the focus of STEM educational opportunities has been directed towards neurotypical students (Hwang \& Taylor, 2016). Furthermore, Ehasan et al. (2018) and Moon et al. (2012) report that educators need further preparation to teach STEM-related content and address the unique challenges each student with ASD may face when engaging in STEM activities. These particular challenges may include developing higher-order thinking skills, identifying connections across subject areas, problem-solving, and working within teams work (Basham \& Marino, 2013).	

Beyond K-12 contexts, challenges persist for both students with ASD and instructors in higher education. Only 32\% of individuals with ASD are enrolled in college, which is the third-lowest rate of postsecondary attendance among students with disabilities and the general population. Of those students, an estimated 34.31\% study STEM, with higher concentrations in science or computer science (Wei et al., 2012). Students with ASD face particular challenges following secondary school. Social functioning and self-regulation in unfamiliar living situations increased independence and responsibility for educational success, and inadequate understanding and training of faculty members are among the challenges that impact students with ASD in higher education (Hendricks \& Wehman, 2009; Shmulsky et al., 2019). 

The issues observed in the educational context remain beyond higher education or following transition from high school, where an estimated 80\% of adults with ASD are unemployed or underemployed (NAIR, 2015). According to Booth (2016), there are not only barriers to applying and being hired, but obstacles within the workplace environment related to communication, social interaction, sensory issues, and bullying, harassment, and discrimination. These barriers must be addressed by not only preparing individuals with ASD for the workforce environment but also by adequately training employers and employees about the disability and how to support individuals with ASD in a manner that will benefit all personnel. 

\subsection*{\textbf{\textit{Importance of STEM Access for Individuals with ASD}}}

With the permeation of STEM in every aspect of daily life, it is critical that individuals with ASD be provided with the necessary opportunities to access STEM content (Israel et al., 2013). Although accessibility to STEM instruction and appropriate supports in K-12 education may not lead to the individual entering a STEM major or workforce, STEM skills translate beyond the field. STEM instruction helps students with ASD to solve daily problems and make decisions (National Academy of Engineering and National Research Council, 2009). As a critical feature, STEM education involves the gaining of knowledge and skills and their application in authentic activities and contexts. Practice with solving problems found in the real world translates to using STEM knowledge and skills to solve problems found in both life and work (Holmlund et al., 2018; Kelley \& Knowles, 2016). Additionally, students participate in these activities as part of a group or team, and the social aspect becomes an integral component of the learning process. These communicative interchanges allow for students to learn and collaborate with others as they solve authentic problems in a coordinated manner (Chen et al., 2019). Through STEM, individuals with ASD may become more fulfilled and productive citizens and be successful in the twenty-first century (National Academy of Engineering and National Research Council, 2002; National Academy of Engineering and National Research Council, 2009; Zollman, 2012). Problem-solving and professional skills translate to all aspects of life and in any employment opportunity. STEM knowledge and skills may help an individual with ASD to fix a piece of broken technology or appliance, streamline a process at their workplace, more efficiently complete mundane tasks, and/or collaborate with others. 

\subsection*{\textbf{\textit{Access to and Supports to Engage in Engineering}}}

Due to the importance of STEM skills in future employment and other disciplines, effective instructional strategies must be identified to enhance accessibility to STEM for students with ASD early on in education (Fleury et al., 2014). However, the discussion of access must go further to include considerations of the supports and systemic changes that must occur within the educational system and workforce to fully engage and include this population. The repertoire of studies identifying effective STEM-spe-cific supports and practices for individuals with ASD is sparse and there is a need for research-based interventions to teach STEM (Smith et al., 2013). Furthermore, the scarcity of literature on effective, evidence-based strategies to specifically integrate \textit{engineering} skills for students with ASD marks a particularly critical area for future research. According to a systematic review conducted by Ehsan et al. (2018), there are no empirical research studies that focus on teaching engineering skills to students with ASD. 

Access to engineering-specific instruction and engaging in experiences of scientific inquiry is critical to student learning. It ensures that the underrepresented population has “equity in access, participation, and achievement. Ensuring that all students have the opportunities to develop habits, knowledge, and practices will enable individuals to productively participate in today’s world, make informed decisions about their lives, and be successful in an engineering career if they choose to pursue one” (Advancing Excellence in P-12 Engineering Education \& The American Society for Engineering Education, 2020, p. 41). Inclusion of these diverse learners must therefore involve pathways to engineering instruction and opportunities by exploring their potential and their needs (Ehsan \& Cardella, 2019). Additionally, further steps must be taken to ensure success in both educational and workplace settings through individualized supports and to effectively prepare educators, peers, employees, and colleagues. 

\section*{AIMS}

Given the importance of access to and inclusion of individuals with ASD in engineering and the dearth of empirical research focused on teaching engineering skills to this population (Ehsan et al., 2018), the subsequent sections aim to present the available, but limited, literature on the perspectives of educators and suggested strategies focused on supporting students with ASD in engineering. The strategies that emerge from the literature, which include a conference proceeding, chapters within books, implications from literature reviews, and qualitative studies on experiential knowledge and recommendations of faculty in higher education, span K-12 and higher education, and places of employment. The presentation of the available literature and subsequent sections also underscore future directions for research to continue to address the aforementioned issues concerning access and engagement in the field.

A summary of the available literature on the perspectives of educators and strategies aimed at supporting the engagement of individuals with ASD in engineering are presented following an overview of engineering, the engineering design process, and how engineering may impact individuals with ASD. The presented strategies are relevant to K-12 and higher education and are then followed by strategies that relate specifically to employment and the workplace environment.

\section*{ELEMENTS OF ENGINEERING}

Engineering involves the systematic and iterative approach to designing solutions and often requires skills such as innovation, problem-solving, and critical thinking to solve real-world problems (Lucas \& Hanson, 2016). The engineering design process generally involves the identification and formulation of a problem, designing a solution to the problem, creating and testing the solution, iterating and redesigning the solution to further optimize it, and communicating and disseminating the identified solution to stakeholders. 

By nature, engineering is interdisciplinary and incorporates mathematics, science, and technology, as well as reading, writing, communication, and design and making skills (Rogers \& Portsmore, 2004). Engineering also encompasses habits of mind. Identified by Lucas and Hanson (2016), the six dispositions of engineers include: systems thinking, problem-finding, visualizing, improving, creative problem-solving, and adapting. As mentioned previously, systems thinking involves understanding connections and constructions. Finding problems to make improvements, as well as using effective approaches to problem-solving are critical in engineering. Furthermore, visualization of an abstract idea and communicating solutions are essential. Continual improvements on solutions and products highlight the mindset of engineers to help society through optimization. Creative problem-solving leads to innovative and novel solutions that draw upon different disciplines. Lastly, adapting may involve tinkering to make small adjustments to optimize a solution, or it may involve drawing upon the expertise of others. 

Working within a team is an important element in STEM, and teamwork is an especially critical competency in engineering. The Accreditation Board of Engineering and Technology (ABET) accreditation criteria requires teamwork, which is identified as ‘‘an ability to function effectively on a team whose members together provide leadership, create a collaborative and inclusive environment, establish goals, plan tasks, and meet objectives’’ (p. 5). According to Murzi et al. (2020), effective teams including the following attributes: (1) shared goals and values, (2) commitment to team success, (3) motivation, (4) interpersonal skills, (5) open and effective communication, (6) constructive feedback, (7) ideal team composition, (8) leadership, (9) accountability, (10) interdependence, and (11) commitment to team process and performance. Within industry, effective teams are agile and able to increase productivity (Varvel et al., 2004). Furthermore, team diversity, including life experiences, leads to each member’s ideas being challenged, helps to avoid pitfalls such as groupthink, and helps to generate disparate ideas and innovative solutions (Smith-Doerr et al., 2017).

\section*{ENGINEERING AND COLLATERAL SKILLS}

Engagement in STEM instruction leads to the gaining of skills that reach far beyond the field itself (National Academy of Engineering and National Research Council, 2009). Engineering, specifically, showcases how skills gained by engaging in the engineering design process may translate to success in other facets, including socially and behaviorally. The aforementioned dispositions, such as problem-solving, creativity, adaptation, as well as other professional skills, including social skills, communication, and self-regulation, are essential when engaging in the engineering design process. However, it is important to note that these skills are beneficial in vocations outside of engineering and have application in all aspects of daily life. If provided with opportunities to effectively engage in engineering and with opportunities for generalization, these skills will increase the success of individuals with ASD across their lifetime, including navigating problems and developing interpersonal relationships. However, access to these educational opportunities does not automatically lead to success. Instructional programming, accommodations, and supports including behavioral interventions can narrow the gap between education standards and achievements based on their cognitive functioning. \\These individualized supports are provided by educators, social workers, employers, and other stakeholders, including peers, in both contexts. 

\section*{STRATEGIES FOR TEACHING ENGINEERING TO STUDENTS WITH ASD IN K-12 AND HIGHER EDUCATIONS}

Ehsan and Cardella (2020) state, “To help children get the most out of these experiences, we need to design appropriate engineering activities while considering their needs, assets, and individual potential” (p. 9). The recommendations presented in the available literature on facilitating engineering within K-12 classrooms and engineering courses at the college or university level include academic, social, and behavioral supports. This suggests that engaging in engineering instruction is not limited to addressing its cognitive demands but also the ability to develop social-emotional skills often required in group work settings. These provide a foundation of general guidelines when engineering programs are implemented in different educational contexts such as inclusion classrooms and self-contained classrooms.

\subsection*{\textbf{\textit{General Perspectives from Engineering Educators}}}

The distinguishing traits of an effective engineer in industry and practice must be identified to align instructional programming across all educational settings. Such traits are typically described as \textit{technical skills} since engineers require precision, attention to detail, complex sequence and patterns, and applying theories in real-life situations. The characteristics exhibited by some individuals with ASD may be assets in engineering programs and fields. According to Gobbo et al. (2018), faculty in institutions of higher education identified “attention to detail, the ability to follow complex directions, maintaining proper sequence, recognizing patterns, and using patterns” (p. 13) to be abilities of some students with ASD that are valuable in the field. Attention to detail was specifically identified as an asset, especially with regard to persistence in following procedures and facing problems. Instructors also characterized some students as diligent, with steady work hab-its, and preparedness and interest in engineering. 

However, these characteristics may also hinder the important elements of being an engineer. Identifying novel solutions and changing plans based on information and feedback from stakeholders are important elements of being an engineer. In the engineering design process, for example, students are expected to redesign and improve products based on the results of testing and evaluation. This requires characteristics of open-mindedness, patience, and persistence since engineering tasks could be complex. However, Shmulsky et al. (2019) reported that higher education instructors of students with ASD identified more intense rigidity in thinking, persistence, and attention to detail that could potentially lead to adverse fixations on a problem or nuance and inhibit critical thinking skills. These characteristics may be barriers, especially when there are alterations to materials or sequences. Based on interviews with faculty members, Shmulsky et al. (2019) found that “problems come up when a procedure, assignment, or activity does not ‘go according to script’ and students have to shift the direction of their thinking” (p. 50) and that a new discovery may be missed due to inflexible thinking.

According to Newport and Elms (1997), engineering educators will also need to focus on many areas other than technical competence to produce highly effective engineers. The authors found that effective engineers are perceived to have better interpersonal abilities which include respecting other’s opinions and teamworking skills. Shmulsky et al. (2019) and White et al. (2016) identified emotional regulation, such as expressing frustration and social interactions with peers to be areas of concern for students with ASD. “Faculty noted that an emotional display can be further complicated if the student expressing frustration is not immediately aware of how others perceive him or her, and students on the spectrum tend to have more difficulty in this area” (p. 49). Engineering design courses oftentimes involve teaming, which necessitates social skills and interaction. This may be challenging for students with ASD and may impede participation. Furthermore, inflexibility and rigidity concerning rules and procedures were identified as potential obstacles to working within teams or groups (Gobbo et al., 2018). White et al. (2016) also found that challenges related to self-determination to be a concern expressed by faculty members. Uncertainty about the future, advocating for accommodations, and independent living skills may also further hinder students with ASD in and beyond college coursework.

While the available literature appears to characterize students with more mild features of ASD, it is important to note, that educators must avoid overgeneralization of this disability. As a spectrum disorder, each individual with ASD is inherently unique and that variability around cognition, communication, behavior, and sensory processing must be considered when providing supports. Though not fully represented in the literature, individuals with ASD across the spectrum may be interested in engineering and can be supported to be more included in this field.  

\subsection*{\textbf{\textit{Self-Advocacy and Academic Supports in Classrooms}}}

Engineering education requires executive functioning skills that may be a challenge for students with ASD.  Educators may consider supporting students by helping to segment a large, complex engineering project into smaller, more manageable tasks. Additionally, timers, remind\-ers, and organizational structures, such as color-coding an engineering notebook, may be helpful supports (Van Bergeijk et al., 2014). For example, students may be working to construct a bridge over a stream that frequently floods. Teachers may support students during this long-term project by helping to structure students’ engineering notebooks and provide feedback regarding project management. This may include an engineering portfolio where students organize and reflect on activities that promote not only organization but also metacognition. This is specifically helpful when engineering activities require modeling as a form of visual support.

Inclusive classrooms for engineering or STEM does not assume well-trained or knowledgeable educators in the area of ASD. Therefore, it may be crucial for students with ASD to communicate with faculty members about issues and advocate for specialized needs (Delp, 2017) despite available disability resources in most educational institutions. To address this concern, educators of all levels may be able to review a student’s individualized education plan (IEP), which may serve as a beneficial reference to guide advocacy for supplementary aids and services, academic supports, and accommodations. This comprehensive plan provides a historical background of a student’s range of abilities that may be helpful for a teacher to consider when planning for independent and group tasks. An engineering teacher for example may provide extended time and a schedule to support with time management and organization. It may also be necessary to provide assistive technology to complete some tasks when verbal communication is limited. In some situations, the use of an augmentative and alternative communication (AAC) device provides an opportunity to communicate their thoughts with a team.

\subsection*{\textbf{\textit{Individualization based on Interests}}}

Highly engaging engineering programs provide opportunities for students to explore problems or questions within their areas of interest. Many project-based learning approaches that integrate the engineering design process sustain attention since students have invested time and effort in the subject matter they have chosen. Individuals with ASD demonstrate circumscribed interests or intense focus on certain areas where neurotypical individuals are predicted to have less attentional priority. Faculty have identified that tailored assignments, which align with the interest areas of the student with ASD, may help with further engineering engagement (Gobbo et al., 2018). For example, a student interested in shoe design may be engaged in an engineering design challenge focused on improving shoe treads to better navigate icy sidewalks. Another student may be interested in video games, and enjoy a design challenge focused on improving the capabilities of a handheld controller. Grandin and Duffy (2004) also recommend melding talents an individual with ASD may have with the course material and options, such as allowing the use of hand drawing drafts or using Computer-Aided Design (CAD).

\subsection*{\textbf{\textit{Generalization and Prior Knowledge}}}

Individuals with ASD may struggle with generalizing or transferring skills acquired in structured settings to real-world situations. Furthermore, higher-order thinking skills needed for engineering may be challenging for students as they may struggle to use prior knowledge to tackle problems and identify solutions. Therefore, K-12 educators need to consider and plan for opportunities for students to apply professional skills, as well as engineering design skills, to other facets of their daily lives (Ehsan et al., 2018). In direct alignment with engineering practices, problem identification and problem-solving are also applicable in countless scenarios beyond engineering instruction. Often, these skills are taught and practiced with intervention specialists through simple peer mediation role-playing, informal groups, and other structured small group instruction on conflict resolution. Collaborations with families and community members may provide opportunities for the generalization of these skills in large, natural settings.

\subsection*{\textbf{\textit{Social Skills, Communication, Peer Mentors, and Grouping}}}

Focusing on social skills and communication has been connected to greater success in transitioning from high school to college, as well as overall college success (Hendricks \& Wehman, 2009). As an essential component in engineering disciplines, teaming may lead students with ASD to create new friendships and have additional opportunities to practice social skills as they are emersed in an engineering design challenge. However, simply placing students into groups does not automatically lead to the development of effective teamwork skills (Gallegos \& Peeters, 2011). Training is needed for students to work within teams, as well as for educators to help promote such skills (Murzi et al., 2020). Furthermore, educators must be aware that individuals with ASD are at increased risk of bullying and must be proactive in preventing such occurrences (Schroeder et al., 2014). The nuances needed, as well as age-appropriate interpersonal interactions are necessary as students transition from high school, and again from postsecondary education and into employment (White et al., 2016). Delp (2017) suggests that peer mentors should be identified to support students with ASD in both modeling appropriate social behavior and providing opportunities for social interactions in team projects. For example, during a water filtration challenge, a peer mentor can help model and remind the student with ASD to follow the engineering design process, listen to other classmates’ ideas, suggest their own, and utilize flexible, creative thinking when identifying the solution. Peers can also support one another when testing results are not ideal or when some become frustrated. 

However, it is important to note that peer mentors must be adequately trained. Similar to the evidence-based practice of peer-mediated instruction and intervention (PMII; Steinbrenner et al., 2020), systematic instruction must be delivered to peer mentors to provide support to the diverse needs and abilities of students with ASD during naturally occurring opportunities through modeling, feedback, and various reinforcements. Additionally, social engineering of teams may allow educators to purposefully assign partners and groups that will further provide support to students with ASD (Gobbo et al., 2018; Shmulsky et al., 2019). As mentioned, students with ASD may experience anxiety and frustration when engaged in highly collaborative tasks. One approach is the use of flexible grouping where students can be paired or participate in a smaller or bigger group that is based on multiple categories such as interest, achievement, and skill level. The flexibility of such an approach gives students with ASD more control and allows anticipation over the types of tasks, which may significantly improve engagement. Specifically within a self-contained classroom, educators and other classroom support staff may play the role of peer mentors. Lastly, educators may consider the modification of project expectations to allow them to be completed individually (Gobbo et al., 2018). However, this should be thoughtfully considered due to the importance of teamwork in engineering.

\subsection*{\textbf{\textit{Applied Behavior Analysis}}}

Applied Behavior Analysis (ABA) techniques are core features in many evidence-based practices for students with ASD. In a systematic review of the literature, Ehsan et al. (2018) found that the use of least-to-most prompting, including verbal and video prompting, as well as the use of self-modeling and teacher-modeling, to be effective in teaching science and mathematics skills. Both prompting and modeling may also be critical when supporting students with ASD in engineering. However, Wright et al. (2020) state that the use of ABA, and specifically video modeling to teach engineering skills, requires further research to determine the full impact of these strategies. 

However, the perspective of individuals with autism, advocates, and those engaged in disability studies have raised concerns regarding ABA. Specifically, ABA reinforces the notion that treatment is necessary for those who are ‘different’ and is in direct opposition to efforts to accept neurodiversity – the viewpoint that differences are not abnormal and should not be treated in a manner that requires intervention (Devita-Raeburn, 2016). Autistic author, artist, advocate, and speaker, Max Sparrow (2016) documents the long-term impacts of ABA on individuals and presents the core tenants of ABA and intense regiment as infringing upon the individuality, autonomy, and overall well-being of individuals with ASD. It is important to note that these controversies are mostly tied to discrete trial training (DTT) developed by Ivar Lovaas.

Nonetheless, given the controversy surrounding ABA and potential negative implications, it is critical that individuals acknowledge these differing viewpoints, and consult individuals with ASD and those knowledgeable in supporting such students when incorporating such practices in engineering. Continual reflection on the appropriateness of these and all interventions is necessary and avoids a “one size fits all” approach.

\subsection*{\textbf{\textit{Using Universal Design for Learning as a Framework in Engineering Instruction}}}

The available literature references Universal Design for Learning (UDL) as a framework that should be applied when supporting individuals with ASD in engineering. Developed by the Center for Applied Special Technology (CAST, 2018), the UDL framework, which consists of three core principles, aims to support educators at the onset of instructional planning to minimize barriers and maximize accessibility to accommodate learner differences. In doing so, teachers provide both flexibility and choice. Recognized as benefiting all students, it is important to particularly emphasize UDL in this context as it may not be prevalent in all classrooms in K-12 and higher education, especially as it relates to engineering instruction.

Application of the UDL framework may help educators plan opportunities for integrating engineering in an engaging and meaningful manner, and address the variable strengths, needs, and interests of each student with ASD. However, it is important to note that while there is literature investigating the impact of UDL in postsecondary STEM education for students with disabilities (Schreffler et al., 2019), there is limited literature that specifically examines the impact of UDL in engineering instruction.  
The first principle of UDL is multiple means of engagement and focuses on the “why” of learning or the motivation to learn. The principle focuses on the classroom environment, student interest and persistence, and self-regulation. The following are considerations for incorporating this UDL principle for all students, including individuals with ASD: 

\begin{itemize}
    \item 	Limit sensory stimulation and distractions within the classroom environment, including noise and visual stimulation.
    \item 	Vary the length of work sessions and availability of breaks.
    \item 	Provide support in whole and small group discussions and work.
    \item 	Provide models, scaffolds, and feedback regarding coping skills.
\end{itemize}

A teacher can try to dim lights within the classroom or laboratory or be aware of the visual stimulation of videos or images. Creating a safe casing for an egg drop challenge may lead to frustrations. Reminders about appropriate coping mechanisms and expected behaviors may be beneficial for all students. Allowing breaks, such as a short walk in the hallway, and returning to the project may also support students. 

Clarity of instructions and expectations for activities, assignments, and behaviors may further support all students, including those with ASD. Gobbo et al. (2018) reported that “Faculty found that concrete questions and assignments worked better than vague, open-ended prompts. Faculty emphasized the importance of providing direct instruction about how to participate in group work and discussions” (p. 14). Predictability through structured courses and lectures may also help students with ASD know what to expect and help to curtail anxiety. However, it is important to note that engineering coursework and challenges may purposefully be open-ended and fluid to allow for creative thinking and the identification of a range of solutions, as well as provide an authentic engineering design experience (Bartholomew \& Strimel, 2018). In their qualitative single-case study analysis observing the engineering experiences of a nine-year-old child with mild ASD engaging in problem scoping alongside his mother, Ehasan and Cardella (2020) also concluded that children with ASD, with or without adult support, need to be exposed to both well-structured and ill-structured or open-ended design problems. Therefore, scaffolded supports should be considered when providing specific instructions and more close-ended prompts, with a maintained goal of helping students to gradually become more successful in engineering design solutions to open-ended, shifting problem statements. 

The second principle, multiple means of representation, focuses on the “what” of learning. This principle highlights the ways in which educators may support learners to perceive and comprehend presented information by considering the appropriateness of content to enable processing, and other supports and prompts. Gobbo et al. (2018) found that faculty members applied this principle by using varied presentation formats, such as notes posted online, videos, webpages, and readings. Moon et al. (2012) emphasized the importance of flexible course materials that may be accessed digitally, which may allow for accessibility features such as scalable fonts and captions for images and videos. The following are additional considerations for incorporating this UDL principle for all students, including individuals with ASD:

\begin{itemize}
    \item Provide multiple pathways to understanding vocabulary, symbols, and new information, including alternative text descriptions and visual supports, and embedded support through hyperlinks.
    \item Emphasize key ideas and connections with the use of outlines, graphic organizers, and concept maps.
    \item Provide opportunities to revisit key ideas and how they are connected.
\end{itemize}

An engineering design challenge focused on building a tower stable enough to withstand an earthquake may require teachers to help students understand key terms related to the magnitude and intensity of earthquakes. As students are testing their structures, these key terms and ideas can be revisited to support further understanding. 

The third principle of UDL focuses on the “how” of learning. By providing multiple means of action and expression, educators support students with navigating the learning environment through accommodations and accessibility tools, as well as with demonstrating their learning in a number of ways. Students with ASD may have limited fine-motor skills and may struggle with parts and system models using tangible objects. Therefore, considerations must be made regarding assistive technologies. The principle may also be applied through options for assessments, such as tests, take-home assessments, and online and in-class discussions. Moon et al. (2012) also recognized the importance of UDL in providing equitable and accessible tests and evaluations. The following are additional considerations for incorporating this UDL principle for all students, including individuals with ASD:

\begin{itemize}
    \item Provide alternatives to rate, timing, speed, and range of motor action when interacting with materials and technologies.
    \item Support the use of assistive technologies, including text-to-speech and AAC devices.
    \item Support planning and strategy development by including prompts to show work and to stop and think, and guides to chunk larger tasks and goals into manageable steps.
    \item Provide alternative ways of demonstrating learning, which may include but are not limited to integration of technology, using art for modeling, creative writing, interviews, and other guided activities.
\end{itemize}

The suggested strategies could be applied to a design challenge that involves creating a tool that will allow a single person to bring in as many grocery bags into the house as possible. Students, including those with ASD, may require support when using 3D printing software, printers, and other technologies for prototyping. As they work in their team, AAC devices may help them fully communicate their ideas and collaborate with others. A teacher can provide additional support by guiding the class through the engineering design process and probing students as they design and test their solutions.  

\subsection*{\textbf{\textit{Universal Design and Engineering Curricula}}}

Related to the application of UDL in instruction, is the integration of universal design (UD) in engineering curricula. According to Blaser et al. (2015) UD in engineering coursework: 

\begin{quotation}
\noindent
    Provides a potential framework for integrating disability, accessibility, and usability topics across the engineering curriculum. Universally designed products are designed to be usable by the largest audience possible. Teaching future engineers to apply UD principles in product design challenges them to consider how their current and future endeavors can help others and thereby improve the world around them, which will ultimately result in future products and environments that are more broadly accessible. (p. 26.935.2)
\end{quotation}

UD in engineering helps to welcome a broader audience, including individuals with disabilities and underrepresented minorities, by encouraging all students to engineer solutions that are accessible and responsive to the diverse needs of society. For example, students may identify solutions to help individuals with limited fine motor skills to use a standard keyboard or utilize touch screens at stores or on personal tablets. Another design challenge may involve designing discreet clothing that is responsive to the sensory needs of an individual with ASD.  These opportunities to communicate with and design solutions for diverse stakeholders may allow all students to gain different perspectives and lead engineering students to more successfully work within diverse engineering teams that may include individuals with ASD.

Despite these benefits, UD is not oftentimes present in engineering curricula. Blaser et al. (2015) reported that students within institutions of higher education did not learn about disability-related content or accessibility in any of their engineering or other STEM courses. The authors do identify an emerging effort to integrate UD in first-year and capstone engineering design courses, which involve eliminating accessibility barriers in technology, assistive technology, toys, athletic equipment, and laboratory spaces. Educators in K-12 and higher education are encouraged to consider the broad application of both UDL and UD in engineering instructional practices and curricula to increase awareness and acceptance of the disability community and accessibility for individuals with disabilities, including ASD, in a multitude of ways.

\section*{STRATEGIES AND SUPPORTS FOR STUDENTS WITH ASD IN EMPLOYMENT}

As a critical mandate of both K-12 and higher education, and in conjunction with data showing a larger percentage of unemployment and underemployment among individuals with ASD, preparing students to be successful in future employment is critical (Gobbo et al., 2018). As mentioned previously, professional skills, such as social interaction, effective communication, working collaboratively within a team, conflict resolution, and self-regulation are essential throughout schooling and beyond. 

As part of IEP transition planning, individuals with ASD should be involved in the process of becoming independent young adults. At this stage, students should be encouraged to partake in community service hours, internship opportunities, and paid work opportunities at relevant organizations or companies. These opportunities can continue at the college or university level, as well as through research assistant positions with professors. These opportunities and the support of and feedback from mentors and others in the field of engineering may help an individual be more successful in navigating future employment (Van Bergeijk et al., 2014). 

However, placing the responsibility on individuals with ASD to assimilate to the work environment is inequitable. Engineering faculty have stated that supports may be necessary for places of employment to help individuals with ASD with the social demands and hidden curriculum of a work environment and increased responsibilities and expectations from employers (Gobbo et al., 2018). Booth (2016) identified a number of barriers and best practices for employers: 

\begin{itemize}
    \item An induction or orientation to provide introductions to key personnel, routines, and processes.
    \item Consider the communication preferences and sensory environment and make adjustments to accommodate multiple pathways for communicating and needs regarding sensory overstimulation.
    \item Train managers and employers about ASD, neurodiversity, and how to provide supports regarding communication and sensory stimulation.
    \item Work to reduce bullying, harassment, and discrimination within the workplace.
\end{itemize}

Foster opportunities to socially interact and provide training for all employees, which challenges the inaccurate belief that only individuals with ASD require social skills training. 

The above supports can help an employee with ASD to be successful, fulfilled, and accepted in the workplace. As documented in “A Visible Career on the Spectrum” (2020) a proactive, mindful manager can help change the trajectory and impact the success of an employee with ASD. The anonymous author documents how her manager has supported office moves through advanced preparation and conversations about the transition and helped her to prepare and rehearse for important meetings. Additional supports provided included specific instructions and reminders for projects. The actions of the manager permeated the entire workplace environment and helped to increase understanding and ultimately empowered the author to advocate for herself. “After all this time of being misunderstood, I have witnessed first-hand how some openness, acceptance, patience, and support could allow me to be genuinely happy at work. As a result, my productivity and creativity are now at their highest” (p. 22).

Though these are recommendations for the workplace environment, lessons can be learned and applied to the K-12 education setting and institutions of higher education. More needs to be done to fully apply inclusive practices that allow for \textit{all} individuals, whether it be in the classroom or in the workplace to thrive. 

\section*{DISCUSSION}

Individuals with disabilities, including persons with ASD, are underrepresented in STEM disciplines, and concerning engineering, educators and employers face unique challenges and rewards to providing access and engaging this population through appropriate and individually responsive supports. The above sections present the available literature on the perspectives of educators and suggested strategies aimed at supporting students with ASD in engineering. Spanning across K-12 education (Ehsan \& Cardella, 2020; Van Bergeijk et al., 2014), higher education (Delp, 2017; Gobbo et al., 2018; Shmulsky et al., 2019), and workplaces (Booth, 2016), the findings emphasize that all individuals have a role in supporting persons with ASD in engineering education and employment. Changes must occur to develop a classroom and workplace culture that accepts and embraces neurodiversity. In order for individuals with ASD to be fully valued members of any engineering team and context, educators, employees, students, and colleagues must be trained and prepared to support this unique population. 

\subsection*{\textbf{\textit{Limitations and Implications for Future Research}}}

There remains no empirical literature examining the effectiveness of interventions to specifically integrate engineering skills for individuals with ASD (Ehasn et al., 2018). The summary of the included literature present recommendations based on the available and limited, research literature and include perspectives and suggestions from educators based upon observation and experience. However, the limited literature included also underscores future directions for research. 

As indicated in a number of the included articles (Delp, 2017; Gobbo et al., 2018; Shmulsky et al., 2019) there is emerging literature presenting suggested strategies for faculty in higher education to better support this population. However, integration of engineering can and should begin early, as Ehsan and Cardella (2019) have stated that children with ASD “need to be exposed to engineering design activities and need to have opportunities to practice their design skills and competencies” (p. 14). While it is possible to glean and generalize these recommendations to the K-12 setting, more literature is needed within this specific context, as well as added literature outlining research-supported strategies applicable to higher education and employment. 

It is also important that future publications include empirical studies and present \textit{research-based} interventions and implications for practitioners and others in the field of engineering. The summarized strategies presented in this article emerge from a conference proceeding (Delp, 2017), books (Booth, 2016; Moon et al., 2020; Van Bergeijk et al., 2014), and implications from literature reviews (Ehsan et al., 2018; Hendricks \& Wehman, 2009; Wright et al., 2020). The available literature, specifically in terms of the chapters referenced, provide a greater focus on science, technology, and mathematics, with limited suggestions regarding engineering. There are also few studies presented, including conclusions and implications from Ehsan and Cardella (2020), which was published after the systematic review conducted by Ehsan et al., (2018) and qualitative studies on experiential knowledge of and suggestions from engineering faculty in higher education (Gobbo et al., 2018; Shmulsky et al., 2019). 

Furthermore, there needs to be research that represents the entirety of the autism spectrum. The available literature (“A Visible Career on the Spectrum,” 2020; Van Bergeijk et al., 2014) assumes or presents perspectives and strategies for individuals requiring less intense supports or are considered \textit{higher functioning}, and may only need support with navigating social aspects of engineering, higher education, and employment. Furthermore, the voices of individuals with ASD must be at the forefront. The message from the autism advocacy community is clear: “Nothing about us without us.” Presenting the experiences and perspectives of persons with ASD will help increase acceptance and the understanding of neurodiversity. 

There is also much work that needs to be done to prepare educators, faculty members in higher education, employers, and peers and colleagues to better understand the disability and support individuals with ASD to be more successful in all contexts. While this article focuses largely on the skills and strategies in which educators and employers may implement to help individuals with ASD to be successful in engineering, this is only a component of a greater initiative to increase accessibility to and diversity in this field.

More specific supports need to be identified to not only address the spectrum of the disability, but also the many disciplines within engineering, including aerospace, electric, and mechanical engineering. With additional research, the inclusion and preparation of students with ASD in the field of engineering and other vocational opportunities that require engineering skills and dispositions may be improved. 

In conclusion, the skills and strategies identified in the article provide implications for educators and may provide an initial place to begin integrating engineering practices within the K-12 classroom. With these initial strategies, further considerations may be made concerning greater, individualized supports for students with ASD to successfully engage in engineering. 

\end{large}
\include{} 
\section*{\textbf{REFERENCES}}\par 

\leftskip 0.25in
\parindent -0.25in 
Accreditation Board for Engineering and Technology (ABET). (2018). Criteria for Accrediting Engineering Programs 2019–2020. ABET. \url{https://www.abet.org/accreditation/accreditation-criteria/criteria-for-accrediting-engineer-ing-programs-2019-2020/}

Advancing Excellence in P-12 Engineering Education \& The American Society for Engineering Education (2020). \textit{Framework for P-12 engineering learning: A defined and cohesive educational foundation for P-12 engineering.} Washington, DC: American Society for Engineering Education.

A visible career on the spectrum: An engineer with autism explains how she has succeeded in the workplace, and what employers can do to be more supportive. (2020). \textit{TCE: The Chemical Engineer, 945,} 20–23.

Baron-Cohen, S. (2002). The extreme male brain theory of autism. \textit{Trends in Cognitive Sciences, 6}(6), 248–254.

Baron-Cohen, S. (2009). Autism: The empathizing-systemizing (E-S) theory. \textit{Annals of the New York Academy of Science, 1156,} 68–80. 

Bartholomew, S. R., \& Strimel, G. J. (2018). Factors influencing student success on open-ended design problems. \textit{International Journal of Technology \& Design Education, 28}(3), 753–770. 

Basham, J. D., \& Marino, M. T. (2013). Understanding STEM education and supporting students through universal design for learning. \textit{Teaching Exceptional Children, 45}(4), 8-15.

Baybee, R. W. (2010). What is STEM education? \textit{Science,} 329, 996–996.

Blaser, B., Steele, K. M., \& Burgstahler, S. E. (2015). Including universal design in engineering courses to attract diverse students. \textit{Proceedings of the ASEE Annual Conference} \& Exposition.

Booth, J. (2016). \textit{Autism equality in the workplace: Removing barriers and challenging discrimination.} Jessica Kingsley Publishers.

Carnevale, A. P., Smith, N., \& Melton, M. (2011). \textit{STEM.} Washington, DC: Georgetown University Center on Education and the Workforce. 

Center for Applied Special Technology (CAST). (2018). Universal design for learning guidelines version 2.2. Retrieved from \url{http://udlguidelines.cast.org}. 

Chen, L., Yoshimatsu, N., Goda, Y., Okubo, F., Taniguchi, Y., Oi, M., Konomi, S., Shimada, A., Ogata, H., \& Yamada, M. (2019). Direction of collaborative problem solving-based STEM learning by learning analytics approach. \textit{Research \& Practice in Technology Enhanced Learning, 14}(1), 1–28. 

Delp, D. (2017). Where resources end and teaching begins: Experience with students with autism spectrum disorders in the freshman engineering curriculum. \textit{Proceedings of the ASEE Annual Conference \& Exposition.}

Devita- Raeburn, E. (2016, August 10). \textit{The controversy over autism’s most common therapy.} Spectrum. \url{https://www.spectrumnews.org/features/deep-dive/controv-ersy-autisms-common-therapy/}

Ehsan, H., Rispoli, M., Lory, C., \& Gregori, E. (2018). A systematic review of STEM instruction with students with autism spectrum disorders. \textit{Review Journal of Autism and Developmental Disorders,} 5(4), 327. 

Ehsan, H., \& Cardella, M. E. (2019). Investigating Children with Autism’s Engagement in Engineering Practices: Problem Scoping (Fundamental). \textit{Proceedings of the ASEE Annual Conference \& Exposition,} 15027–15043. 

Ehsan, H., \& Cardella, M. E. (2020). Capturing children with autism’s engagement in engineering practices: A focus on problem scoping. \textit{Journal of Pre-College Engineering Education Research (J-PEER), 10(1).} 

Fleury, V. P., Hedges, S., Hume, K., Browder, D. M., Thompson, J. L., Fallin, K., et al. (2014). Addressing the academic needs of adolescents with autism spectrum disorder in secondary education. \textit{Remedial and Special Education, 35,} 68–79. 

Gallegos, P. J., \& Peeters, J. M. (2011). A measure of teamwork perceptions for team-based learning. \textit{Currents in Pharmacy Teaching and Learning, 3}(1), 30–35.

Gobbo, K., Shmulsky, S., \& Bower, M. (2018). Strategies for teaching STEM subjects to college students with autism spectrum disorder. \textit{Journal of College Science Teaching, 47}(6), 12–17.

Giffi, C., Dollar, B., Gangula, B. \& Drew Rodriguez, M. (2015). Help wanted American manufacturing competitiveness and the looming skills gap. \textit{Deloitte Review. 16,} 96-113

Grandin, T., \& Duffy, K. (2004). \textit{Developing talents: Careers for individuals with Asperger syndrome and high functioning Autism.} Shawnee Mission, KS: Autism Asperger Publishing Company.

Hendricks, D. R., \& Wehman, P. (2009). Transition from school to adulthood for youth with autism spectrum disorders. \textit{Focus on Autism and Other Developmental Disabilities, 24,} 77–88. 

Holmlund, T. D., Lesseig, K., \& Slavit, D. (2018). Making sense of “STEM education” in K-12 contexts. \textit{International Journal of STEM Education, 5}(1), 32.

Hwang, J., \& Taylor, J. C. (2016). Stemming on STEM: A STEM education framework for students with disabilities. \textit{Journal of Science Education for Students with Disabilities,} 19, 4. 

Israel, M., Maynard, K., \& Williamson, P. (2013). Promoting literacy-embedded, authentic STEM instruction for students with disabilities and other struggling learners. \textit{Teaching Exceptional Children,} 45, 18–25. 

Kelley, T. R., \& Knowles, J. G. (2016). A conceptual framework for integrated STEM education. \textit{International Journal of STEM Education, 3}(1), 11.

Lucas, B., \& Hanson, J. (2016). Thinking like an engineer: Using engineering habits of mind and signature pedagogies to redesign engineering education. \textit{International Journal of Engineering Pedagogy, 6}(2), 4–13. 

Maenner M. J., Shaw K. A., Baio J., et al. Prevalence of Autism Spectrum Disorder Among Children Aged 8 Years — Autism and Developmental Disabilities Monitoring Network, 11 Sites, United States, 2016. MMWR Surveill Summ 2020; 69(No. SS-4):1–12.

Moon, N. W., Todd, R. L., Morton, D. L., \& Ivey, E. (2012). \textit{Accommodating students with disabilities in science, technology, engineering, and mathematics (STEM).} Atlanta, GA: Center for Assistive Technology and Environmental Access, Georgia Institute of Technology. 

Murzi, H. G., Chowdhury, T. M., Karlovšek, J., \& Ruiz Ulloa, B. C. (2020). \textit{Working in large teams: Measuring the impact of a teamwork model to facilitate teamwork development in engineering students working in a real project. International Journal of Engineering Education, 36}(1 B), 274–295.

Newport, C. L. ((Cheryl L. 1969. (1995). \textit{Effective engineers / by C. Leigh Newport; supervised by D.G. Elms.} Dept. of Civil Engineering, University of Canterbury.

National Academy of Engineering and National Research Council. (2002). \textit{Technically speaking: Why all Americans need to know more about technology.} Washington, DC: The National Academies Press. 

National Academy of Engineering and National Research Council. (2009). \textit{Engineering in K-12 education: Understanding the status and improving the prospects.} Washington, DC: The National Academies Press.

National Autism Indicators Report (2015). \textit{Transition into Young Adulthood. Philadelphia, PA: Life Course Outcomes Research Program.} A.J. Drexel Autism Institute, Drexel University.

National Science Foundation. (2013). \textit{Women, minorities, and persons with disabilities in science and engineering: 2013.} Washington, DC: The National Academies Press.

Rogers, C., \& Portsmore, M. (2004). \textit{Bringing Engineering to Elementary School. Journal of STEM Education: Innovations \& Research, 5}(3/4), 17–28.

Schroeder, J. H., Cappadocia, M. C., Bebko, J. M., Pepler, D. J., \& Weiss, J. A. (2014). Shedding light on a pervasive problem: A review of research on bullying experiences among children with autism spectrum disorders. \textit{Journal of Autism and Developmental Disorders, 44}(7), 1520–1534. 

Shmulsky, S., Gobbo, K., \& Bower, M. W. (2019). STEM faculty experience teaching students with autism. \textit{Journal of STEM Teacher Education: 53(9).}

Smith, B. R., Spooner, F., \& Wood, C. L. (2013). Using embedded computer-assisted explicit instruction to teach science to students with autism spectrum disorder. \textit{Research in Autism Spectrum Disorders, 7,} 433–443. 

Smith-Doerr, L., Alegria S. N., \& Sacco, T. (2017). How diversity matters in the US science and engineering workforce: A critical review considering integration in teams, fields, and organizational contexts. \textit{Engaging Science, Technology, and Society, 3,} 139–153. 

Sparrow, M. (2016, October 20). ABA. Unstrange Mind. \url{http://unstrangemind.com/aba/}

Steinbrenner, J. R., Hume, K., Odom, S. L., Morin, K. L., Nowell, S. W., Tomaszewski, B., Szendrey, S., McIntyre, N. S., Yücesoy-Özkan, S., \& Savage, M. N. (2020). Evidence-based practices for children, youth, and young adults with Autism. The University of North Carolina at Chapel Hill, Frank Porter Graham Child Development Institute, National Clearinghouse on Autism Evidence and Practice Review Team.

Van Bergeijk, E., Ranaldo, M., \& Shtayermman, O. (2014). Teaching STEM to students with autism spectrum disorders. In Green, S. L. (Ed.), \textit{S.T.E.M. education: strategies for teaching learners with special needs} (pp. 81-99). Nova Science Publisher’s, Inc.

Varvel, T., Adams, S. G., Pridie, S. J., \& Ruiz Ulloa, B. C. (2004). Team Effectiveness and Individual Myers-Briggs Personality Dimensions. \textit{Journal of Management in Engineering, 20}(4), 141–146.

Wei, X., Yu, J. W., Shattuck, P., McCracken, M., \& Blackorby, J. (2012). Science, Technology, Engineering, and Mathematics (STEM) participation among college students with an Autism Spectrum Disorder. \textit{Journal of Autism and Developmental Disorders, 43,} 1539–1546. 

Wei, X., Christiano, E., Yu, J., Blackorby, J., Shattuck, P., \& Newman, L. (2013). Postsecondary pathways and persistence for STEM versus non-STEM majors among college students with an Autism Spectrum Disorder. \textit{Journal of Autism and Developmental Disorders, 44,} 1159–1167. 

White, S. W., Elias, R., Salinas, C. E., Capriola, N., Conner, C. M., Asselin, S. B., Miyazaki, Y., Mazefsky, C. A., Howlin, P., \& Getzel, E. E. (2016). Students with autism spectrum disorder in college: Results from a preliminary mixed methods needs analysis. \textit{Research in Developmental Disabilities, 56,} 29–40.

Wright, J. C., Knight, V. F., \& Barton, E. E. (2020). A review of video modeling to teach STEM to students with autism and intellectual disability. \textit{Research in Autism Spectrum Disorders, 70.}

Zimmman, A. \& Ice, L. (2021, January 19). \textit{Why computer occupations are behind strong STEM employment growth in the 2019–29 decade.} U.S. Bureau of Labor and Statisics. \url{https://www.bls.gov/opub/btn/vol-ume-10/why-computer-occupations-are-behind-strong-stem-employment-growth.htm}

Zollman, A. (2012). Learning for STEM literacy: STEM literacy for learning. \textit{School Science and Mathematics, 112}(1), 12–19. 

\end{document}
