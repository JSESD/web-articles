\documentclass[11.5pt]{sig-alternate} % sets document style to sig-alternate
% packages
% typesetting
%\usepackage{dirtytalk} % typset quotations easier (\say{stuff})
\usepackage{hanging} % hanging paragraphs
\usepackage[defaultlines=3,all]{nowidow} % avoid widows
\usepackage[pdfpagelabels=false]{hyperref} % produce hypertext links, includes backref and nameref
\usepackage{xurl} % defines url linebreaks, loads url package
\usepackage{microtype}
%\usepackage[super]{nth} % easily create superscript ordinal numbers with \nth{x}
\usepackage{textcomp}
\newcommand{\texttildemid}{\raisebox{0.4ex}{\texttildelow}}
% layout
%\usepackage{enumitem} % control layout of itemize, enumerate, description
\usepackage{fancyhdr} % control page headers and footers
\usepackage{float} % improved interface for floating objects
%\usepackage{multicol} % intermix single and multiple column pages
% language
\usepackage[utf8]{inputenc} % accept different input encodings
\usepackage[english]{babel} % multilanguage support
% misc
\usepackage{graphicx} % builds upon graphics package, \includegraphics
%\usepackage{lastpage} % reference number of pages
%\usepackage{comment} % exclude portions of text (?)
\usepackage{xcolor} % color extensions
\usepackage[backend=biber, style=apa]{biblatex} % sophisticated bibliographies % necessary for HTML to display author info and date on abstract page
\usepackage{csquotes} % advanced quotations, makes biblatex happy
\usepackage{authblk} % support for footnote style author/affiliation
% tables and figures
\usepackage{tabularray}
%\usepackage{array} % extend array and tabular environments
\usepackage{caption} % customize captions in figures and tables (rotating captions, sideways captions, etc)
%\usepackage{cuted} % allow mixing of \onecolumn and \twocolumn on same page
\usepackage{multirow} % create tabular cells spanning multiple rows
%\usepackage{subfigure} % deprecated, support for manipulation of small figures
%\usepackage{tabularx} % extension of tabular with column designator "x", creates paragraph-like column whose width automatically expands
%\usepackage{wrapfig} % allows figures or tables to have text wrapped around them
%\usepackage{booktabs} % better rules
% dummy text
%\usepackage{blindtext} % blind text dummy text
%\usepackage{kantlipsum} % Kant style dummy text
\usepackage{lipsum} %lorem ipsum dummy text
% other helpful packages may be booktabs, longtable, longtabu, microtype

\pagestyle{fancy} % sets pagestyle to fancy for fancy headers and footers

% header and footer
% modern way to set header image
\renewcommand{\headrulewidth}{0pt} % defines thickness of line under header
\renewcommand{\footrulewidth}{0pt} % defines thickness of line above header
\setlength\headheight{80.0pt} % sets height between top margin and header image, effectively moves page contents down
\addtolength{\textheight}{-80.0pt} % seems to affect the lower height. maybe only works properly if footer numbers enabled?
\fancyhf{}
\fancyhead[CE, CO]{\includegraphics[width=\textwidth]{headerImage.png}}
% footer
%\fancyfoot[LE,LO]{Article Title Here \\ DOI: }% left footer article title and doi
%\fancyfoot[CE,CO]{{}} % center footer empty
%\fancyfoot[RE,RO]{\thepage} % right footer page numbers
%\pagenumbering{arabic} % arabic (1, 2, 3) numbering in footer

\hypersetup{colorlinks=true,urlcolor=blue} % sets link color to blue
\urlstyle{same} % sets url typeface to same as rest of text

% set caption and figure to italics, label bold, left align captions, does not transfer to HTML
\captionsetup{labelfont=bf, font={large, it}, justification=raggedright, singlelinecheck=false}
\renewcommand\theContinuedFloat{\alph{ContinuedFloat}}

%this next bit is confusing, but essentially changes the width of the abstract. Seems to have been copied from this https://tex.stackexchange.com/questions/151583/how-to-adjust-the-width-of-abstract
\let\oldabstract\abstract
\let\oldendabstract\endabstract
\makeatletter %changes @ catcode to enable modification (in parsep)
\renewenvironment{abstract} %alters the abstract environment
{\renewenvironment{quotation}%
               {\list{}{\addtolength{\leftmargin}{1em} % change this value to add or remove length to the the default ?
                        \listparindent 1.5em%
                        \itemindent    \listparindent%
                        \rightmargin   \leftmargin%
                        \parsep        \z@ \@plus\p@}%
                \item\relax}%
               {\endlist}%
\oldabstract}
{\oldendabstract}
\makeatother %changes @ catcode to disable modification

% checks
% italics
% links -
% dashes -
% tildes -
% dollars -
\begin{document}

\title{Science Instruction for Secondary Students with Emotional or Behavioral Disorders: A Guide for Curriculum Development}

\author[1]{\large \color{blue}Tal Slemrod}
\author[1]{\large \color{blue}Shelley Hart}
\author[1]{\large \color{blue}Leah Wood}
\author[1]{\large \color{blue}William Coleman}

\affil[1]{California State University, Chico}
\affil[2]{California Polytechnic State University, San Luis Obispo}
\affil[3]{Renton School District}

\toappear{}
%% ABSTRACT
\maketitle
\begin{@twocolumnfalse} 
\begin{abstract}
\item 
\textit {This article provides a step-by-step guide for the organization and development of science lessons and units, to support the academic and behavioral performance of secondary students with challenges with related disabilities. This clinical practice guide provides a process for curriculum development for students with emotional or behavior disorders (EBD) in the science classroom. Steps include recommendations, goals, and examples for administrators and educators to discover appropriate plans and interventions to promote engagement and learning, including supporting success on State mandated High Stakes Assessments}
\\ \\
Keywords: Emotional Behavioral Disorder, Science, Special Education Curriculum, Curriculum Design, Instruction Design
\end{abstract}
\end{@twocolumnfalse}

%% AUTHOR INFORMATION

\textbf{*Corresponding Author, Tal Slemrod}\\
\href{mailto: tslemrod@csuchico.edu}{(tslemrod@csuchico.edu)} \\
\textit{Submitted  Jul 11 2018}\\
\textit{Accepted Oct 12 2018 } \\
\textit{Published online November 26th, 2018} \\
\textit{DOI: 10.14448/jsesd.10.0006} \\
\pagebreak
\clearpage
\begin{large}
\section*{INTRODUCTION}
\textit{Mr. Kelty’s Biology classroom is a friendly, informal space. On the whiteboard are daily announcements and schedules. Above the white board are two pull down maps and a white screen for use with a projector (on a cart). There are two bookshelves on the left wall with a few biology textbooks, which are rarely used as they frustrate the students because they have difficulty engaging in or understanding the content.}

\textit{The class period generally follows the same daily routine. Class is held from 9:40 to 10:30 each morning. As students enter, Mr. Kelty welcomes them and provides them with their materials for a daily science activity. There is an obvious and strong rapport between Mr. Kelty and the students, but he struggles to keep them engaged in science instruction. He has a background in science but does not have curricular materials to use. He approaches his principal and exclaims, “I don’t know what to do. Science is hard for my students and I don’t know how to design instruction that supports their needs. I need help!”}
 
Students with challenging behaviors often struggle in academic and social settings. Specifically, students with Emotional or Behavioral Disorders (EBD) often display high levels of aggression (verbal and physical) and have difficulty socially and academically. Consequently, students with EBD often have lower academic grades, increased disciplinary problems, and increased dropout rates (Wagner, Kutash, Duchnowski, Epstein, \& Sumi, 2005). In general, students with EBD who have been expelled from high school are reported to be less successful as adults (Wynne, Ausikaitis, \& Satchwell, 2013). For example, in regards to postschool outcomes, students with EBD are less likely to find employment, live independently, or attend a 4-year college than their peers without disabilities, and they are more likely to report criminal histories (Wagner \& Newmann, 2012). While these statistics have improved over the last decade (Wagner \& Newmann, 2012; McFarland et al., 2018), 35\% of students with EBD drop out before completing high school, the highest rate among students from any disability category (McFarland et al., 2018). With 5\% of students served under IDEA receiving support under the disability category of EBD, more needs to done to ensure their success in secondary settings (McFarland et al., 2018).

The difficulties that many students with EBD encounter in a school environment are compounded by state mandated high stakes assessments. Secondary science has gained national attention as both states and the federal government work to make science proficiency part of the high school graduation requirement. Nationally, the agenda to improve students’ science aptitude has been the focus of major initiatives (e.g., the U.S. Department of Education’s \textit{Race to the Top} program). The development of the Next Generation Science Standards (NGSS; NGSS Lead States, 2013) represents a national effort to provide teachers with a set of standards for teaching both scientific and engineering content and practices. About 40 states have either adopted the NGSS or a state-specific version of the standards. While the goal of the standards is to identify the knowledge, skills, and practices necessary to be successful and competitive in a 21st century post-school reality, the standards themselves do not offer tools, supports, or methodologies for effectively teaching science to students. Consequently, the movement towards an increased accountability for performance in science, coupled with science standards that dictate the “what” but not the “how,” presents a challenge to both teachers and students, including students with EBD.

\section*{SCIENCE AND EBD}

Many students (with and without disabilities) struggle specifically in their science classes (Millar, 1991; Myers \& Fouts, 1992). Students with EBD, however, have additional challenges that require specific evidence-based practices and supports in order to learn, retain, and generalize science content and practices (Watt, Therrien, \& Kaldenberg, 2014). Science proficiency encompasses the knowledge of science content (e.g., characteristics of rocks), the ability to engage in scientific practices (e.g., asking questions, collecting data), and the ability to apply this knowledge to real world situations. For example, the content and practices developed in a science class can lead to an increased understanding of the natural world as well as the ability to investigate questions and solve problems. However, designing effective academic interventions for science instruction for students with EBD can be complex. A variety of factors impact student success in the science classroom, including (a) reading and writing (Parmar, Deluca \& Janzak, 1994; Sheparad \& Adjogah, 1994; Steel, 2004), (b) math skills (Olson \& Platt, 2004), (c) the ability to relate prior knowledge to new material, (d) academic motivation (Scruggs \& Mastropieri, 2000), and (e) attentiveness (Steele, 2004). Students with disabilities, including EBD, often have difficulty with these particular skills (Mercer \& Mercer, 2005). For example, a teacher may assign a research project that requires students to comprehend expository text, create outlines, conduct research and locate references, interpret visual information (including graphs or raw data), develop reports, and communicate findings with others. It can be difficult for teachers to provide the instruction and supports that ensure all students, including students with EBD, can benefit from learning science content and practices.

\section*{ACADEMIC INTERVENTIONS FOR STUDENTS WITH EBD}
	
 The records of legislative history on special education are expansive and comprehensive. Despite this, it is clear that the academic needs of children with EBD are not being met(see Wehby, Lane, \& Falk, 2003 for a discussion). Even with considerable progress around special education policies and services, students in the current public school systems are often not receiving the full range of academic accommodations that they should. However, finding the most appropriate interventions is a complex task. The modern challenge is that in addition to developing appropriate behavioral interventions, there is a need to develop supports for both daily academic instruction (including science instruction) and high stakes assessments. 
	
 Students with EBD need a wide variety of academic and behavioral programs, services, and supports to succeed (Wagner et al., 2005). These include a structured teaching environment, including the provision of explicit, systematic, and highly interactive direct instruction delivered in learner-friendly, memorable ways (Boudah, Lenz, Bulgren, Schumaker, \& Deshler, 2000). Additionally, supports include (a) independent learning strategies (Deshler, Ellis, \& Lenz, 1996); (b) opportunities for peer-mediated learning, including class-wide and reciprocal peer tutoring (King-Sears \& Cummings, 1996; Wright, Cavanaugh, Sainato, \& Heward, 1995), as well as cooperative learning (Putnam, Spiegel, \& Bruininks, 1995); and (c) teachers with a strong repertoire of behavior-management skills to decrease inappropriate behaviors and increase pro-social behaviors (Landrum, Tankersley, \& Kauffman, 2003; Walker et al., 1998). 

\section*{BEHAVIORAL AND SOCIAL INTERVEN-\\TIONS FOR STUDENTS WITH EBD}

Various teacher-led interventions to support students with EBD have been established and analyzed for effectiveness (Losinski, Maag, Katsiyannis, \& Ennis, 2014; Pierce, Reid, \& Epstein, 2004), however, the importance of developing meaningful interpersonal relationships should be underscored (Mihalas, Morse, Allsopp, \& McHatton, 2009). For example, ensuring that at least one staff or faculty member in the school is checking in with each student (Check In/Check Out) may improve the referral process so that these students receive services they need—services that may mitigate further internalizing and externalizing challenges (Cheney et al., 2009). The establishment of adult-student relationships can have an important impact, not only on socioemotional development (e.g., internalizing and externalizing behaviors; Murray \& Murray, 2004) but also on academic success (e.g., dropout rates; Muller, 2001). When considered in the context of science instruction, behavioral interventions might consider the collaborative peer-based contexts of small or large group science lessons. While behavioral interventions have been found to be effective in reducing undesirable behaviors and promoting prosocial or on-task behaviors (Gresham, 2015), it is clear that these interventions alone are not enough to promote academic success; typically, even when behavioral interventions are in place, a lack of supportive and accessible curriculum in science leads to an increase of both academic and behavioral challenges in the classroom.

\section*{DEVELOPING CURRICULUM}
	
To support academic success in the science classroom, an effective framework for academic interventions in science for students with EBD must include both behavioral interventions and academic interventions. Just as importantly, the structure of how to plan and develop science curriculum must be deliberate and organized. The following is an adapted step-by-step guide (Lenz \& Schumaker, 1999) for supporting students with EBD in the science classroom while simultaneously preparing students for high-stakes assessments. This guide can be useful in assisting general and special education administrators, teachers, and support staff when working with youth with EBD. Consider the original case study about Mr. Kelty and his task of designing effective science instruction for his students. For Mr. Kelty, and all other teachers of students with EBD, we offer this four-step process.

\subsection*{Step 1. Identify Relevant Academic and Behavioral Strengths and Areas of Need}

The first step consists of gathering information about students and existing curricular programs, if applicable. Teachers can begin by identifying and understanding the grade level science standards (content and practices) and specific interests of the students. The goal is to identify areas of overlap or opportunities to embed student interests in the grade level content standards. Consider administering a “science interests inventory” at the beginning of the school year or several times per year to gauge interest in possible topics. Add real-world examples related to employment options (e.g., career opportunities in STEM-related fields) or leisure or recreational activities (e.g., gaming, cooking, camping) that relate to science content or practices. Promote engagement by developing a class library of expository texts at the students’ reading level that relate to the topics identified in the interests inventory. Finally, add a place in the classroom for students to pose questions about the natural world or problems they would like to solve. Offer incentives or points related to a class-wide token economy system for students who offer plausible solutions or research answers. By identifying specific strengths and interests and promoting a culture of scientific discovery, students will be more likely to engage in the material with which they are provided.

Next, evaluate the materials available for teaching science. If a district or school site has adopted a certain commercial program for teaching secondary science, evaluate the materials and instructional formats. Features to consider include:
\begin{itemize}
    \item \textit{Teacher preparation}: Does the curriculum include measurable, observable lesson goals? Does the instruction follow a logical and systematic sequence across lessons and units? Does the program include materials that promote active student engagement and response? Does the program include materials that increase comprehension of core concepts (including the use of images, illustrative diagrams, and graphic organizers)?
    \item \textit{Teacher input}: Does instruction include opportunities to access or provide related knowledge? Does the lesson begin with a review of vocabulary or concepts already learned? Does the lesson include explicit instruction in new vocabulary or concepts? Does the lesson include clear models of science practices, such as posing questions or evaluating data?
    \item \textit{Student input}: Do students have access to expository texts and other print materials at their instructional level? Do students have multiple and frequent opportunities to express what they know? Do students have opportunities to work collaboratively with others? Are engagement features embedded in the curricular program?
\end{itemize}

Identifying instructional barriers is an important step in helping to promote time-on-task, comprehension of content and practices, and overall engagement in science. Many behavioral challenges may be mitigated by (a) identifying personally relevant or high-interest science content and (b) identifying barriers or omissions in existing instruction. The final part of Step 1 is to identify student-specific behavioral challenges that can be addressed as part of the instructional design process. This may include a careful review of behavior intervention plans, talking with the students’ families or other teachers and related services providers, and talking with the student directly. 

\subsection*{Step 2. Set Goals for Curricular Development}

Students with EBD have specific needs. In order to meet those needs, it is likely staff will need support and continued professional development (e.g., training in the NGSS, trainings for supporting the needs of students who are EBD, trainings about evidence-based practices for secondary students). At this stage, it is important to develop an action plan for how to proceed. Concurrently, it is vital to clearly consider the scope of (a) what adaptations need to occur (see Step 1) and (b) identify who will be responsible for making these adaptations. 

To complete the work of designing, adapting, or modifying effective science curriculum and preparing materials for implementation in the science classroom, it is recommended that initial meetings be conducted with both site administrators as well as various faculty of the school (e.g., general and special education teachers and behavior specialists, as needed). During the initial meeting, the special education teacher can share analysis of the needs of the students, including the findings from the evaluation of any existing curricular materials. The team should develop a plan for providing the science teacher (or co-teaching pair) with the necessary materials to remove barriers and effectively meet the needs of the students. The science teacher should discuss the need for students to access materials that allow them to access grade-level content (such as expository texts that match both the grade level content standards and their reading level) and hands-on materials to promote both learning and engagement. This may also include advocating for additional time to plan or develop materials. This process should be collaborative and constructive, with the teacher agreeing for ongoing check-ins from site administration and faculty on the progress and success of instructional design and implementation. 

It is important to articulate clear goals for curricular adaptations and design. Within this step, it is recommended to address the areas where students are struggling. Three goals for the teacher should be established: (a) to craft or adapt an age appropriate curriculum that directly addresses the needs of students with EBD; (b) to ensure the curriculum aligns with science standards (e.g. NGSS); and (c) to provide high stakes assessment preparation materials to prepare students for the state end-of-year exam (e.g., the end of year state biology exam).

\subsection*{Step 3. Design and/or Adapt Instruction}

After developing a collaborative plan for constructing appropriate science curriculum, teachers are ready to design or adapt instruction. First, prioritize grade-level content that matches the interests of students. Consider both science content (the domains of science, including physical science, Earth and space science, life science, and engineering and technology) and science practices (the “doing of science,” including asking questions, collecting and analyzing data, etc.). Ensuring subject matter is well aligned with grade-level Standards promotes learning and engagement and prepares students for high-stakes science assessments. Second, the presentation and delivery of the content should be adapted or modified to meet the specific needs of the students (Wagner et al., 2006). Following recommendations from Lenz and Schumaker (1999), content should be adapted to connect student learning to instructional materials and content adaptations should be made to meet current local and state education standards, as well as to meet students’ individualized education programs (IEPs). This may include addressing specific academic goals that may relate to science instruction, including decoding, reading comprehension, or mathematical computations or reasoning. Similarly, students may have goals related to time-on-task or work completion. Third, to ensure students are supported and prepared for end-of-grade assessments, teachers should carefully identify any accommodations or modifications stated on the IEP and embed them in both science instruction and assessment. 

Specifically, research and evidence-based interventions for supporting math, language/vocabu\-lary, and test taking strategies should be embedded throughout the curriculum. Special consideration should be made for communication or language needs. For example, if an identified barrier (in Step 1) is decoding, teachers can plan to provide expository texts in e-text formats, that allow for text-to-speech capabilities. Similarly, the use of graphic organizers and sentences starters or word banks can provide supports for students who have difficulty expressing what they know in writing. Additional examples of strategies for improving language and vocabulary skills for students with EBD include: 
\begin{itemize}
    \item \textit{Development of a word wall through guided reading} (Schulman \& daCruz Payne, 2000). After providing word/vocabulary lists, it is recommended that words (once taught) should be depicted on a large chart or taped on a wall for the entire class to see. Additionally, students may make an activity out of the word wall – such as creating key-word mnemonics (a vocabulary learning strategy used to teach content specific vocabulary, using words and pictures already familiar with the learner; King-Sears, Mercer, \& Sindelar, 1992; Mastropieri, Scruggs, Whittaker, \& Bakken, 1994). This activity would not only improve the state required vocabulary knowledge and context, it would improve students’ motivation and participation in their own learning.
    \item \textit{Incremental rehearsal} is an academic intervention that is used to address issues of poor retention and increase fluency of retrieval of basic facts (Joseph, 2006). Flash cards and vocabulary review activities might help the teacher practice terms with students.
\end{itemize}

While a teacher may choose to design novel science instruction that is individualized and incorporates research and evidence-based practice, the practice of making curricular adaptations to existing material is most likely the most efficient and parsimonious route. One approach to making curricular adaptations would be to use previously created lessons and curriculum materials that are based heavily on literacy (e.g., incorporate language and expository texts), and adapt the lesson using strategies by Carter et al. (2005), Mastropieri and Scruggs (1987), and Bulgren (2004). The adaptations would be to increase scaffolding of the concepts, theories, and vocabulary in order to build the knowledge and skills required to learn science. For example, preteaching science vocabulary related to the content or practices will prime students and promote success. Teaching the concepts of complex vocabulary or scientific concepts through example and non-example training (e.g., show multiple examples of concepts to clearly demonstrate key features of a target concept) ensures students can use related vocabulary to communicate their understanding of key concepts. These types of adaptations can be incorporated into an existing lesson that simply tells teachers to “teach xxxx vocabulary” or “read xxxx text.” Collaborative reading strategies, like Peer Assisted Learning Strategy (Simmons, Fuchs, \& Fuchs, 1995) or Collaborative Strategic Reading (Klingner, Vaughn, Dimino, Schumm, \& Bryant, 2001), can be embedded in lessons that rely heavily on expository text decoding, fluency, or comprehension. Finally, Self-Regulated Strategy Development (SRSD; Harris, Graham, MacArthur, Reid, \& Mason, 2011) is an approach that incorporates mnemonics to explicitly teach students with EBD to develop their writing and self-regulation strategies across content areas.

Finally, plan to support behavioral needs by incorporating individual and class-wide supports. For students who respond well to adult attention, a Check In/Check Out system may increase time-on-task and work completion. In this system, students receive contingent and specific feedback and reinforcement for performing specific tasks or behaviors. If students have clear rules for earning points by engaging in science lessons or remaining on-task during science lessons, they may be more likely to participate in the lesson. Students can earn points towards a class-wide cash out for demonstrating appropriate collaborative behaviors during small group activities, for example. Additionally, teachers should consider environmental factors that promote success, including clear classroom procedures and rules (including safety rules for hands-on science demonstrations), a physical environment that promotes collaboration and allows for teachers to monitor and circulate effectively, and clearly posted schedules and lesson objectives. 

\subsection*{Step 4. Implement, Evaluate, and Modify Adaptations}

Once science curricular materials have been appropriately adapted or developed, teachers must consider the instruction itself. On the one hand, students with disabilities, including students with EBD, benefit from systematic, direct, and explicit instruction. On the other hand, learning about both the content and practices of science necessitates flexibility and discovery. A careful combination of these paradigms is one way to nurture wonder and also promote successful acquisition of knowledge and skills for students with EBD. Explicit and systematic approaches may be necessary for teaching foundational knowledge, such as vocabulary acquisition or steps to a scientific process. Observational and Experiential Learning (Mastropieri \& Scruggs, 1987) could promote higher order thinking and a deeper interest in the content matter, for example. Methods for providing experiential learning could be through media presentations, outside activities, and activities in the classroom. For example, a teacher could pretrain the steps to the science investigation (to practice the scientific method) through a video model, and then students could create and engage in their own investigations inside and outside the classroom. Scaffolds would be used to support learning and motivation throughout the process. Within the lesson, each distinct part of the project would be clearly described, and students would be assessed with a project checklist as a visual support and mechanism for self-monitoring throughout the project activities.

To ensure long-lasting student success, the process of curricular development should be iterative and reflective. Teachers should plan to routinely evaluate their own curriculum and instruction and make adjustments as needed. The evaluation and adjustment of the adaptation of the materials would be threefold. First, after establishing teacher needs, a binder should be created that includes: (a) standard-specified science materials organized by disciplinary core ideas (e.g., required biology vocabulary for students, biology learning standards, sample quizzes, and review materials); (b) lessons, activities, support materials, assessments, and review materials for the units along with a curriculum map for a course in its entirety; and (c) additional high stakes assessment preparation materials that aligned to the corresponding lessons (e.g., practice prompts, word banks of grade-level vocabulary both within and across units).  Concurrently, supplemental textbooks, expository texts, and online resources may be added to assist in providing students with additional information, lessons, and activities that add to the lessons.

Second, to evaluate the effectiveness of the curriculum, pre- and post-test data on student performance should be collected. It is recommended that teachers measure both foundational knowledge (i.e., knowledge of content vocabulary and key concepts) as well as proficiency in scientific practices (e.g., using a rubric, to what extent can students use inquiry to answer a question? To what extent can students use the engineering problem solving process to solve problems?). Third, students may receive daily scores for both academic and behavior using both teacher and self-scored evaluations. 

\section*{IMPLICATIONS AND CONCLUSIONS}

Designing and adapting science lessons can (a) increase motivation by incorporating a high level of hands-on, project-based activities; (b) increase understanding of science content (e.g., facts and concepts related to the science content) and practices (e.g., problem solving, inquiry); and (c) support integrated learning across disciplines to support overall outcomes and performance on high stakes assessments. To effectively teach science to students with EBD, one requires finding both the appropriate behavioral and academic interventions that address specific student needs. Additionally, because high quality science education necessitates integrated understanding of other disciplines (e.g., math, reading, writing), science educators must be well-versed in how to teach and support other content areas as well. 

While there are many challenges administrators and teachers have in developing organized and quality science lessons for students with emotional and behavioral disabilities, it is imperative to be thoughtful and intentional when developing or adapting science instruction that meets the needs of students with EBD. By following the process identified in this guide, teachers have the power to provide students with EBD the supports and instruction necessary for both the learning and doing of science.
 
\section*{ACKNOWLEDGEMENTS}
Thank you to Drs. Jean Schumaker and Keith Lenz for permission to use their process and steps as part of this manuscript.

\end{large}
\clearpage
\section*{REFERENCES}\par 

\leftskip 0.25in
\parindent -0.25in 

Boudah, D. J., Lenz, B. K., Bulgren, J. A., Schumaker, J. B., \& Deshler, D. D. (2000). Don't water down! Enhance content learning through the unit organizer routine. \textit{Teaching Exceptional Children, 32}(3), 48-56.

Bulgren, J. (2004), Effective content-area instruction for all students. In Thomas E. Scruggs, Margo A. Mastropieri (Eds.) \textit{Research in Secondary Schools (Advances in Learning and Behavioral Disabilities, Volume 17)} (pp.147 – 174). Emerald Group Publishing Limited.

Carter, E. W., Wehby, J., Hughes, C., Johnson, S. M., Plank, D. R., Barton-Arwood, S. M., \& Lunsford, L. B. (2005). Preparing adolescents with high-incidence disabilities for high-stakes testing with strategy instruction. \textit{Preventing School Failure, 49}, 55-62.

Cheney, D. A., Stage, S. A., Hawken, L. S., Lynass, L., Mielenz, C., \& Waugh, M. (2009). A 2-year outcome study of the check, connect, and expect intervention for students at risk for severe behavior problems. \textit{Journal of Emotional and Behavioral Disorders, 17}, 226-243.

Deshler, D. D., Ellis, E. S., \& Lenz, B. K. (1996). \textit{Teaching adolescents with learning disabilities} (2nd Ed.). Denver, CO: Love Publishing.

Gresham, F. (2015). Evidence-based social skills interventions for students at risk for EBD. \textit{Remedial and Special Education, 36}, 100-104.

Harris, K. R., Graham, S., MacArthur, C. A., Reid, R., \& Mason, L. H. (2011). Self-regulated learning processes and children’s writing. In B. J. Z. D. H. Schunk (Ed.), \textit{Handbook of self-regulation of learning and performance} (pp. 187-202). New York, NY: Rouledge/Taylor \& Francis Group. 

Joseph, L. M. (2006). Incremental rehearsal: A flashcard drill technique for increasing retention of reading words. \textit{The Reading Teacher, 59}, 803-807.	

King-Sears, M. E., \& Cummings, C. S. (1996). Inclusive practices of classroom teachers. \textit{Remedial and Special Education, 17}, 217-225.

King-Sears, M. E., Mercer, C. D., \& Sindelar, P. T. (1992). Toward independence With keyword mnemonics a strategy for science vocabulary instruction. \textit{Remedial and Special Education, 13}, 22-33.

Klingner, J. K., Vaugh, S., Dimino, J., Schumm, J. S., \& Bryant, D. (2001). \textit{Collaborative strategic reading: Strategies for improving comprehension}. Longmont, CO: Sopris West.

Landrum, T. J., Tankersley, M., \& Kauffman, J. M. (2003). What is special about special education for students with emotional or behavioral disorders? \textit{The Journal of Special Education, 37}, 148-156.

Lenz, K., \& Schumaker, J. (1999). \textit{Adapting language arts, social studies, and science materials for the inclusive classroom}. Reston, VA: The Council for Exceptional Children.

Losinski, M., Maag, J. W., Katsiyannis, A., \& Ennis, R. P. (2014). Examining the effects and quality of interventions based on the assessment of contextual variables: A meta-analysis. \textit{Exceptional Children, 80}, 407-422.

Mastropieri, M. A., \& Scruggs, T. E. (1987). \textit{Effective instruction for special education}. Boston, MA: Little, Brown.

Mastropieri, M. A., Scruggs, T. E., Whittaker, M. E., \& Bakken, J. P. (1994). Applications of mnemonic strategies with students with mild mental disabilities. \textit{Remedial and Special Education, 15}, 34-43.

McFarland, J., Hussar, B., Wang, X., Zhang, J., Wang, K., Rathbun, A., … Bullock Mann, F. (2018).  \textit{The condition of education 2018} (NCES 2018-144). Washington, DC: National Center for Education Statistics. Retrieved from \url{https://nces.ed.gov/pubsearch/pubsinfo.asp?pubid=2018144}.

Mercer, C. D., \& A. R. Mercer. 2005. \textit{Teaching students with learning problems}. Upper Saddle River, NJ: Pearson Merrill Prentice Hall. 

Mihalas, S., Morse, W. C., Allsopp, D. H., \& McHatton, P. A. (2009).  Cultivating caring relationships between teachers and secondary students with emotional and behavioral disorders: Implications for research and practice.  \textit{Remedial and Special Education, 30}, 108-125. 

Millar, R. (1991). Why is science hard to learn? \textit{Journal of Computer Assisted Learning, 7}, 66-74.

Muller, C. (2001). The role of caring in the teacher-student relationships: Promoting the social and emotional health of early adolescents with high incidence disabilities. \textit{Childhood Education, 78}, 285-290. 

Murray, C., \& Murray, K. M. (2004). Child level correlates of teacher-student relationships: An examination of demographic characteristics, academic orientations, and behavioral orientations. \textit{Psychology in the Schools, 41}, 751-762. 

Myers III, R. E., \& Fouts, J. T. (1992). A cluster analysis of high school science classroom environments and attitude toward science. \textit{Journal of Research in Science Teaching, 29}, 929-937. 

NGSS Lead States. (2013). \textit{Next generation science standards: For states, by states}. Washington, DC: The National Academies Press.

Parmar, R. S., Deluca, C. B., \& Janczak, T. M. (1994). Investigations into the relationship between science and language abilities of students with mild disabilities. \textit{Remedial and Special Education, 15}, 117-126.

Pierce, C. D., Reid, R., \& Epstein, M. H. (2004). Teacher-mediated interventions for children with EBD and their academic outcomes: A review. \textit{Remedial and Special Education, 25}, 175-188.

Putnam, J. W., Spiegel, A. N., \& Bruininks, R. H. (1995). Future directions in education and inclusion of students with disabilities: A Delphi investigation. \textit{Exceptional Children, 61}, 553-576.
	
Schulman, M. B., \& daCruz Payne, C. (2000). \textit{Guided reading: Making it work}. Chicago, IL: Scholastic Inc.	

Scruggs, T. E., \& Mastropieri, M. A. (2000). The effectiveness of mnemonic Instruction for students with learning and behavior problems: An update and research synthesis. \textit{Journal of Behavioral Education, 10}, 163-173.

Shepard, T., \& Adjogah, S. (1994). Science performance of students with learning disabilities on language-based measures. \textit{Learning Disabilities Research \& Practice, 9}, 219-225.

Simmons, D. C., Fuchs, L., \& Fuchs, D. (1995). Effects of explicit teaching and peer-mediated instruction on the reading achievement of learning disabled and low-performing students. \textit{Elementary School Journal, 95}, 387-407.

Steele, M. M. (2004). Teaching science to students with learning problems in the elementary classroom. \textit{Preventing School Failure, 49}, 19.

Wagner, M., Friend, M., Bursuck, W. D., Kutash, K., Duchnowski, A. J., Sumi, W. C., \& Epstein, M. H. (2006). Educating students with emotional disturbances: A national perspective on school programs and services. \textit{Journal of Emotional and behavioral Disorders, 14}, 12-30.

Wagner, M., Kutash, K., Duchnowski, A. J., Epstein, M. H., \& Sumi, W. C. (2005). The children and youth we serve: National picture of the characteristics of students with emotional disturbances receiving special education. \textit{Journal of Emotional and Behavioral Disorders, 13}, 79-96.

Wagner, M., \& Newman, L. (2012). Longitudinal transition outcomes of youth with emotional disturbances. \textit{Psychiatric Rehabilitation Journal, 35}, 199-208. 

Walker, H. M., Kavanagh, K., Stiller, B., Golly, A., Severson, H. H., \& Feil, E. G. (1998). First step to success: An early intervention approach for preventing school antisocial behavior. \textit{Journal of Emotional and Behavioral Disorders, 6}, 66-80.

Watt, S. J., Therrien, W. J., \& Kaldenberg, E. R. (2014). Meeting the diverse needs of students with EBD in inclusive science classrooms. \textit{Beyond Behavior, 23}, 14-19. 

Wehby, J. H., Lane, K. L., \& Falk, K. B. (2003). Academic instruction for students with emotional and behavioral disorders. \textit{Journal of Emotional and Behavioral Disorders, 11}, 194-197.

Wright, J. E., Cavanaugh, R. A., Sainato, D. M., \& Heward, W. L. (1995). Somos todos ayudantes \& estudiantes: A demonstration of a classwide peer tutoring program in a modified Spanish class for secondary students identified as learning disabled or academically at-risk. \textit{Education and Treatment of Children}, 18, 33-52.	

Wynne, M. E., Ausikaitis, A. E., \& Satchwell, M. (2013). Adult outcomes for children and adolescents with EBD: Understanding parents’ perspectives. \textit{Sage Open, 3}, 1-14.

\end{document}