\documentclass[11.5pt]{sig-alternate} % sets document style to sig-alternate
% packages
% typesetting
%\usepackage{dirtytalk} % typset quotations easier (\say{stuff})
\usepackage{hanging} % hanging paragraphs
\usepackage[defaultlines=3,all]{nowidow} % avoid widows
\usepackage[pdfpagelabels=false]{hyperref} % produce hypertext links, includes backref and nameref
\usepackage{xurl} % defines url linebreaks, loads url package
\usepackage{microtype}
%\usepackage[super]{nth} % easily create superscript ordinal numbers with \nth{x}
\usepackage{textcomp}
\newcommand{\texttildemid}{\raisebox{0.4ex}{\texttildelow}}
% layout
%\usepackage{enumitem} % control layout of itemize, enumerate, description
\usepackage{fancyhdr} % control page headers and footers
\usepackage{float} % improved interface for floating objects
%\usepackage{multicol} % intermix single and multiple column pages
% language
\usepackage[utf8]{inputenc} % accept different input encodings
\usepackage[english]{babel} % multilanguage support
% misc
\usepackage{graphicx} % builds upon graphics package, \includegraphics
%\usepackage{lastpage} % reference number of pages
%\usepackage{comment} % exclude portions of text (?)
\usepackage{xcolor} % color extensions
\usepackage[backend=biber, style=apa]{biblatex} % sophisticated bibliographies % necessary for HTML to display author info and date on abstract page
\usepackage{csquotes} % advanced quotations, makes biblatex happy
\usepackage{authblk} % support for footnote style author/affiliation
% tables and figures
\usepackage{tabularray}
%\usepackage{array} % extend array and tabular environments
\usepackage{caption} % customize captions in figures and tables (rotating captions, sideways captions, etc)
%\usepackage{cuted} % allow mixing of \onecolumn and \twocolumn on same page
\usepackage{multirow} % create tabular cells spanning multiple rows
%\usepackage{subfigure} % deprecated, support for manipulation of small figures
%\usepackage{tabularx} % extension of tabular with column designator "x", creates paragraph-like column whose width automatically expands
%\usepackage{wrapfig} % allows figures or tables to have text wrapped around them
%\usepackage{booktabs} % better rules
% dummy text
%\usepackage{blindtext} % blind text dummy text
%\usepackage{kantlipsum} % Kant style dummy text
\usepackage{lipsum} %lorem ipsum dummy text
% other helpful packages may be booktabs, longtable, longtabu, microtype

\pagestyle{fancy} % sets pagestyle to fancy for fancy headers and footers

% header and footer
% modern way to set header image
\renewcommand{\headrulewidth}{0pt} % defines thickness of line under header
\renewcommand{\footrulewidth}{0pt} % defines thickness of line above header
\setlength\headheight{80.0pt} % sets height between top margin and header image, effectively moves page contents down
\addtolength{\textheight}{-80.0pt} % seems to affect the lower height. maybe only works properly if footer numbers enabled?
\fancyhf{}
\fancyhead[CE, CO]{\includegraphics[width=\textwidth]{headerImage.png}}
% footer
%\fancyfoot[LE,LO]{Article Title Here \\ DOI: }% left footer article title and doi
%\fancyfoot[CE,CO]{{}} % center footer empty
%\fancyfoot[RE,RO]{\thepage} % right footer page numbers
%\pagenumbering{arabic} % arabic (1, 2, 3) numbering in footer

\hypersetup{colorlinks=true,urlcolor=blue} % sets link color to blue
\urlstyle{same} % sets url typeface to same as rest of text

% set caption and figure to italics, label bold, left align captions, does not transfer to HTML
\captionsetup{labelfont=bf, font={large, it}, justification=raggedright, singlelinecheck=false}
\renewcommand\theContinuedFloat{\alph{ContinuedFloat}}

%this next bit is confusing, but essentially changes the width of the abstract. Seems to have been copied from this https://tex.stackexchange.com/questions/151583/how-to-adjust-the-width-of-abstract
\let\oldabstract\abstract
\let\oldendabstract\endabstract
\makeatletter %changes @ catcode to enable modification (in parsep)
\renewenvironment{abstract} %alters the abstract environment
{\renewenvironment{quotation}%
               {\list{}{\addtolength{\leftmargin}{1em} % change this value to add or remove length to the the default ?
                        \listparindent 1.5em%
                        \itemindent    \listparindent%
                        \rightmargin   \leftmargin%
                        \parsep        \z@ \@plus\p@}%
                \item\relax}%
               {\endlist}%
\oldabstract}
{\oldendabstract}
\makeatother %changes @ catcode to disable modification

% checks
% italics - 
% links -
% dashes -
% tildes -
\begin{document}

\title{Using Argument-based Science Inquiry to Improve Science Achievement for Students with Disabilities in Inclusive Classrooms}

\author[1]{\large \color{blue}Jonte’ C. Taylor}
\author[2]{\large \color{blue}Ching-mei Tseng}
\author[1]{\large \color{blue}Angelique Murillo}
\author[3]{\large \color{blue}William Therrien}
\author[4]{\large \color{blue}Brian Hand}

\affil[1]{Pennsylvania State University}
\affil[2]{National Chi Nan University}
\affil[3]{University of Virginia}
\affil[4]{University of Iowa}

\toappear{}
%% ABSTRACT
\maketitle
\begin{@twocolumnfalse} 
\begin{abstract}
\item 
\textit{The increased emphasis on STEM related careers and the use of science in everyday life makes learning science content and concepts critical for all students including those with disabilities. Further, the National Resource Council (NRC, 2012), indicate that more emphasis is being placed on being able to critically think about science concepts in and outside of the classroom.  Unfortunately, students with disabilities perform well below their non-disabled peers on science achievement assessments.  While there may be a number of factors that contribute to poor performance of students with disabilities in science classrooms, one consideration is instruction methodology.  The NRC (2012) and the Next Generation Science Standards (NGSS Lead State, 2013) task teachers and students to better understand how science is connected to the everyday world through the use of inquiry-based methods.  Research suggest that students with disabilities can learn science content via inquiry-based instruction; however, learning occurs best when instructional supports are provided with inquiry (Rizzo \& Taylor, 2016; Scruggs, \& Mastropieri, 2007; Scruggs, Mastropieri, Bakken, \& Brigham, 1993; Taylor et al., 2011).  The current study examines the effects of an argument-based supported inquiry approach (Science Writing Heuristic, SWH; Hand \& Keys, 1999) on science achievement for students with disabilities.  Data were collected for third, fourth and, fifth grade students with disabilities in treatment and comparison groups.  The results indicate that students in the SWH group scored significantly better than the students in the comparison group on post-test science achievement scores and authors display stronger effect size results as well.  Implications for teaching science to students with disabilities are discussed. 
}
\\ \\
Keywords: science education, disabilities, science achievement, special education, science inquiry, STEM
\end{abstract}
\end{@twocolumnfalse}

%% AUTHOR INFORMATION

\textbf{*Corresponding Author, Jonte’ C. Taylor}\\
\href{mailto: jct215@psu.edu }{(jct215@psu.edu)} \\
\textit{Submitted Jan 3 2016 }\\
\textit{Accepted Dec 12 2017} \\
\textit{Published online February 26th, 2018} \\
\textit{DOI:10.14448/jsesd.10.0001} \\
\pagebreak
\clearpage
\begin{large}
\section*{INTRODUCTION}
Science achievement for students in the U.S. has long been a source of concern for educational stakeholders including teachers, administrators, educational policy makers, and those who understand that poor science performance has negative implications for the future of the country. Villanueva (2010) highlighted the importance of science achievement as a means of advancement for industrial and economic success.  These concerns were even voiced from the highest office in the land.  Former president Obama mentioned the impact of science achievement on the nation’s future economic prospects during a State of the Union speech (White House, Office of the Press Secretary, 2011).  Terrell (2007) noted that according to the United States Department of Labor, jobs related to science, technology, engineering, and math (STEM) fields will grow tremendously and even low paying STEM jobs will be viable options for livable wages.  The growth in STEM careers makes it essential for all students to get a solid foundation in STEM education.  

As there is little debate regarding the importance of STEM education, what that looks like in educational settings varies greatly.  Science instruction, particularly at the elementary level is especially fraught with inconsistency.  For students with disabilities (SWDs), uneven science instruction in early years results in poorer science achievement performance over time (Therrien, Taylor, Hosp, Kaldenberg, \& Gorsh, 2011).  Predictably, SWDs have consistently scored significantly lower than their peers without disabilities over the course of multiple years and in multiple grades every year on national standardized science assessments.

\subsection*{National Science Achievement for Students with Disabilities}

Examining national assessments gives a much more nuanced view of science performance for SWDs.  According to the Nation’s Report Card in Science for 2015 (National Center for Education Statistics, NCES, 2015), 66\% of 8th grade SWDs scored at the below basic level on the 2015 National Assessment of Educational Progress (NAEP) as compared to 28\% of 8th grade SWDs.  Basic level proficiency “denotes partial mastery of prerequisite knowledge and skills that are fundamental for proficient work at each grade” (NCES, 2012, p. 14).  Subsequently, those same students with disabilities had mean scale score that was 34 points (significantly) lower than their non-disabled peers. Some of the reasons for SWDs poor performance in science achievement can be partially traced to how science instruction occurs.
 
\subsection*{Classroom Science Instruction for Students with Disabilities}

Science instruction can often be heavily laden with and connected to language and terminology acquisition.  Additionally, general education classroom science instruction was conducted with emphasis placed on textbook reading and lecture style presentations (Scruggs, Mastropieri, Bakken, \& Brigham, 1993; Steele, 2004; Therrien, Taylor, Watt, Kaldenberg, 2014).  Science instruction that focuses on students learning via textbook or on lecture style presentations are problematic for students with disabilities in inclusive science classrooms as they tend to work against their strengths because of the heavy cognitive and focal loads required. Textbook and lecture style instruction in science is ineffective at engaging low achieving and/or SWDs.  In an effort to improve science achieve and to mirror the practices of science, inquiry-based science instruction is now considered the preferred method of instruction for all students (National Research Council, [NRC], 2012).  

The focus on improving science instruction by promoting and supporting the use of inquiry-based instruction is for moving away from textbook and lecture style teaching in an effort to better align science learning with the practices of science (Next Generation Science Standards [NGSS] Lead States, 2013). Prior to the promotion of inquiry-based instruction, science instruction consisted of teachers lecturing to their students who simply record the information and attempt to memorize it for the test.  Through that process, the students are not truly learning the material and what they do “learn” is often forgotten after they have taken the exam possibly leaving students with disabilities further behind.  The use of inquiry-based instruction has the support from the professional science community (The American Association for the Advancement of Science, AAAS), science education community (National Science Teachers Association, NSTA), and the national educational initiatives (Next Generation Science Standards, NGSS).  Further, inquiry-based science instruction research for SWDs has some evidence of success for learning science content (Rizzo \& Taylor, 2016; Scruggs \& Mastropieri, 2007; Scruggs et al., 1993).

\subsection*{Inquiry-based Science Instruction for Students with Disabilities}

Inquiry-based instruction places heavier emphasis on student-directed personal experiences in science.  That is, students have more opportunities to lead discussions, collaborate in the practices of science, and perform hands-on activities (NGSS Lead States, 2013).  The use of inquiry-based instruction is the preferred teaching method; however, it is not without complications.  Defining what inquiry-based instruction is and how it should look is not entirely clear in the research or practitioner literature (Rizzo \& Taylor, 2016; Therrien, et al., 2011; Therrien, Taylor Watt, Kaledenberg, 2014).  Various terminologies have been used in the literature to describe what can be categorized as inquiry-based instruction (e.g., inquiry-based, hands-on instruction, discovery learning). As described by Klahr and Li (2005) inquiry-based instruction may have a number of components and variables that make research vary.  

The NRC (2012) suggested that inquiry-based instruction include students learn how to: collect, use, and interpret data; make claims using evidence, and discuss science as debate to support claims with evidence from data.  Further, previous research support the use of hands-on activities as a component to inquiry-based instruction (NGSS Lead States, 2013; NRC, 2012; Therrien et al., 2014). Even with the suggestions from the NRC and NGSS, teaching practices related to inquiry-based instruction can vary widely.  Moreover, it has been suggested that inquiry approaches can and should be considered as a spectrum and categorized (Martin-Hansen, 2002; Rizzo \& Taylor, 2016; Scruggs \& Mastropieri, 1994; Scruggs \& Mastropieri, 2007).  In their review of the effectiveness of all types of inquiry-based instruction for students with disabilities, Rizzo and Taylor (2016) found that when using the inquiry framework categorized by Martin-Hansen (2002) that studies that used inquiry with more supports were more effective for science learning.

\subsection*{Conceptual Framework and Purpose of Study}

The Science Writing Heuristic (SWH, Hand \& Keys, 1999) is an argument-based science inquiry approach.  The SWH approach is designed to involve students in inquiry, argumentation, and experimentation as a means of learning science and improving critical thinking skills.  Yore, Bisanz, and Hand (2003) describe the SWH as being based on the theories that include writing-to-learn strategies (i.e., students learn about science through writing about science experiences), science literacy (i.e., understanding the content, concepts, and processes of science), and inquiry-based instruction (i.e., collecting data, making claims, testing hypotheses, and providing evidence).  The SWH approach was developed to provide students an opportunity to make broad conceptual framework connections to science knowledge through debate with peers and experimentation design while allowing multiple means of displaying content to help with deep understanding of science themes and concepts (Taylor et al., 2011).  Hand and Keys (1999) developed the SWH in a manner that requires students to use “questions, claims, and evidence” to display their understanding of science content and concepts.  The SWH provides science instruction and science learning from a student directed, teacher guided manner honing in on improving student understanding of science content knowledge, scientific processing skills, and general critical thinking skills.

While numerous studies have shown the SWH to be an effective approach to teaching science to general education students (Hand \& Norton-Meier, 2011), only one study specifically focused on the effectiveness of the SWH for students with disabilities.  Taylor et al. (2012) found that students with disabilities in SWH schools outperformed their peers in non-SWH classrooms on standardized science measures over one and multiple years.  In an effort to add to the research base regarding the SWH and students with disabilities, this study analyzes the differences in science achievement for students with disabilities with a comparison group of students with disabilities. Specifically, the researchers attempt to answer the following questions:
\begin{enumerate}
    \item Is there a significant difference in achievement on mean standardized science scores for students with disabilities in SWH classrooms when conducting pre-intervention and post-intervention comparisons?
    \item Is there a significant difference in achievement on mean standardized science scores for students with disabilities in SWH classrooms when conducting post-intervention comparisons with a comparison group?
    \item Do students with disabilities in SWH classrooms display larger effect sizes on mean standardized science scores when conducting pre-intervention and post-intervention comparisons with a comparison group?
\end{enumerate}

\section*{METHOD}

\subsection*{Participants}

Nine treatment (SWH) and nine comparisons schools with similar student population characteristics (matched pairs) participated in the study.  Treatment participants were schools a part of a research project examining the effects of a specific science instructional method on student science achievement.  The study was conducted in rural areas of a Midwest state.   Each school reflected a homogeneous group with similar student population distributions by ethnicity, SES, and special needs status.  These school districts consist of approximately 5\% minority students and 95\% white students. Gender distribution was nearly identical across the treatment and comparison groups, and was predominantly female (nearly 53\%). Third, fourth, and fifth grades students with Individualized Education Programs (IEP) and in inclusive science classrooms were included in both treatment and comparison groups.

The study had 463 students (treatment and comparison groups) with IEPs in inclusive science classrooms.  During the pre-intervention phase, the treatment group had 238 students with 225 students in the comparison group.  Due to attrition at post-intervention, there were 208 treatment students and 199 comparisons students.  The final total number of students at post-intervention was 407.  

\subsection*{Intervention}

The SWH (Hand \& Keys, 1999) is a guided argument-based approach to classroom science inquiry.  The SWH approach involves the use of inquiry, argumentation, and experimentation as a means of learning science and improving critical thinking skills.  Hand and Keys (1999) developed the SWH in a manner that requires students to use “questions, claims, and evidence” to display their understanding of science content and concepts.  The SWH provides science instruction and science learning from a student directed, teacher guided manner honing in on improving student understanding of science content knowledge, scientific processing skills, and general critical thinking skills.

The SWH mainly stresses the use of argument-based inquiry, it also incorporates a number of other intervention methods and strategies to provide students with disabilities added support at the pre-instruction, during instruction, and post instructional phases of science teaching.  Prior to instruction, teachers plan for the possibility of connecting student ideas to broad topics or “big ideas”.  During instruction, The SWH encourages teachers to use individual, small group, and whole class instruction in a manner as seamless as possible to give students time to share information and knowledge as well as reflect on their own understanding.  The SWH requires students to keep science notebooks or journals and document their experiences with learning new and different information.  Post instruction, teachers are encouraged to use a multitude of methods (multimodal representations) to have students express what they have learned during the course of the instruction in both content understanding and knowledge growth.

\subsubsection*{Teacher and student templates}
To scaffold the teaching and learning process during science instruction, the SWH uses teacher and student templates (see Table 1).  The teacher template provides teachers with a guide to help understand the activities and processes as they relate to the SWH approach. Teachers were directed to use the templates to help construction lessons, instructional activities, and to be mindful of the process that students may need to understand content and concepts.  Teachers were taught to use the templates as pedagogical support, but were also encouraged to adjust the sequence for their classroom needs.  The student template emphasizes the elements to understand the scientific process.  Using the templates, students develop questions, make claims, and provide evidence to support or reject claims.  

\begin{table*}[!htbp]
\caption{Student and Teacher Template for the Science Writing Heuristic Approach}
\begin{tabular}{lll}
\hline
 & \textbf{Student Template} & \textbf{Teacher Template} \\ \hline
\textit{Beginning Ideas} & What are my questions? & Exploration of pre-instruction understanding through individual or group concept mapping or working through a computer simulation. \\ \hline
\textit{Tests} & What did I do? & Pre-laboratory activities, including informal writing, making observations, brainstorming, and posing questions. \\ \hline
\textit{Observations} & What did I see? & Participation in laboratory activity. \\ \hline
\textit{Claims} & What can I claim? & Negotiation phase I - writing personal meanings for laboratory activity. \\ \hline
\textit{Evidence} & How do I know? How can I support my claim? & Negotiation phase II - sharing and comparing data interpretations in small groups. \\ \hline
\textit{Reading} & How do my ideas compare with others ideas? & Negotiation phase III - comparing science ideas to textbooks for other printed resources. \\ \hline
\textit{Reflection} & How have my ideas changed? & Negotiation phase IV - individual reflection and writing. \\ \hline
\textit{Writing} & What is the best explanation to describe what I have learned? & Exploration of post-instruction understanding through concept mapping, group discussion, or writing a clear explanation. \\ \hline
\end{tabular}
\end{table*}

\subsection*{Measure}

The Iowa Test of Basic Skills (ITBS) is a standardized measure that is used as the achievement test measure for the state in which the study was conducted.  The ITBS consists of a battery of achievement subtests in various broad content areas (e.g. reading comprehension, math, science, etc.) and specific content areas (e.g. scientific inquiry, life science, physical science, etc.) (Hoover, Dunbar, \& Frisbie, 2001).  ITBS is used as an accountability measure by providing student data including:  standard scale scores, percentile ranks, grade equivalent scores, and other achievement test related indices.  For the purposes of the current study, the ITBS science standard scores were examined as pre/post-measures for both treatment and comparison groups.

\subsection*{Procedure}

The treatment group teachers engaged in a 1-year SWH professional development project cycle. Teachers in the treatment group were provided instruction on how to implement the SWH approach through a week-long professional development program during the summer and follow-up professional development sessions and support throughout the school year.  The comparison group teachers were not given any information regarding the use of SWH approach and were asked to continue science instruction as they usually would.   Comparison group science instruction varied in classrooms but consisted of a combination of textbook-based instruction, lecture style instruction, and kit-based science instruction.  

At the beginning of the school year (within the first two weeks of school) at each school, teachers administered the ITBS science test.  Both treatment and comparison schools completed the pre-intervention ITBS science assessments in August of the school year.  Post-intervention assessments were completed within one month of each school’s summer dismissal.  

\subsection*{Statistical Analyses}

To answer the research questions, the authors conducted \textit{t}-test analyses with \textit{p}-values using mean scores and standard deviations and effect size analyses using Cohen’s \textit{d}.  Specifically, a paired-samples \textit{t}-test was used to compare pre-test and post-test mean standard scores of students identified as receiving IEP supports in treatment conditions (SWH classrooms) and comparison conditions (traditional science classrooms) as well as post-test comparisons between both groups.  Additionally, the authors used an effect size analysis (Cohen’s \textit{d}) to determine the comparative effectiveness of the SWH approach as measured by the pre- and post-mean scores for treatment and comparison groups’  and comparison of post-test mean performance between both groups based on performance on the ITBS science measure.  All data analyses were conducted at a .05 alpha level.

\section*{RESULTS}

The current study results include analyses of: 1) mean pre/post-intervention standard science score analysis for treatment and comparison groups (question 1); 2) mean post-intervention standard science score analysis between treatment and comparison groups (question 2); and 3) effect size comparison on pre/post-intervention standard science scores for treatment and comparison groups (question 3).  The researchers used \textit{t}-test comparisons for questions one and two and effect size analyses (Cohen’s \textit{d}) for question three. T-test scores were conducted at the .05 alpha level for significant mean differences.  As described by Cohen (1988), effect size interpretations consist of: small effect, ES = below .50; medium effect, ES = .50-.80; and large effect, ES = above .80.  

An analysis of pre-intervention test scores was conducted to determine if there was a significant difference between treatment and comparison groups prior to intervention.  Both groups had similar pre-intervention means and number of participants.  Results indicate no significant difference during pre-intervention between groups, t(462) = .153, p < .879, 95\% CI [.153, .879] (see Table 2).

\begin{table*}[th]
\caption{Pre-Intervention t-test Comparison between Treatment and Comparison Groups}
\begin{tabular}{lccccc}
\hline
Groups & \textit{n} & Mean (\textit{SD}) & 95\% CI & \textit{t} & \textit{p} \\ \hline
 &  &  & -1.5232 – 1.3032 & 0.1530 & 0.8785 \\
Treatment & 238 & 117.23 (\textit{7.33}) &  &  & \\
Control & 225 & 117.34 (\textit{8.14}) &  &  & \\ \hline
\end{tabular}
\\ \\ \textit{Note.} CI = confidence intervals; SD = standard deviation.
\end{table*}

\subsection*{Within Group Mean Analyses (Question \#1)}

The authors analyzed means and standard deviations from pre- to post-intervention phases on ITBS science standard scores for both the treatment and the comparison students.  Results from both students groups indicate growth with groups.  Treatment group students had a mean score improvement from 117.23 to 122.75.  T-test results for the treatment group was significantly different at 0.05 alpha level [t(208) = 7.7590, p < 0.001].  Comparison group student had a mean score improvement of 117.34 to 119.52.  T-test results for the comparison group indicate a significant difference from pre to post intervention [t(1199) = 2.7002, p < 0.007].  See Table 3 for complete results table.  

\subsection*{Between Groups Mean Analysis (Question \#2)}

Post-intervention analysis was conducted between treatment and comparison groups.  When comparing post-test differences only between groups, there is a significant difference at the 0.05 alpha level.  The treatment group out-performed the comparison group significantly as indicated by the t-test results [t(407) = 4.0189, p < 0.001] (see Table 3).  

\begin{table*}[th]
\caption{Within and Between Pre/Post t-test Analyses for Treatment and Comparison Groups}
\begin{tabular}{lccccc}
\hline
 & \textit{n} & Mean (\textit{SD}) & 95\% CI & \textit{t} & \textit{p} \\ \hline
Within Group & & & & & \\
Treatment & & & & & \\
\cline{1-1}
Pre & 238 & 117.23 (\textit{7.33}) & & & \\
Post & 208 & 122.75 (7.68) & & & \\
 &  &  & -6.9182 to - 4.1218 & 7.7590 & < .001* \\
Control  &  &  &  &  & \\
\cline{1-1}
Pre & 225 & 117.34 (\textit{8.14}) &  &  & \\
Post & 199 & 119.52 (\textit{8.47}) &  &  & \\
 &  &  & -3.7669 to - 0.5931 & 2.7002 & .007* \\
Between Groups\textsuperscript{a} &  &  & 1.6500 - 4.8100 & 4.0189 & < .001* \\ \hline
\end{tabular}
\\ \\ \textit{Note.}  CI = confidence intervals. SD = standard deviation. *significant at the < .01 level. \textsuperscript{a} = post-test intervention analysis.
\end{table*}

\subsection*{Within and Between Groups Effect Size Analyses (Question \#3)}

Effect size analyses of the pre- and post-interven\-tion mean differences within groups and post-intervention between groups indicate improvements in achievements.  The treatment group scores resulted in a moderate effect size improvement (d = 0.740), while the comparison group scores resulted a small effect size improvement (d = 0.260) as indicated from pre- to post-test intervention mean standard scores.  Analysis of post-test mean differences between treatment and comparison groups indicate a moderate effect size improvement (d = 0.400).  See Table 4 for complete effect size results.

\begin{table}[th]
\caption{Effect Size Analyses of Treatment and Comparison Group Means}
\begin{tabular}{lcc}
\hline
 & Cohen’s \textit{d} & 95\% CI \\ \hline
Comparison 1\textsuperscript{a} & 0.740 & 1.432 - 0.044 \\
Comparison 2\textsuperscript{b} & 0.260 & 1.051 - 0.524 \\
Comparison 3\textsuperscript{c} & 0.400 & 0.386 - 1.185 \\ \hline
\end{tabular}
\\ \\ \textit{Note.} CI = confidence intervals. \textsuperscript{a} = Treatment group pre- and post-test comparison. \textsuperscript{b} = Comparison group pre- and post-test comparison. \textsuperscript{c} = Treatment and comparison group post-test comparison.
\end{table}

\section*{DISCUSSION}

This study examined the effects of the Science Writing Heuristic (SWH), an argument-based inquiry approach to teaching, on science achievement for students with disabilities.  Using the Iowa Test of Basic Skills (ITBS), comparisons were made between treatment and comparison groups on science achievement mean scores.  ITBS scores were obtained with pre/post-inter\-vention means, between group post-test means, and effect size data analyses conducted.  Results suggest that using the SWH is effective in improving science achievement for students with disabilities.  

Argumentation through knowledge construction and guided inquiry with the use of student and teacher templates are components of the SWH.  Evidence supporting the use of the SWH for SWDs includes the significant improvement from pre-intervention to post-intervention on the ITBS, the significant difference between students in SWH classrooms compared to those in traditional science classrooms, and as evidenced through higher effect size scores for the treatment group over the control group.  Coe (2002) suggests that instead of statistical significance, the use of effect sizes may answer the more practical question of the effectiveness of an intervention.  Previous research has supported the use of effect size calculations for educational purposes and decision making (Banda \& Therrien, 2008; Bernhardt, 2004).  As suggested by Scruggs, Mastropieri, and Boon (1998), to improve science learning and understanding for students with disabilities a more structured approach to inquiry is needed.  SWH students improved in science achievement by almost three quarters of a standard deviation.  Based on the three research questions this study, the SWH supports significant improvements for SWDs in science achievement when compared within group and between comparison groups.  Additionally, when evaluating the practical impact of the SWH on science achievement for SWDs, treatment students’ performance indicate bigger gains across comparisons.

\subsection*{Limitations}

There are a number limitations associated with the results of this study. First, the quality of teaching using the SWH approach by means of teacher fidelity could not be determined.  Although teachers in the treatment group received initial training, continuous support throughout the year, and corrective feedback, there authors did not have access to implementation quality on the following: a) SWH implementation; b) professional development training, or c) feedback from trainers.  The addition of teaching fidelity data would have allowed for more complex analyses and deeper understanding of the implications that the intervention may have on science achievement for SWDs.  Second, the location of the study provided limitations as well.  The student population was very homogenous with respect to racial and ethnic groups.  Third, the researchers were only able to identify students as being eligible for the study through the demographic information provided prior to taking the assessment.  This identification provided information that stated if students receive services via IEP.  There was no way to determine the eligibility of each student by disability type (e.g., students with learning disabilities, students with intellectual disabilities, student with emotional and behavior disabilities, etc.).  As such, in-depth analysis on how the intervention may have supported achievement for specific disability type was not possible.   Lastly, participant attrition may have contributed to the final results of the study.  The researchers were not fully able to identify all of the reasons for participant attrition.

\section*{CONCLUSION}

SWDs can be successful in science classrooms with the appropriate approach and supports.  The study suggests some effectiveness in using the SWH approach in teaching science content to students with disabilities.  The study results indicate that there can be significant growth in science learning can be achieved for SWDs over the course of a year.  While both the treatment and comparison groups showed significant growth from pre-test to post-test, when comparing strictly post-test differences between groups, the SWH group scored significantly better than the comparison group.  Furthermore, effect size comparisons show that using the SWH approach had a stronger effect on science achievement than the typical science instruction when compared using standardized assessment. 

Past research has shown that SWDs can be successful in SWH classrooms and in a broader context, inquiry-based learning classrooms.  Previous studies have provided evidence to support various types of supports that could help SWDs be successful in science classrooms, and particularly inquiry-based classrooms.  Some of the successful strategies and supports have included the use of supported inquiry (Taylor et al, 2012), teaching of specific science facts (Scruggs and Mastropieri, 1994), mnemonic instruction (Scruggs, Mastropieri, Berkeley, \& Marshak, 2010), and peer learning strategies (Bowman-Perrott, Greenwood, \& Tapia, 2007).

Future research involving the SWH approach should focus on using the approach in more educational settings with SWDs.  Additional settings should include other locales (i.e., outside of the Midwest) and various socioeconomic settings (i.e., urban, rural, and suburban areas). Examination of the SWH approach and its effectiveness with specific disability types should be considered as potential research area as well.  Additionally, researchers should focus on supports and strategies that enhance or assist students with disabilities in science classrooms.   The importance of understanding science content and concepts should not be understated or overlooked.  Students with disabilities can benefit from understanding science as both an avenue for employment and as a part of general life and living skills.

\section*{ACKNOWLEDGEMENTS}

The research reported here was supported by the Institute of Education Sciences, U.S. Department of Education, through Grant R305B10005 to The University of Iowa. The opinions expressed are those of the authors and do not represent views of the Institute or the U.S. Department of Education.

\end{large}
\clearpage
\section*{REFERENCES}\par 

\leftskip 0.25in
\parindent -0.25in 

Banda, D. R. \& Therrien, W. J. (2008).  Teacher’s guide to meta-analysis.  \textit{Teaching Exceptional Children, 41}(2), 66-71.

Bernhardt, V. (2004).\textit{ Data analysis for continuous school improvement}.  Eye on Education, 	Larchmont: NY.

Bowman-Perrott, L., Greenwood, C. R., \& Tapia, Y. (2007). The efficacy of CWPT used in 	secondary alternative school classrooms with small teacher/pupil ratios and students with 	emotional and behavioral disorders. \textit{Education and Treatment of Children, 30}, 65-87. doi:10.1353/etc.2007.0014

Bradley, E. L. (2009).  \textit{General education teachers' attitudes toward the inclusion of students 	with disabilities} (Master’s thesis). Available from ProQuest Dissertations and Thesis 	database. (UMI No.  1465935)

Coe, R., (2002). \textit{It’s the effect size, stupid. What effect size is and why it is important}. A paper presentation to the Annual Conference of the British Educational Research Association, England. Retrieved from \url{http://www.leeds.ac.uk/educol/documents/00002182.htm} 

Cohen, J. (1988). \textit{Statistical power analysis for the behavioral sciences} (2nd ed.). Hillsdale, NJ: Erlbaum.

Hand, B., \& Keys, C. (1999). Inquiry investigation: A new approach to laboratory reports. \textit{The Science Teacher, 66}, 27-29.

Hand, B., Norton-Meier, L., Staker, J., \& Bintz, J.(2009).  \textit{Negotiating science: The critical role 	of argument in student inquiry}.  Portsmouth,, NH:  Heinemann.

Hoover, H. D., Dunbar, S. B., \& Frisbie, D. A. (2007).  \textit{Iowa test of basic skills (Form C ) [Test 	description]}. Itasca, IL: Riverside.

Jimenez, B. A., Browder, D. M., Spooner, F., \& Dibiase, W. (2012). Inclusive Inquiry 	Science Using Peer-Mediated Embedded Instruction for Students with Moderate 	Intellectual Disability. \textit{Exceptional Children, 78}(3), 301-317

Klahr, D., \& Li, J. (2005). Cognitive research and elementary science instruction: From the 	laboratory, to the classroom, and back. \textit{Journal of Science Education and Technology, 14}, 217–238. doi:10.1007/s10956-005-4423-5

Martin-Hansen, L. (2002).  Defining inquiry: Exploring the many types of inquiry in the science	classroom.  \textit{The Science Teacher, 69}(2), 34-37.

Minner, D. D., Levy, A. J., \& Century, J. (2010). Inquiry-based science instruction—what is it and does it matter? Results from a research synthesis years 1984 to 2002. \textit{Journal of 	Research in Science Teaching, 47}(4), 474-496.

National Center for Education Statistics. (2012). \textit{The nation's report card: science 2011} (NCES 	2012-465).  Washington, DC: Institute of Education Sciences, U.S. Department of Education.

National Center for Education Statistics. (2015). Washington, DC: Institute of Education 	Sciences, U.S. Department of Education. Retrieved from \url{http://nces.ed.gov}

National Research Council (2012). \textit{A Framework for K-12 Science Education: Practices, 	Crosscutting Concepts, and Core Ideas}. Washington, DC: National Academy Press.

Next Generation Science Standards (NGSS) Lead States. (2013). Next generation 	science 	standards: For states, by states.

Norton-Meier, L., Hand, B., Hockenberry, L., \& Wise, K. (2008). \textit{Questions, claims, and 	evidence: The important place of argument in children's science writing}.  Portsmouth, 	NH:  Heinemann.

Palincsar, A. S., Magnusson, S. J., Cutter, J.,\& Vincent, M. (2002). Supporting guided-inquiry instruction. Teaching Exceptional Children, 34(3), 88–91.

Provasnik, S., Kastberg, D., Ferraro, D., Lemanski, N., Roey, S., and Jenkins, F. (2012). \textit{Highlights from TIMSS 2011: Mathematics and science achievement of U.S. fourth- 	and eighth-grade students in an international context (NCES 2013-009)}.  Washington, 	DC: National Center for Education Statistics, Institute of Education Sciences, U.S. 	Department of Education. 

Rizzo, K. L. \& Taylor, J. C. (2016).  Effects of inquiry-based instruction on science 	achievement for students with disabilities: An analysis of the literature.  \textit{Journal of 	Science Education for Students with Disabilities, 19}(1), 2.

Scruggs, T. E., \& Mastropieri, M. A. (1994). The construction of scientific knowledge by 	students with mild disabilities. \textit{Journal of Special Education, 28}, 307–321.

Scruggs, T. E., \& Mastropieri, M. A. (2007).  Science learning in special education: The case for constructed versus instructed learning. \textit{Exceptionality, 15}, 57-74. 	doi:10.1080/09362830701294144

Scruggs, T. E., Mastropieri, M. A., Bakken, J. P., \& Brigham, F. J. (1993). Reading versus 	doing: The relative effects of textbook-based and inquiry-oriented approaches to science 	learning in special education classrooms. \textit{The Journal of Special Education, 27}(1), 1-15.

Scruggs, T. E., Mastropieri, M. A., Berkeley, S. L., \& Marshak, L. (2010).  Mnemonic strategies: Evidence-based practice and practice-based evidence. \textit{Intervention In School \& Clinic, 	46}(2), 79-86. doi:10.1177/1053451210374985

Scruggs, T. E., Mastropieri, M. A., \& Boon, R. (1998). Science education for students with 	disabilities: A review of recent research. \textit{Studies in Science Education, 32}, 21–44.

Stage, E. K., Asturias, H., Cheuk, T., Daro, P. A., \& Hampton, S. B. (2013).  Opportunities and challenges in next generation standards. \textit{Science, 340}(6130), 276-277.

Steele, M. (2004). Teaching science to middle school students with learning problems. 	\textit{Preventing School Failure, 49}(1), 19–22.

Strike, K. A. (1997).  Toward a coherent constructivism.  In J. Novak (Ed.),\textit{ Proceedings of the 2nd International Seminar Misconceptions and Educational Strategies in Science and 	Mathematics, Vol. 1}. Ithaca, NY: Cornell University, 481-489.

Taylor, J. C., Therrien, W. J., Kaldenberg, E., Watt, S., Chanlen, N., \& Hand, B. (2011). 	Using an inquiry-based teaching approach to improve science outcomes for students with 	disabilities: Snapshot and longitudinal data.  \textit{Journal of Science Education for Students 	with Disabilities, 15}(1), 27-39.

Taylor, J. C., Chanlen, N., Therrien, W. J., Hand, B. (2014).  Improving critical thinking with science inquiry.  \textit{Academic Exchange Quarterly, 18}(1), 77-84. 

Terrell, N. (2007). STEM occupations: High-tech jobs for a high-tech economy. \textit{Occupational Outlook Quarterly, 51}(1), 26.

Therrien, W. J., Taylor, J. C., Hosp, J. L., Kaldenberg, E. R., \& Gorsh, J. (2011). Science instruction for students with learning disabilities: A meta-analysis. \textit{Learning Disabilities Research \& Practice, 26}, 188-203. doi:10.1111/j.1540-5826.2011.00340.x

Therrien, W. J., Taylor, J. C., Watt, S., Kaldenberg, E. (2014). Science instruction for 	students with emotional and behavioral disorders. \textit{Remedial and Special Education, 35}(1), 15-27.  doi: 10.1177/0741932513503557

Villanueva, M. G. (2010). \textit{Integrated teaching strategies model for improved scientific literacy in second-language learners}. (Unpublished doctoral dissertation). Nelson Mandela Metropolitan University, Port Elizabeth, South Africa.

White House, Office of the Press Secretary. (2011, January 25). Remarks by the president in the state of union address. Retrieved from \url{http://www.whitehouse.gov/the-press-office/2011/01/25/remarks-president-state-union-address}.

Yore, L. D., Bisanz, G. L., \& Hand, B. M. (2003). Examining the literacy component of science 	literacy: 25 years of language arts and science research.  \textit{International Journal of Science Education, 25}, 689-725.

\end{document}